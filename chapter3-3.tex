\section{The theorem of Chomsky-Schützenberger}

The Chomsky-Schützenberger theorem is one of the fundamental theorems in the
theory of formal languages, especially the context-free languages \cite{ChSch}.

In the literature, the proof of this theorem is always based on grammars.

We will give an automata-theoretic access to this theorem. We present this
theorem at this place in the book because it can be very well appended to the
section on pushdown classes (chapter III.1) from a proof technical point of
view. A similar access to this theorem has been given by Goldstine
\cite{Goldstine77} (?).

The Chomsky-Schützenberger theorem characterizes the the context-free languages
by a very simple subclass of context-free languages with the use of a
homomorphism.

In this respect the theorem is an analogon to lemma 7 in chapter II.1 where we
proved that each regular language can be presented as a homomorphic image of a
local language.

We want to formulate the theorem now.

Let $D(X)$ be the Dyck language over the alphabet $X$ (see chapter I.3).

Remark: In the literature, often the language
\[ \setof{w\in (\unioninv{X})^* \mid |w|= \epsilon\text{ in the free group
}F(X)} 
\] 
is called Dyck language. We denote this language as {\em symmetric Dyck
language}.

With our notations the following theorem holds:
\begin{theorem}[Chomsky-Schützenberger]
To each algebraic language $L\in ALG(X^*)$ there exists an alphabet $\Sigma
\subset X_\infty$, a regular set $R\in REG((\unioninv{\Sigma})^*)$ and a monoid
homomorphism $\phi: (\unioninv{\Sigma})^* \to X^*$ such that
\[ L = \phi(D(\Sigma)\cap R) \]
\end{theorem}

\begin{proof}
$L\in ALG(X^*) \Rightarrow$ it exists a PDA $\kappa=(\fa{A},\pdstore{K})$ with
$L = L_\kappa$, where $\fa{A}=(G, X, S, F, \alpha)$ is the corresponding finite
automaton.

For the proof we use the graph $G'=(V',E')$ we constructed in theorem 2 of
section III.1 and the mapping $\gamma: E' \to \hgroup{Y}$.

Our goal is to construct a graph $G''$ with edges from $\unioninv{\Sigma}$.

To do so, we define a relation $\rho \subset E'\times E'$ on the edge set of
$G'$ such that for edges $e_1, e_2\in E'$ it holds
\[ e_1 \rho e_2 \iff e_1 = (e, y),\ e_2 = (e', y', y)\text{ with }\delta(e, y) =
yy' \]

This relation can be represented by a bipartite graph that has the edges from
$E'$ as its vertices and whose edges connect edges from $E'$ is these are
related by $\rho$.

\transrem{TODO: write down the proof in a more readable form than in the book and
illustrate its ideas using diagrams.}
\end{proof}

We get the following
\begin{corollary}
Foe each PDA $\kappa$ there exists a Dyck language $D(\Sigma)$ and a local set
$R$ over $\unioninv{\Sigma}$ and a monoid homomorphism $\phi:
(\unioninv{\Sigma})^* \to X^*$ such that
\begin{enumerate}
  \item $\phi(D(\Sigma\cap R)) = L_\kappa$
  \item For $w\in X^*$ it holds $<\kappa, w> = \card{\phi^{-1}(w)}$
\end{enumerate}
\end{corollary}

\begin{proof}
\begin{enumerate}
  \item Theorem of Chomsky-Schützenberger
  \item Follows from the construction in the proof
\end{enumerate}
\end{proof}

We even proved slightly more:

Let \[ <\kappa, w> := \card{\setof{\pi\in\pathcat{G}(S,F)\mid
\alpha(\pi)=w,\ \delta(\pi,\epsilon)=\epsilon}}
\]
Problem: Is it possible to always find a PDA $\kappa$ accepting an algebraic
language $L \in ALG(X^*)$ such that for each word $w$ it holds $<\kappa, w> <
\infty$?

If the definition of the pushdown store is generalized such that $\delta(X
\times Y) \subset Y*$ is allowed, is then the condition $<\kappa, w> <
\infty$ always satisfiable?

At the end of this section we want to mention the following (for the interested
reader):

Because $D(\Sigma) \in ALG(X_\infty^*)$ and by theorem 1 in section III.2 the
class of algebraic languages is closed under monoid homomorphism and by theorem
3 in chapter III.2 under intersection with regular sets, from the theorem of
Chomsky-Schützenberger follows that the class of algebraic languages over the
alphabet $X_\infty$ equals the class of context-free languages.

The problem above will be investigated in a later chapter.
