\section{Graphs, categories and functors}

Before defining graphs formally, we want to describe what we mean by a graph. A
graph consists of points and edges. Each edge connects two points which are not
necessarily different. You can imagine a graph as streetmap, the cities are the
points and the streets are the edges of the graph. The edges may be oriented
such that they have a one-way direction. Paths in graphs are sequences of edges
that you could drive for example with a car without violating the traffic rules.

One can show that every graph as we will formally define has, with a certain
restriction, a faithful(?) image in $\mathbb{R}^3$, see \cite{Wagner}. The
points of the graph are here the points in $\mathbb{R}^3$, the edges are lines
in $\mathbb{R}^3$ which do not intersect pair-wise.

The mentioned restriction is that the graph must not have more points than the
cardinality of $\mathbb{R}^3$. The restriction concerning the edges is more
severe: It say that there is at most one edge between two points and that the
graph has no loops. Loops are edges with just a single point.

From what has been said we see that we may use a concrete geometric picture of
a graph without getting our intuition mistaken. The following definition of a
graph nevertheless doe not contain any geometry.

\begin{definition}[graph]
A {\bf graph} $G = (V, E)$ consists of a non-empty set $V$ of points (also
called vertices) and a set $E$ of edges and a mapping $\rho: E -> Pot(V)$ with
$card(\rho(e)) <= 2$ for $e \in E$. $\rho(e)$ is the set of {\em border points}
of $e$.
\end{definition}

Border points of an edge do not need to be different. If $card(\rho(e)) = 2$ we
call $e$ a {\bf line}, if $card(\rho(e)) = 1$ we call it a {\bf loop}.

\begin{definition}[loop-free]
A graph is called {\bf loop-free} if it does not contain a loop.
\end{definition}

We introduce an orientation for the edges. 

\begin{definition}[oriented graph]
A graph $G = (V,E)$ is called an {\bf oriented graph} if there are two mappings
$Q : E \to V$ and $Z : E \to V$ with $\rho(e) = {Q(e), Z(e)}$ for all $e \in E$.
\end{definition}

$Q(e)$ is called the {\bf source} and $Z(e)$ the target point of $e$. The
notions of loop and line are naturally transferred to oriented graphs.

For each graph one can assign the corresponding oriented graph $\hat{G}$ by
defining two edges $(P_1,e,P_2)$ and $(P_2,e,P_1)$ for every edge $e$ with border points
$P_1$ and $P_2$ and defining $Q((P_1,e,P_2)) = P_1 = Z((P_2,e,P_1))$ and
$Q((P_2,e,P_1)) = P_2 = Z((P_1,e,P_2))$.

\begin{definition}[connected graph]
A graph $G = (V,E)$ is called {\bf connected} if in the corresponding oriented
graph $\hat{G}$ for each points $P$ and $P'$ there exist edge sequences $e_1,
\ldots, e_k$ with $Q(e_1) = P, Z(e_k) = P'$ and $Z(e_i) = Q(e_{i+1})$ for all
$i = 1, \ldots, k-1$.
\end{definition}

\begin{definition}[ordered graph]
A loop-free graph $G = (V,E)$ is called {\bf ordered} if for each point $P \in
V$ holds: There exists a unique (up-to cyclic permutation) ordering on the set
$\{ e \in E \ |\ P \in \rho(e) \}$.
\end{definition}

Notation: $\{ e \in E \ |\ P \in \rho(e) \}$ is called the {\bf cycle} belonging
to $P$ ($cycle(P)$).

Explanation: Image each point and its adjacent edges to be stuck on a litte
circle as in the following figure:

FIGURE

\begin{definition}[ordered graph]
A loop-free, graph $G$ is called {\bf ordered} if for all points
$P \in V$ it	holds: There exists an ordering 
\[ e_1, \ldots, e_k, e_m', \ldots, e_1'\]
such that
\[ \{ e_1, \ldots, e_k \} = \{ e \in E \mid Z(e) = P \} \]
and 
\[ \{ e_1', \ldots, e_m' \} = \{ e \in E\ |\ Q(e) = P \} \]
\end{definition}

$e_1, \ldots, e_k$ is called the {\bf ordering} of the incoming edges of $P$ and
$e_1', \ldots, e_m'$ the ordering of the outgoing edges of $P$.

Example:

FIGURE

\begin{definition}[path]
A {\bf path} in an oriented graph $G$ is a sequence 
\[ \pi = (Q_1, e_1, \ldots, e_k, Z_k) \]
with $k \geq 1,\ e_1, \ldots, e_k \in E,\ Q(e_1) = Q_1,\ Z(e_k)=Z_k$ and
$Q(e_{i+1}) = Z(e_i)$ for $i = 1, \ldots, k-1$.
\end{definition}

We extends the mappings $Q$ and $Z$ onto paths by defining $Q(w) := Q_1$ and
$Z(w) := Z_k$. $Q_1$ is called the {\bf start point} and $Z_k$ the {\bf end
point} of path $w$.

$k$ is the {\bf length} of $w$, written as $L(w) = k$. For $k = 0$ we declare
for all points $P \in V$ that $w = (P, P)$ is the path of length 0 from $P$ to
$P$.

Paths in arbitrary graphs are defined by switching to the oriented graph
$\hat{G}$.

\begin{definition}[subpath]
Let $w = (Q, e_1, \ldots, e_k, Z_k)$ be a path. A path $w' = (Q'_1, e'_1,
\ldots, e'_m, Z'_m)$ is called a {\bf subpath} of $w$, if it holds: $\exists i,
i \leq i \leq k$ such that $e'_j = e'_{i+j-1},\ j = 1, \ldots, m$ and $i + m -
1 \leq k$.
\end{definition}

A path is called {\bf closed} if $Q(w) = Z(w)$, it is called a {\bf circle} if
it is closed and does not contain any closed subpath $w'$ with $L(w') > 0$.

\begin{definition}[circle-free graph]
A graph $G = (V, E)$ is called {\bf circle-free} if there are no circles in $G$.
\end{definition}

For our purposes we will only consider oriented graphs. For these graphs the
following definition reflects a special connectivity property.

\begin{definition}[star, center]
Let $G = (V, E)$ be an oriented graph and $P \in V$. $G$ is called a {\bf star
around P} if for each $P' \in V$ there exists a path $w_{P'}$ with $Q(w_{P'}) =
P$ and $Z(w_{P'}) = P'$. $P$ is called the {\bf center} of $G$.
\end{definition} 

We want to introduce now a special kind of graph that plays a central role in
the theory of formal languages.

\begin{definition}[tree]
A {\bf tree} is a circle-free star where for all $P \in V$ it holds $card(\{ e
\in E\ |\ Z(e) = P\}) \leq 1$.
\end{definition}

The following lemma holds:

\begin{lemma}
A tree has exactly one center which is called the {\bf root}.
\end{lemma}

Historical remark: Leonard Euler (1735) at a walk in Königsberg asked himself if
he could traverse each of the seven bridges over the Memel in such
a way that he would traverse each bridge exactly once. In the figure below you
can see a graph describing the situation. Euler gave a simple criterion for the
existance of paths that traverse each edge of a graph exactly once (the so
called Euler paths).

FIGURE

Now we want to concatenate paths or mathematically, define a product operation
on paths. We define:
\[ (Q_1, e_1, \ldots, e_k, Z_k) \cdot (Q_{k+1}, e_{k+1}, \ldots, e_n, Z_n) \]
\[:= (Q1,e_1, \ldots, e_n, Z_n)\ \mbox{if}\ Q_{k+1} = Z_k \]

That means you can concatenate two paths if the end point of the first is the
start point of the second path. Obviously it holds:

\begin{enumerate}
  \item The product of paths is associative (if defined)
  \item For each point $P$ of a graph $G$ there exists exactly one path $1_P :=
  (P, P)$ such that for each path $w$ it holds:
  \item
  	\begin{eqnarray*}
    	w \cdot 1_P = w,\ \mbox{if}\ Z(w) = P \\
    	1_P \cdot w = w,\ \mbox{if}\ Q(w) = P
  	\end{eqnarray*}
\end{enumerate}

We denote the set of paths of a graph $G$ with $\pathcat{G}$.

$\pathcat{G}$ is called the {\bf path category} of $G$ and $G$ in this
context is also called {\bf schema}. $\pathcat{G}$ is an important
special case of a category.

Notation: 
\[ \pathcat{G}(P,P') := \{ w \in \pathcat{G} \mid Q(w) = p,\ Z(w) = P' \} \]

Categories are algebraic structures with a {\em partial} operation.

\begin{definition}[category]
$C = (O, M, Q, Z, \circ)$ is called a {\bf category} if the axioms (K1) to
(K4) are fulfilled:
\begin{enumerate}
  \item[(K1)] $O$ and $M$ are sets and $Q: M \to O$ and $Z: M \to O$ are
  mappings.\\
  $Q(f)$ is the source of $f$ and $Z(f)$ is the target of $f$, $O$ is the set of
  {\bf objects} and $M$ the set of {\bf morphisms} of the category $C$.
  \item[(K2)] For $f, g \in M$ the operation $\circ$ is defined if $Q(g) =
  Z(f)$. In this case it holds $f \circ g \in M, Q(f \circ g) = Q(f), Z(f
  \circ g) = Z(g)$.
  \item[(K3)] The associative law $(f \circ (g \circ h)) = (f \circ g) \circ h$
  holds in the sense that each of both sides is defined if one of both is.
  \item[(K4)] For each object $w \in O$ there exists a unit morphism $1_w \in M$
  with $Q(1_w) = Z(1_w) = w$ and for all morphisms $f, g \in M$ with $Q(f) =
  Z(g) = w$: $1_w \circ f = f$ and $g \circ 1_w = g$. It can be easily shown
  that there exists exactly one unit morphism for each object $w$.
\end{enumerate}
\end{definition}


Notations: $Obj(C) := O$ is the {\bf set of objects} and $Mor(C) := M$ the
{\bf set of morphisms} of the category $C$.

Historical remark: Euler was already interested in graphs and paths in graphs.
The path category has already been used before the notion of category even
existed. The axiomatic formulation of categories and its importance for many
areas of mathematics has been elaborated by S.\ Eilenberg and S.\ MacLane in
1945 \cite{EiMa}. Their work has stimulated a broad, very abstract theory of
categories. We will only use the notations for structures which are categories
and some elementary concepts which also in the theory of formal languages lead
to fruitful questions.

We explain the notion of category on a number of examples:

\begin{enumerate}
  \item {\bf The category of relations} \\
  Define $REL(O) = (O, M, Q, Z,\circ)$ by:
  \begin{itemize}
    \item Let $O$ be a set of sets ($O \notin O)$. 
  	\item $ M = \{ (A,B,R)\ |\ A \in O, B \in O, R \subset A \times B \}$
  	\item $Q(A,B,R) = A, \quad Z(A,B,R) = B$
  	\item $(A,B,R_1) \circ (B,C,R_1) = (A,C,R')\\
  		\mbox{where}\ R'= \{ (a,c)\ |\ \exists b \in B : (a,b) \in R_1\ \mbox{and}\
  		(b,c) \in R_2 \} $
  \end{itemize}
  With these definitions $REL(O)$ becomes a category.
  
  \item {\bf The category of matrices} \\
  Let $MAT(\mathbb{Q}) = (O. M, Q, Z, \circ)$ with 
  \begin{itemize}
    \item $O = \mathbb{N}$
    \item $M = $ the set of $k \times n$ matrices, $k, n \in \mathbb{N}$, with
    entries from $\mathbb{}$.
    \item For a $k \times n$ matrix $A_{k,n}$ define source and target mappings
    by
    \begin{itemize}
	    \item[] $Q(A_{k,n} = k)$ the number of rows
  	  \item[] $Z(A_{k,n} = n)$ the number of columns
    \end{itemize}
  \end{itemize}
	With the matrix multiplikation as category operation $\circ$ the set
	$MAT(\mathbb{Q})$ becomes a category. Units in this category are the $n
  \times n$ unit matrices.
\end{enumerate}

Analogously to the monoid homomorphisms we introduce structure-preserving
mappings between categories, named {\bf functors}.

\begin{definition}[functor]
Let $C_i = (O_i, M_i, Q_i, Z_i, \circ_i), i = 1, 2$ be two categories and
$\phi_1: O_1 \to O_2$ and $\phi_2: M_1 \to M_2$ be mappings.

$\phi = (C_1, C_2, \phi_1, \phi_2)$ is called a {\bf functor} from $C_1$ to
$C_2$ if the axioms (F1) to (F3) hold:
\begin{itemize}
  \item[(F1)] The diagram
	\begin{tikzcd}[column sep=large,row sep=large]
	 O_1 \arrow[leftarrow]{r}{Q_1} \arrow{d}{\phi_1} & M_1 \arrow{r}{Z_1}
	 \arrow{d}{\phi_2} & O_1 \arrow{d}{\phi_1} \\
	 O_2 \arrow[leftarrow]{r}{Q_2} & M_2 \arrow{r}{Z_2} & O_2
	\end{tikzcd}
	is commutative.
  
  \item[(F2)] $\phi_2(f \circ_1 g) = \phi_2(f) \circ_2 \phi_2(g)$ for all $f, g
  \in M_1$ with $Z(f) = Q(g)$.
  \item[(F3)] $\phi_2(1_w) = 1_{\phi_1(w)}$ for all $w \in O_1$.
\end{itemize}
\end{definition}

A functor $\phi$ is called injective (surjective, bijective) if $\phi_1$ and
$\phi_2$ are injective (surjective, bijective).

Let's look at some examples:

{\bf Example 1}: Consider the following oriented graphs $G_1$ and $G_2$:

FIGURE

$G1 = (V_1, E_1)$ represents an infinite binary tree. From each point of the
tree two edges go out which are labeled with $f$ and $g$.

$G_2 = (V_2, E_2)$ consists of a single point $P_0$ and two loops labeled with
$f$ and $g$ respectively.

Consider the path categories $\pathcat{G_1}$ and $\pathcat{G_2}$. For $P
\in V_1$ define $\phi_1(P) := P_0$ and $\phi_2(1_P) := 1_{P_0}$.

For an edge $e \in E_1$ we define:

$ \phi'(e) = \left\{ \begin{array}{l}
	f\quad \mbox{if $e$ is marked with $f$} \\ 
	g\quad \mbox{if $e$ is marked with $g$}
	\end{array} \right. $

Now we define for $(P, e_1, \ldots, e_n, P') \in \pathcat{G_1}$:

$\phi_2((P, e_1, \ldots, e_n, P')) = (P_0, \phi'_2(e_1), \ldots, \phi'_2(e_n),
P_0)$.

Obviously $\phi = (\pathcat{G_1}, \pathcat{G_2}, \phi1, \phi_2)$ is a
functor.

It is a special functor because

\begin{enumerate}
  \item $\phi$ is surjective
  \item If $P_1$ is a point in $G_1$ and $\bar{w}$ is a path in $G_2$, then
  there exists exactly one path $w$ in $G_1$ with $Q(w) = P_1$ such that
  $\phi_2(w) = \bar{w}$.
\end{enumerate}

{\bf Example 2}: Let graphs $G1, G_2$ be given as follows:

FIGURE

Then there exists a surjective functor from $\pathcat{G_1}$ to
$\pathcat{G_2}$.

It is possible to construct surjective functors which fulfill (2) from example 1
and other surjective functors which don't.

{\bf Example 3}: Let $G_1$ and $G_2$ be given as:

FIGURE

We define: 
\[ \begin{array}{r@{\quad = \quad}l}
\phi_1(P_1) & Q_1 \\
\phi_1(P_2) & Q_2 \\
\phi_1(P_3) & Q_2 \\
\phi_1(P_4) & Q_3 \\
\phi_2((P_1, s, P_2)) & (Q_1, f, Q_2) \\
\phi_2((P_3, r, P_4)) & (Q_2, g, Q_3)
\end{array} \]

For the units, the defintion of $\phi_2$ is clear. 

One can see that $\phi = (\pathcat{G_1}, \pathcat{G_2}, \phi_1, \phi_2)$
is a functor.

It is remarkable that $\phi_2(\pathcat{G_1})$ is not a category because this
set is not closed under the $\circ$ operation.

{\bf Example 4}: Let $G_1$ and $G_2$ be given as follows:

FIGURE

We define:
\[ \begin{array}{r@{\quad = \quad}l}
\phi_1(1) & 1' \\
\phi_1(2) & 2' \\
\phi_1(3) & 3' \\
\phi_1(4) & 3' \\
\phi_1(5) & 3' \\
\phi_2(a) & a' \\
\phi_2(b) & b' \\
\phi_2(c) & c' \\
\phi_2(d) & d' \\
\phi_2(e) & 1_3 \\
\phi_2(f) & 1_3 \\
\phi_2(g) & 1_3
\end{array} \]

$\phi = (\pathcat{G_1}, \pathcat{G_2}, \phi_1, \phi_2)$ is a functor.

{\bf Example 5}: The graph $G$ shall be defined by

FIGURE

Additionally, the following matrices are given:

\[
a' = \left( \begin{array}{cccc}
1 & 0 & 2 & 1 \\ 0 & 1 & 2 & 5 \\ 1 & 1 & 2 & 1
\end{array} \right)
\qquad 
b' = \left( \begin{array}{ccc}
1 & 0 & 1 \\ 0 & 1 & 1 \\ 2 & 2 & 2 \\ 1 & 5 & 1
\end{array} \right)
\qquad 
c' = \left( \begin{array}{ccc}
4 & 5 & 6 \\ 1 & 2 & 3
\end{array} \right)
\]

\[
d' = \left( \begin{array}{cc}
7 & 4 \\ 5 & 3 \\ 3 & 5 \\ 4 & 7
\end{array} \right)
\qquad
e' = \left( \begin{array}{cccc}
1&2&3&4 \\ 2&3&4&1 \\ 3&4&1&2 \\ 1&0&0&0
\end{array} \right)
\qquad
f' = \left( \begin{array}{cc}
1&2 \\ 2&0
\end{array} \right)
\]

We consider $\pathcat{G}$ and $MAT(\mathbb{N})$, the category of matrices
over $\mathbb{N}$.

We define $\phi_1(i) = i$ for $i = 2,3,4$ and $\phi'_2(x) = x'$ for $x \in \{
a, b, c, d, e, f \}$.

$\phi'_2$ can be extended in a unique way to a mapping $\phi_2 :
\pathcat{G} \to MAT(\mathbb{N})$ such that $\phi = (\pathcat{G},
MAT(\mathbb{N}), \phi_1, \phi_2)$ is a functor.

We want to define now some special properties of functors.

\begin{definition}
Let $G_1, G_2$ be ordered graphs, $\phi = (\pathcat{G_1}, \pathcat{G_2},
\phi_1, \phi_2)$ a functor.

$\phi$ is called {\bf ordered} or {\bf order preserving} if it holds:

Let $\phi_1(P) = P' \in V_2$ for any $P \in V_1$, then for the ordering $e_1,
\ldots, e_k, e'_m, \ldots, e'_1$ which belongs to $P$ it holds:

$\phi_2(e_1), \ldots, \phi_2(e_k), \phi_2(e'_m), \ldots, \phi_2(e'_1)$ is
contained in the ordering that belongs to $P'$ in the given order. 
\end{definition}

It is possible that lines coincide which are counted only once in that case.

Let's give an example for this definition:

Let $P \in V_1$ be a point with ordering $e_1, e_2, e_3, e'_4, e'_3, e'_2,
e'_1$ and $P' \in V_2$ be a point with ordering $r_1, r_2, r'_5,
r'_4, r'_3, r'_2, r'_1$ as shown in the following figure:

FIGURE

Define $\phi$ by $\phi_1(P) = P'$ and 
\[ \phi_2(e_1) = r_1, \phi_2(e_2) = r_2, \phi_2(e_3) = r_2 \]
\[ \phi_2(e'_1) = r'_1, \phi_2(e'_2) = r'_3, \phi_2(e'_3) = r'_4, \phi_2(e'_4) = r'_5 \]

Then $\phi$ respects the ordering in point $P$.

\begin{definition}
Let $G_1, G_2$ be oriented graphs and $\phi = (\pathcat{G_1}, \pathcat{G_2},
\phi_1, \phi_2)$ be a functor.

$\phi$ is called {\bf regular} $\Leftrightarrow$ the restriction of $\phi_2$
to the set $\{ e \in E_1\ |\ Q(e) = P \}$ and $\{ e' \in E_2\ |\ Q(e') = \phi_1(P) \}$ and to $\{
e \in E_1\ |\ Z(e) = P \}$ and $\{ e' \in E_2\ |\ Z(e') = \phi_1(P) \}$ for $P
\in V_1$ is bijective.
\end{definition}

To each incoming / outgoing edge of a point $P \in V_1$ corresponds exactly
one incoming / outgoing edge of $\phi_1(P) \in V_2$.

In our example, $\phi$ was not regular.

We slightly weaken the definition of a regular functor by only postulating
regularity on the outgoing edges.

\begin{definition}
Let $G_1, G_2$ be oriented graphs and $\phi = (\pathcat{G_1}, \pathcat{G_2},
\phi_1, \phi_2)$ be a functor.

$\phi$ is called {\bf out-regular} $\Leftrightarrow$ the restriction of $\phi_2$
to the set $\{ e \in E_1\ |\ Q(e) = P \}$ and $\{ e' \in E_2\ |\ Q(e') = \phi_1(P) \}$ for
$P \in V_1$ is bijective.
\end{definition}

The following lemma holds:

\begin{lemma}
If $\phi = (\pathcat{G_1}, \pathcat{G_2}, \phi_1, \phi_2)$ is an
out-regular functor, then \\ $\phi(\pathcat{G_1})$ is a category.
\end{lemma}

Our next lemma describes a well-known fact from graph theory that has found many
applications.

\begin{lemma}
To each circle-free star $G = (V, E)$ relative to a point $P$ there exists a
tree $B$ and an out-regular functor $(\pathcat{B}, \pathcat{G},
\phi_1, \phi_2)$ mapping the root of the tree $B$ to the point $P$. $B$ is
determined up to isomorphisms.
\end{lemma}



































