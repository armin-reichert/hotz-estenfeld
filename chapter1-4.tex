\section{Graphs, categories and functors}

Before defining graphs formally we want to describe what we mean by a graph. A
graph consists of points and edges. Each edge connects two points which are not
necessarily different. You can imagine a graph as streetmap, the cities are the
points and the streets are the edges of the graph. The edges may be oriented
such that they have a one-way direction. Paths in graphs are sequences of edges
that you could drive for example with a car without violating the traffic rules.

One can show that every graph, as we will formally define, has (with a certain
restriction) a faithful image in $\mathbb{R}^3$, see \cite{Wagner}. The points
of the graph become points in $\mathbb{R}^3$, the edges become connecting lines
in $\mathbb{R}^3$ which do not pairwise intersect.

The mentioned restriction concerning the points is that the graph must not have
more points than the cardinality of $\mathbb{R}^3$ which is the cardinality of
the continuum.

The restriction concerning the edges is more severe: It tells that there is at
most one edge between two points and that the graph has no loops.
Loops are edges with just a single border point.

From what has been said we see that we may use a concrete geometric imagination
of a graph without getting our intuition mistaken. The following definition of a
graph nevertheless does not contain any geometry.

\begin{definition}
A {\bf graph} $G = (V, E)$ consists of a non-empty set $V$ of points
(vertices), a set $E$ of edges and a mapping $\rho: E \to \powset{V}$ with
$\card{\rho(e)} <= 2$ for $e \in E$. $\rho(e)$ is the set of {\em border points}
of $e$.
\end{definition}

Border points of an edge do not need to be different. If $\card{\rho(e)} = 2$ we
call $e$ a {\bf line}, if $\card{\rho(e)} = 1$ we call it a {\bf loop}.

\begin{definition}
A graph is called {\bf loop-free} if it does not contain any loops.
\end{definition}

We introduce an orientation for the edges. 

\begin{definition}
A graph $G = (V,E)$ is called an {\bf oriented graph} if there are two mappings
$Q : E \to V$ and $Z : E \to V$ with $\rho(e) = \setof{Q(e), Z(e)}$ for all $e
\in E$.
$Q(e)$ is called the {\bf source} and $Z(e)$ the {\bf target} point of $e$.
\end{definition}

\transrem{The letters $Q$ and $Z$ come from the German words for source
(''Quelle'') and target (''Ziel'').}

The notions of loop and line are naturally transferred to oriented graphs.

To each graph one can assign the corresponding oriented graph $\hat{G}$ by
defining two edges $(v_1,e,v_2)$ and $(v_2,e,v_1)$ for each edge $e$ with
border points $v_1$ and $v_2$ and defining $Q((v_1,e,v_2)) = v_1 =
Z((v_2,e,v_1))$ and $Q((v_2,e,v_1)) = v_2 = Z((v_1,e,v_2))$.

\begin{definition}
A graph $G = (V,E)$ is called {\bf connected} if in the corresponding oriented
graph $\hat{G}$ for each pair of points $v$ and $v'$ there exist edge sequences
$e_1, \ldots, e_k$ with $Q(e_1) = v,\ Z(e_k) = v'$ and $Z(e_i) = Q(e_{i+1})$ for
$i = 1, \ldots, k-1$.
\end{definition}

\begin{definition}
A loop-free graph $G = (V,E)$ is called {\bf ordered} if for each point $v \in
V$ holds: There exists a unique (up-to cyclic permutation) ordering on the set
$\setof{e \in E \mid v \in \rho(e)}$.
\end{definition}

Notation: $\setof{e \in E \mid v \in \rho(e)}$ is called the {\bf cycle} of
$v$.

Explanation: Imagine each point and its adjacent edges to be stuck on a litte
circle as in the following figure:

\begin{center}
\begin{tikzpicture}
	\begin{pgfonlayer}{nodelayer}
		\node [style=none] (0) at (0, -0) {};
		\node [style=none] (1) at (-2, 3) {};
		\node [style=none] (2) at (2, 3) {};
		\node [style=none] (3) at (3, -2) {};
		\node [style=none] (4) at (-2, -2) {};
		\node [style=none] (5) at (-3, 1) {};
	\end{pgfonlayer}
	\begin{pgfonlayer}{edgelayer}
		\draw (1.center) to node[auto]{$e_1$} (0.center);
		\draw (2.center) to node[auto]{$e_2$} (0.center);
		\draw (3.center) to node[auto]{$e_3$} (0.center);
		\draw (4.center) to node[auto]{$e_4$} (0.center);
		\draw (5.center) to node[auto]{$e_5$} (0.center);
	\end{pgfonlayer}
\end{tikzpicture}
\end{center}

In this figure the orderings for $\setof{e_1,e_2,e_3,e_4,e_5}$ are:
\begin{eqnarray*}
& e_1, e_2, e_3, e_4, e_5 \\
& e_2, e_3, e_4, e_5, e_1 \\
& \ldots & \\
& e_5, e_1, e_2, e_3, e_4
\end{eqnarray*}

\begin{definition}
A loop-free, oriented graph $G$ is called {\bf ordered} if for all points
$v \in V$ it	holds: In the cycle of $v$ there exists an ordering 
\[ e_1, \ldots, e_k, e_m', \ldots, e_1'\]
such that
\[ \setof{e_1, \ldots, e_k} = \setof{e \in E \mid Z(e) = v} \]
and 
\[ \setof{e_1', \ldots, e_m'} = \setof{e \in E \mid Q(e) = v} \]
\end{definition}

$e_1, \ldots, e_k$ is called the {\bf ordering} of the incoming edges of $v$ and
$e_1', \ldots, e_m'$ the ordering of the outgoing edges of $v$.

{\bf Example:}

\begin{center}
\begin{tikzpicture}
	\begin{pgfonlayer}{nodelayer}
		\node [style=state] (0) at (0, -0) {$v$};
		\node [style=none] (1) at (-2, 3) {};
		\node [style=none] (2) at (2, 3) {};
		\node [style=none] (3) at (3, 1) {};
		\node [style=none] (4) at (2.25, -1.5) {};
		\node [style=none] (5) at (-0.75, -1.5) {};
		\node [style=none] (6) at (-2.75, -0.25) {};
	\end{pgfonlayer}
	\begin{pgfonlayer}{edgelayer}
		\draw [style=transition] (1.center) to node[auto]{$e_1$} (0);
		\draw [style=transition] (2.center) to node[auto]{$e_2$} (0);
		\draw [style=transition] (3.center) to node[auto]{$e_3$} (0);
		\draw [style=transition] (0) to node[auto]{$e_4$} (4.center);
		\draw [style=transition] (0) to node[auto]{$e_5$} (5.center);
		\draw [style=transition] (0) to node[auto]{$e_6$} (6.center);
	\end{pgfonlayer}
\end{tikzpicture}
\end{center}

$e_1,e_2,e_3,e_4,e_5,e_6$ is the ordering of vertex $v$.

$e_1,e_2,e_3$ is the ordering of the incoming edges.

$e_6,e_5,e_4$ is the ordering of the outgoing edges.

\begin{definition}
A {\bf path} in an oriented graph $G$ is a sequence 
\[ \pi = (q, e_1, \ldots, e_k, z),\ k \geq 1 \]
where $ e_1, \ldots, e_k \in E$ are edges and
\begin{eqnarray*}
Q(e_1) & = & q \\
Z(e_{i}) & = & Q(e_{i+1})\text{ for }i = 1, \ldots, k-1 \\
Z(e_k) & = & z 
\end{eqnarray*}
\end{definition}

We extend the source and target mappings $Q$ and $Z$ onto paths by defining
$Q(\pi) := q$ and $Z(\pi) := z$. $q$ is called the {\bf start point} and $z$ the {\bf end
point} of the path $\pi$.

$L(\pi) := k$ is called the {\bf length} of the path $\pi$. For $k = 0$ we
define for all points $v \in V$ that $\pi = (v, v)$ is the path of length 0 from
$v$ to $v$.

Paths in arbitrary graphs are defined by switching to the oriented graph
$\hat{G}$.

\begin{definition}
Let $\pi = (q, e_1, \ldots, e_k, z)$ be a path. A path $\pi' = (q', e'_1,
\ldots, e'_m, z')$ is called a {\bf subpath} of $\pi$ if there exists an $i$
with $1 \leq i \leq k$ and $e'_j = e_{i+j-1},\ j = 1, \ldots, m$ and $i + m -
1 \leq k$.
\end{definition}

A path is called {\bf closed} if $Q(\pi) = Z(\pi)$, it is called a {\bf circle}
if it is closed and does not contain any closed subpath $\pi'$ with $L(\pi') >
0$.

\begin{definition}
A graph $G = (V, E)$ is called {\bf circle-free} if there are no circles in $G$.
\end{definition}

For our purposes we will only consider oriented graphs. For these graphs the
following definition reflects a special connectivity property.

\begin{definition}
Let $G = (V, E)$ be an oriented graph and $v \in V$. $G$ is called a {\bf star
relative to $v$} if for each $v' \in V$ there exists a path $\pi_{v'}$ with
$Q(\pi_{v'}) = v$ and $Z(pi_{v'}) = v'$. $v$ is called the {\bf center} of $G$.
\end{definition} 

We want to introduce now a special kind of graph that plays a central role in
the theory of formal languages.

\begin{definition}
A {\bf tree} is a circle-free star where for all $v \in V$ it holds
\[ \card{\setof{e \in E \mid Z(e) = v}} \leq 1 \]
\end{definition}

The following lemma holds:
\begin{lemma}
A tree has exactly one center which is called its {\bf root}.
\end{lemma}

Historical remark: Leonard Euler (1735) at a walk in Königsberg asked himself if
he could traverse each of the seven bridges over the Memel in such
a way that he would traverse each bridge exactly once. In the figure below you
can see a graph describing the situation. Euler gave a simple condition for the
existence of paths that traverse each edge of a graph exactly once (the so
called Euler paths).

\begin{center}
\begin{tikzpicture}
	\begin{pgfonlayer}{nodelayer}
		\node [style={filled_vertex}] (0) at (0, -0) {};
		\node [style={filled_vertex}] (1) at (0, 2) {};
		\node [style={filled_vertex}] (2) at (0, -2) {};
		\node [style={filled_vertex}] (3) at (3, -0) {};
	\end{pgfonlayer}
	\begin{pgfonlayer}{edgelayer}
		\draw [style=simple] (0) to (3);
		\draw [style=simple, bend right=45, looseness=1.50] (0) to (2);
		\draw [style=simple, bend left, looseness=1.00] (3) to (2);
		\draw [style=simple, bend right, looseness=1.00] (3) to (1);
		\draw [style=simple, bend left=45, looseness=1.50] (0) to (1);
		\draw [style=simple, bend left=45, looseness=1.50] (1) to (0);
		\draw [style=simple, bend left=45, looseness=1.50] (0) to (2);
	\end{pgfonlayer}
\end{tikzpicture}
\end{center}

Now we want to concatenate paths or mathematically, define a product operation
on paths. We define:
\[ (q, e_1, \ldots, e_k, z) \cdot (q', e_{k+1}, \ldots, e_n, z') := (q, e_1,
\ldots, e_n, z')\text{ if } z = q' \]

That means you can concatenate two paths if the end point of the first is the
start point of the second path. Obviously it holds:
\begin{enumerate}
  \item The product of paths (if defined) is associative
  \item For each point $v$ of a graph $G$ there exists exactly one path $1_v :=
  (v, v)$ such that for each path $\pi$ it holds:
  	\begin{eqnarray*}
    	\pi \cdot 1_v = \pi,\text{ if }Z(\pi) = v \\
    	1_v \cdot \pi = \pi,\text{ if }Q(\pi) = v
  	\end{eqnarray*}
\end{enumerate}

We denote the set of paths of a graph $G$ with $\pathcat{G}$. $\pathcat{G}$ is 
called the {\bf path category} of $G$ and in this context $G$ is also called a {\bf
schema}. $\pathcat{G}$ is an important special case of a category.

\transrem{The letter $\mathfrak{W}$ comes from the German word ''Wegekategorie''
(path category).}

Notation: 
\[ \pathcat{G}(v, v') := \setof{\pi \in \pathcat{G} \mid Q(\pi) = v,\ Z(\pi) =
v'}
\]
{\em Categories} are algebraic structures with a {\em partial} operation.

\begin{definition}
$C = (O, M, Q, Z, \circ)$ is called a {\bf category} if the axioms (K1) to
(K4) are fulfilled:
\begin{enumerate}
  \item[(K1)] $O$ and $M$ are sets and $Q: M \to O$ and $Z: M \to O$ are
  mappings.\\
  $Q(f)$ is the {\bf source} of $f$ and $Z(f)$ is the {\bf target} of $f$, $O$
  is the set of {\bf objects} and $M$ the set of {\bf morphisms} of the category $C$.
  \item[(K2)] For $f, g \in M$ the operation $\circ$ is defined iff $Q(g) =
  Z(f)$. In this case it holds $f \circ g \in M,\ Q(f \circ g) = Q(f),\ Z(f
  \circ g) = Z(g)$.
  \item[(K3)] The associative law $f \circ (g \circ h) = (f \circ g) \circ h$
  holds in the sense that each of both sides is defined if one of both is.
  \item[(K4)] For each object $w \in O$ there exists a unit morphism $1_w \in M$
  with $Q(1_w) = Z(1_w) = w$ and for all morphisms $f, g \in M$ with $Q(f) =
  Z(g) = w$: $1_w \circ f = f$ and $g \circ 1_w = g$. It can be easily shown
  that there exists exactly one unit morphism for each object $w$.
\end{enumerate}
\end{definition}

Notations: $Obj(C) := O$ is called the {\bf set of objects} and $Mor(C) := M$
the {\bf set of morphisms} of the category $C$.

Historical remark: Euler was already interested in graphs and paths in graphs
(see above). The notion {\em path category} of a graph had already been used
before the notion {\em category} even existed. The axiomatic formulation of
categories and its importance for many areas of mathematics was elaborated by
S.\ Eilenberg and S.\ MacLane in 1945 \cite{EiMa}. Their work has stimulated a broad, 
very abstract theory of categories. We will use here only the notations for
structures which are categories and some elementary concepts which also in the theory of formal 
languages lead to fruitful questions.

We explain the notion of a category with some examples:
\begin{enumerate}
  \item {\bf The category of relations on sets} \\
  Let $O$ be a set of sets ($O \notin O)$. 
  
  Define $REL(O) := (O, M, Q, Z,\circ)$ by:
  \begin{itemize}
  	\item $M := \setof{(A,B,R) \mid A \in O,\ B \in O,\ R \subset A \times B}$
  	\item $Q(A,B,R) := A, \ Z(A,B,R) := B$
  	\item $(A,B,R_1) \circ (B,C,R_2) := (A,C,R')\\
  	\text{where }R'= \setof{(a,c) \mid \exists b \in B: (a,b)\in R_1\text{ and
  	}(b,c)\in R_2}$
  \end{itemize}
  With these definitions $REL(O)$ becomes a category (exercise) and the
  morphisms $(A, A, \setof{(a, a) \mid a\in A})$ are the units for each set $A
  \in O$.

  \item {\bf The category of matrices over the rational numbers} \\
  Define $MAT(\mathbb{Q}) := (O. M, Q, Z, \circ)$ by:
  \begin{itemize}
    \item $O := \mathbb{N}$
    \item $M :=$ the set of $k \times n$ matrices with entries from
    $\mathbb{Q} (k, n \in \mathbb{N})$.
    \item For a $k \times n$ matrix $A_{k,n}$ define source and target mappings
    by
    \begin{itemize}
	    \item[] $Q(A_{k,n}) := k$ (the number of rows)
  	  \item[] $Z(A_{k,n}) := n$ (the number of columns)
    \end{itemize}
  \end{itemize}
	With the matrix multiplication as category operation $\circ$, $MAT(\mathbb{Q})$ 
	becomes a category. Units in this category are the $n\times n$ unit matrices.
\end{enumerate}

\bigskip
In analogy to the monoid homomorphisms we introduce structure-preserving
mappings between categories called {\bf functors}.

\begin{definition}
Let $C_i = (O_i, M_i, Q_i, Z_i, \circ_i), i = 1, 2$ be two categories and
$\phi_1: O_1 \to O_2$ and $\phi_2: M_1 \to M_2$ mappings.

$\phi = (C_1, C_2, \phi_1, \phi_2)$ is called a {\bf functor} from $C_1$ to
$C_2$ if the axioms (F1) to (F3) hold:
\begin{itemize}
  \item[(F1)] The diagram
	\begin{tikzcd}[column sep=large,row sep=large]
	 O_1 \arrow[leftarrow]{r}{Q_1} \arrow{d}{\phi_1} & M_1 \arrow{r}{Z_1}
	 \arrow{d}{\phi_2} & O_1 \arrow{d}{\phi_1} \\
	 O_2 \arrow[leftarrow]{r}{Q_2} & M_2 \arrow{r}{Z_2} & O_2
	\end{tikzcd}
	is commutative.
  
  \item[(F2)] $\phi_2(f \circ_1 g) = \phi_2(f) \circ_2 \phi_2(g)$ for all $f, g
  \in M_1$ with $Z(f) = Q(g)$.
  
  \item[(F3)] $\phi_2(1_w) = 1_{\phi_1(w)}$ for all $w \in O_1$.
\end{itemize}
\end{definition}

A functor $\phi$ is called injective (surjective, bijective) if $\phi_1$ and
$\phi_2$ are injective (surjective, bijective).

Let's look at some examples:

\begin{example}
Consider the following oriented graphs $G_1$ and $G_2$:

$G_1$:
\begin{center}
\begin{tikzpicture}
	\begin{pgfonlayer}{nodelayer}
		\node [style=none] (0) at (0, -0) {};
		\node [style=none] (1) at (-2, -1) {};
		\node [style=none] (2) at (2, -1) {};
		\node [style=none] (3) at (-3, -2) {};
		\node [style=none] (4) at (-1, -2) {};
		\node [style=none] (5) at (1, -2) {};
		\node [style=none] (6) at (3, -2) {};
		\node [style=none] (7) at (-4, -3) {};
		\node [style=none] (8) at (4, -3) {};
	\end{pgfonlayer}
	\begin{pgfonlayer}{edgelayer}
		\draw [style=transition] (0.center) to node[auto]{$f$} (1.center);
		\draw [style=transition] (1.center) to node[auto]{$f$} (3.center);
		\draw [style=transition] (1.center) to node[auto]{$g$} (4.center);
		\draw [style=transition] (0.center) to node[auto]{$g$} (2.center);
		\draw [style=transition] (2.center) to node[auto]{$f$} (5.center);
		\draw [style=transition] (2.center) to node[auto]{$g$} (6.center);
		\draw [style=transition] (3.center) to node[auto]{$f$} (7.center);
		\draw [style=transition] (6.center) to node[auto]{$g$} (8.center);
	\end{pgfonlayer}
\end{tikzpicture}
\end{center}

$G_1 = (V_1, E_1)$ represents an infinite binary tree. From each point of the
tree two edges go out which are labeled with $f$ and $g$.

$G_2$:
\begin{center}
\begin{tikzpicture}[scale=8.0]
	\begin{pgfonlayer}{nodelayer}
		\node [style=state] (0) at (0, -0) {$v_0$};
	\end{pgfonlayer}
	\begin{pgfonlayer}{edgelayer}
		\draw [style=transition, in=135, out=-135, loop] (0) to node[auto]{$f$} ();
		\draw [style=transition, in=-45, out=45, loop] (0) to node[auto]{$g$} ();
	\end{pgfonlayer}
\end{tikzpicture}
\end{center}

$G_2 = (V_2, E_2)$ consists of a single point $v_0$ and two loops labeled $f$ and $g$ respectively.

Consider the path categories $\pathcat{G_1}$ and $\pathcat{G_2}$. For $v
\in V_1$ define $\phi_1(v) := v_0$ and $\phi_2(1_P) := 1_{P_0}$.

For the edges $e \in E_1$ define
\[ \phi'_2(e) = \begin{cases}
	f & \text{if $e$ is marked with $f$} \\ 
	g & \text{if $e$ is marked with $g$}
\end{cases} \]

Now we define for the paths $(v, e_1, \ldots, e_n, v') \in \pathcat{G_1}$:
\[ \phi_2((v, e_1, \ldots, e_n, v')) := (v_0, \phi'_2(e_1), \ldots,
\phi'_2(e_n), v_0) \]

Obviously $\phi = (\pathcat{G_1}, \pathcat{G_2}, \phi1, \phi_2)$ is a
functor. It is even a special functor because
\begin{enumerate}
  \item $\phi$ is surjective
  \item If $v_1$ is a point in $G_1$ and $\bar{\pi}$ is a path in $G_2$, then
  there exists exactly one path $\pi$ in $G_1$ with $Q(\pi) = v_1$ such that
  $\phi_2(\pi) = \bar{\pi}$.
\end{enumerate}
\end{example}

\bigskip
\begin{example}
Let graphs $G_1, G_2$ be given as follows:

$G_1$:
\begin{center}
\begin{tikzpicture}
	\begin{pgfonlayer}{nodelayer}
		\node [style={filled_vertex}] (0) at (0, -0) {};
		\node [style={filled_vertex}] (1) at (-2, 2) {};
		\node [style={filled_vertex}] (2) at (2, 2) {};
		\node [style={filled_vertex}] (3) at (-2, -2) {};
		\node [style={filled_vertex}] (4) at (2, -2) {};
		\node [style={filled_vertex}] (5) at (-3, -0) {};
		\node [style={filled_vertex}] (6) at (3, -0) {};
	\end{pgfonlayer}
	\begin{pgfonlayer}{edgelayer}
		\draw [style=transition] (0) to (5);
		\draw [style=transition] (0) to (6);
		\draw [style=transition] (5) to (1);
		\draw [style=transition] (5) to (3);
		\draw [style=transition] (6) to (2);
		\draw [style=transition] (6) to (4);
		\draw [style=transition, bend left, looseness=1.25] (1) to (0);
		\draw [style=transition, bend right, looseness=1.25] (1) to (0);
		\draw [style=transition, bend right, looseness=1.00] (2) to (0);
		\draw [style=transition, bend left, looseness=1.00] (2) to (0);
		\draw [style=transition, bend right, looseness=1.00] (3) to (0);
		\draw [style=transition, bend left, looseness=1.00] (3) to (0);
		\draw [style=transition, bend left, looseness=1.00] (4) to (0);
		\draw [style=transition, bend right, looseness=1.00] (4) to (0);
	\end{pgfonlayer}
\end{tikzpicture}
\end{center}

$G_2:$
\begin{center}
\begin{tikzpicture}[scale=8.0]
	\begin{pgfonlayer}{nodelayer}
		\node [style=state] (0) at (0, -0) {$v_0$};
	\end{pgfonlayer}
	\begin{pgfonlayer}{edgelayer}
		\draw [style=transition, in=135, out=-135, loop] (0) to node[auto]{$f$} ();
		\draw [style=transition, in=-45, out=45, loop] (0) to node[auto]{$g$} ();
	\end{pgfonlayer}
\end{tikzpicture}
\end{center}

Then there exists a surjective functor from $\pathcat{G_1}$ to $\pathcat{G_2}$.

It is possible to construct surjective functors which fulfill (2) from example 1
and other surjective functors which don't.
\end{example}

\bigskip
\begin{example}
Let $G_1$ and $G_2$ be given as:

$G_1:\ p_1 \edge{s} p_2 \qquad p_3 \edge{r} p_4$.

$G_2:\ q_1 \edge{f} q_2 \edge{g} q_3$

We define: $\phi_1(p_1) = q_1,\ \phi_1(p_2) = q_2,\ \phi_1(p_3) =
q_2,\ \phi_1(p_4) = q_3,\ \phi_2((p_1, s, p_2)) = (q_1, f, q_2),\ \phi_2((p_3,
r, p_4)) = (q_2, g, q_3)$.

For the units, the definition of $\phi_2$ is clear. 

One can see that $\phi = (\pathcat{G_1}, \pathcat{G_2}, \phi_1, \phi_2)$
is a functor.

It is remarkable that $\phi_2(\pathcat{G_1})$ is not a category because this
set is not closed under the $\circ$ operation.
\end{example}

\bigskip
\begin{example}
Let $G_1$ and $G_2$ be given as follows:

$G_1$:
\begin{center}
\begin{tikzpicture}
	\begin{pgfonlayer}{nodelayer}
		\node [style=state] (0) at (0, -0) {$1$};
		\node [style=state] (1) at (3, -0) {$2$};
		\node [style=state] (2) at (6, -0) {$3$};
		\node [style=state] (3) at (8, 2) {$5$};
		\node [style=state] (4) at (8, -2) {$4$};
	\end{pgfonlayer}
	\begin{pgfonlayer}{edgelayer}
		\draw [style=transition, bend left=45, looseness=1.50] (0) to node[auto]{$b$} (1);
		\draw [style=transition, bend left=45, looseness=1.50] (1) to node[auto]{$a$} (0);
		\draw [style=transition, bend left=60, looseness=1.25] (1) to node[auto]{$c$} (2);
		\draw [style=transition, bend left=45, looseness=1.50] (2) to node[auto]{$d$} (1);
		\draw [style=transition] (2) to node[auto]{$g$} (3);
		\draw [style=transition] (2) to node[auto]{$e$} (4);
		\draw [style=transition] (4) to node[auto]{$f$} (3);
	\end{pgfonlayer}
\end{tikzpicture}
\end{center}

$G_2$:
\begin{center}
\begin{tikzpicture}
	\begin{pgfonlayer}{nodelayer}
		\node [style=state] (0) at (0, -0) {$2'$};
		\node [style=state] (1) at (-3, -0) {$1'$};
		\node [style=state] (2) at (3, -0) {$3'$};
	\end{pgfonlayer}
	\begin{pgfonlayer}{edgelayer}
		\draw [style=transition, bend left=45, looseness=1.50] (1) to node[auto]{$b'$} (0);
		\draw [style=transition, bend left=45, looseness=1.50] (0) to node[auto]{$a'$} (1);
		\draw [style=transition, bend left=60, looseness=1.25] (0) to node[auto]{$c'$} (2);
		\draw [style=transition, bend left=45, looseness=1.50] (2) to node[auto]{$d'$} (0);
	\end{pgfonlayer}
\end{tikzpicture}
\end{center}

We define a functor $\phi$ by:

$\phi_1(1) = 1',\ \phi_1(2) = 2',\ \phi_1(3) = 3',\ \phi_1(4) = 3',\ \phi_1(5) =
3'$,

$\phi_2(a) = a',\ \phi_2(b) = b',\ \phi_2(c) = c',\ \phi_2(d) = d',\ \phi_2(e) =
1_3,\ \phi_2(f) = 1_3,\ \phi_2(g) = 1_3$.

$\phi = (\pathcat{G_1}, \pathcat{G_2}, \phi_1, \phi_2)$ is a functor.
\end{example}

\begin{example}
The graph $G$ shall be defined by

\begin{center}
\begin{tikzpicture}
	\begin{pgfonlayer}{nodelayer}
		\node [style=state] (0) at (-2, -0) {$4$};
		\node [style=state] (1) at (2, -0) {$2$};
		\node [style=state] (2) at (0, 2) {$3$};
	\end{pgfonlayer}
	\begin{pgfonlayer}{edgelayer}
		\draw [style=transition, in=135, out=-135, loop] (0) to node[auto]{$e$} ();
		\draw [style=transition, in=-45, out=45, loop] (1) to node[auto]{$f$} ();
		\draw [style=transition] (0) to node[auto]{$d$} (1);
		\draw [style=transition] (1) to node[auto]{$c$} (2);
		\draw [style=transition, bend right, looseness=1.00] (2) to node[auto]{$a$} (0);
		\draw [style=transition, bend right, looseness=1.00] (0) to node[auto]{$b$} (2);
	\end{pgfonlayer}
\end{tikzpicture}
\end{center}

Additionally, the following matrices are given:

\[
a' = \left( \begin{array}{cccc}
1 & 0 & 2 & 1 \\ 0 & 1 & 2 & 5 \\ 1 & 1 & 2 & 1
\end{array} \right)
\qquad 
b' = \left( \begin{array}{ccc}
1 & 0 & 1 \\ 0 & 1 & 1 \\ 2 & 2 & 2 \\ 1 & 5 & 1
\end{array} \right)
\qquad 
c' = \left( \begin{array}{ccc}
4 & 5 & 6 \\ 1 & 2 & 3
\end{array} \right)
\]

\[
d' = \left( \begin{array}{cc}
7 & 4 \\ 5 & 3 \\ 3 & 5 \\ 4 & 7
\end{array} \right)
\qquad
e' = \left( \begin{array}{cccc}
1&2&3&4 \\ 2&3&4&1 \\ 3&4&1&2 \\ 1&0&0&0
\end{array} \right)
\qquad
f' = \left( \begin{array}{cc}
1&2 \\ 2&0
\end{array} \right)
\]

Consider $\pathcat{G}$, the path category of $G$, and $MAT(\mathbb{N})$,
the category of matrices over $\mathbb{N}$.

We define $\phi_1(i) = i$ for $i = 2,3,4$ and $\phi'_2(x) = x'$ for $x \in
\setof{a, b, c, d, e, f}$.

$\phi'_2$ can be extended in a unique way to a mapping $\phi_2 :
\pathcat{G} \to MAT(\mathbb{N})$ such that $\phi = (\pathcat{G},
MAT(\mathbb{N}), \phi_1, \phi_2)$ is a functor.
\end{example}

\bigskip
We want to define now some special properties of functors.

\begin{definition}
Let $G_1, G_2$ be ordered graphs, $\phi = (\pathcat{G_1}, \pathcat{G_2},
\phi_1, \phi_2)$ a functor. $\phi$ is called {\bf ordered} or {\bf order
preserving} if it holds:

Let $\phi_1(v) = v' \in V_2$ for any $v \in V_1$, then for the ordering $e_1,
\ldots, e_k, e'_m, \ldots, e'_1$ which belongs to $v$ it holds:

$\phi_2(e_1), \ldots, \phi_2(e_k), \phi_2(e'_m), \ldots, \phi_2(e'_1)$ is
contained in the ordering that belongs to $v'$ in the given order. 
\end{definition}

It is possible that lines coincide which are counted only once in that case.

Let's give an example for this definition:

Let $v \in V_1$ be a point with ordering $e_1, e_2, e_3, e'_4, e'_3, e'_2,
e'_1$ and $v' \in V_2$ be a point with ordering $r_1, r_2, r'_5,
r'_4, r'_3, r'_2, r'_1$ as shown in the following figure:

\begin{center}
\begin{tikzpicture}
	\begin{pgfonlayer}{nodelayer}
		\node [style=state] (0) at (0, -0) {$v$};
		\node [style=state] (1) at (7, -0) {$v'$};
		\node [style=none] (2) at (-3, 2) {};
		\node [style=none] (3) at (-0.5, 2) {};
		\node [style=none] (4) at (2, 2) {};
		\node [style=none] (5) at (-3.5, -0.75) {};
		\node [style=none] (6) at (-2.25, -1.75) {};
		\node [style=none] (7) at (0, -2) {};
		\node [style=none] (8) at (2.25, -1.25) {};
		\node [style=none] (9) at (5, 2) {};
		\node [style=none] (10) at (9, 2) {};
		\node [style=none] (11) at (4.5, -0.75) {};
		\node [style=none] (12) at (5.25, -2) {};
		\node [style=none] (13) at (6.75, -2) {};
		\node [style=none] (14) at (9, -1.75) {};
		\node [style=none] (15) at (9.75, -0.75) {};
	\end{pgfonlayer}
	\begin{pgfonlayer}{edgelayer}
		\draw [style=transition] (2.center) to node[auto]{$s_1$} (0);
		\draw [style=transition] (3.center) to node[auto]{$s_2$} (0);
		\draw [style=transition] (4.center) to node[auto]{$s_3$} (0);
		\draw [style=transition] (0) to node[auto]{$s_1'$} (5.center);
		\draw [style=transition] (0) to node[auto]{$s_2'$} (6.center);
		\draw [style=transition] (0) to node[auto]{$s_3'$} (7.center);
		\draw [style=transition] (0) to node[auto]{$s_4'$} (8.center);
		\draw [style=transition] (9.center) to node[auto]{$r_1$} (1);
		\draw [style=transition] (10.center) to node[auto]{$r_2$} (1);
		\draw [style=transition] (1) to node[auto]{$r_1'$} (11.center);
		\draw [style=transition] (1) to node[auto]{$r_2'$} (12.center);
		\draw [style=transition] (1) to node[auto]{$r_3'$} (13.center);
		\draw [style=transition] (1) to node[auto]{$r_4'$} (14.center);
		\draw [style=transition] (1) to node[auto]{$r_5'$} (15.center);
	\end{pgfonlayer}
\end{tikzpicture}
\end{center}

Define $\phi$ by $\phi_1(p) = p'$ and 
\[ \phi_2(s_1) = r_1, \phi_2(s_2) = r_2, \phi_2(s_3) = r_2 \]
\[ \phi_2(s'_1) = r'_1, \phi_2(s'_2) = r'_3, \phi_2(s'_3) = r'_4, \phi_2(s'_4) =
r'_5 \]
Then $\phi$ respects the ordering in point $v$.

\bigskip
\begin{definition}
Let $G_1=(V_1,E_1),\ G_2=(V_2,E_2)$ be oriented graphs and 
\[ \phi = (\pathcat{G_1}, \pathcat{G_2}, \phi_1, \phi_2)\text{ a functor.}
\]

$\phi$ is called {\bf regular functor} $\iff$ the restriction of $\phi_2$
to the set \[ \setof{e \in E_1 \mid Q(e) = v}\text{ and }\setof{e' \in E_2 \mid
Q(e') = \phi_1(v)} \] and to \[ \setof{e \in E_1 \mid Z(e) = v}\text{ and
}\setof{e' \in E_2 \mid Z(e') = \phi_1(v)} \] for $v \in V_1$ is bijective.
\end{definition}

To each incoming (outgoing) edge of a point $v \in V_1$ corresponds exactly
one incoming (outgoing) edge of $\phi_1(v) \in V_2$.

In our example, $\phi$ was not regular. We slightly weaken the definition of a
regular functor by only postulating regularity on the outgoing edges.

\bigskip
\begin{definition}
Let $G_1=(V_1,E_1),\ G_2=(V_2,E_2)$ be oriented graphs and
\[ \phi = (\pathcat{G_1}, \pathcat{G_2}, \phi_1, \phi_2)\text{ a functor.}
\]

$\phi$ is called {\bf out-regular functor} $\iff$ the restriction of $\phi_2$
to the set \[ \setof{e\in E_1 \mid Q(e) = v} \] and \[ \setof{e' \in E_2 \mid
Q(e') = \phi_1(v)} \] for $v \in V_1$ is bijective.
\end{definition}

The following lemma holds (exercise):

\begin{lemma}
If $\phi = (\pathcat{G_1}, \pathcat{G_2}, \phi_1, \phi_2)$ is an out-regular
functor, then \\ $\phi(\pathcat{G_1})$ is a category.
\end{lemma}

Our next lemma describes a well-known fact from graph theory that has found many
applications.

\begin{lemma}
To each circle-free star $G = (V, E)$ relative to a point $v \in V$ there exists
a tree $B$ and an out-regular functor $(\pathcat{B}, \pathcat{G},
\phi_1, \phi_2)$ mapping the root of the tree $B$ to the point $v$. 
\end{lemma}

$B$ is uniquely determined up to isomorphisms.
