\chapter*{Preface}

This book is in principle the second edition of the book Hotz/Walter:
''Automatentheorie und Formale Sprachen II, Endliche Automaten'' (engl.\
''Automata theory and formal languages II, finite automata''), Bibliographisches
Institut, Mannheim, 1969.

The book however has been reworked so extensively that it really can count as
completely new. While in the first edition only the theory of finite automata had been treated, in this edition also an introduction into the theory of context-free languages
is given. This was only possible in the available space by developing the theory
in an automata-theoretic way.

Developing the theory in such a way had already been proposed by Goldstine in
1977 \cite{Goldstine77} and sketched by him in various talks. The motivation
for developing my lecture from which this book originates in that way is however
not based on Goldstine's idea. 

This composition arose almost automatically out of dealing with the works of the
French school. I must emphasize here the book by Jean Berstel on transductions
\cite{Berstel79}. Mr.\ Berstel finally pointed me to the work of Goldstine. I
fully support Goldstine's opinion that it would be worth rethinking the whole theory
 of formal languages along these automata-theoretic lines.

\transrem{''French school'' refers to the works of Schützenberger, Nivat,
Perrot, Sakarovitch, Berstel et al. The often cited book by J. Berstel on
rational transductions in fact originated from lectures of Berstel in Paris and
Saarbrücken where he gave lectures on invitation of Hotz. So his book is
strongly interrelated to this one.}

This book is only an introduction into the theory of formal languages. The
interested reader who wants to gets a deeper understanding of the theory or who
wants to get a different look into it is pointed to the books by Ginsburg,
Harrison or Salomaa. Relations to applications can be found in books on compiler
design.

Dr.\ Klaus Estenfeld worked out my lecture ''Formal Languages I'', which I gave 
on this topic in the winter of 1980/81, as the foundation of this
book and made additions at some places.

Dipl.-Math.\ Bernd Becker carefully read the manuscript and contributed with his
proposals to the success of this book.

The publisher as well as the editors of the series earn our thanks for their
patience of waiting for this second edition.

Saarbrücken, August 1981 \hfill {\em Günter Hotz}
