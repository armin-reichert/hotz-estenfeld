\chapter*{Preface}

This book is meant to be the second edition of the book Hotz/Walter:
''Automatentheorie und Formale Sprachen II, Endliche Automaten''.

\transrem{English title would be ''Automata theory and formal languages II,
finite automata''.}

But as the book has been completely reworked, it may really be taken as a
completely new one.

While in the first edition only the theory of finite automata has been treated,
in this edition also an introduction into the theory of context-free languages
is given. This was only possible in the available space frame because
of an automata-theoretic treatment of the theory.

Such a foundation of the theory has already been proposed by Goldstine in 1977
and has been sketched in various talks. The motivation for developing
my lecture (from which this book originates) in that way is however not based on
his proposal. It almost automatically arose from dealing with the works of the
French school.

\transrem{With the ''French school'' the works of Schützenberger, Nivat, Perrot,
Sakarovitch, Berstel et al.\ are referred to.}

I must emphasize here the book by Jean Berstel on
transductions. Mr.\ Berstel finally pointed my to the work of Goldstine. I fully
support Goldstine's opinion that it would be worth rethinking the whole theory of formal languages along this automata-theoretic lines.

This book is only an introduction into the theory of formal languages. The
interested reader who wants to gets a deeper understanding of the theory or who
wants to get a different look into it is pointed to the books by Ginsburg,
Harrison or Salomaa. Relations to applications can be found in books on compiler
design.

Dr.\ Klaus Estenfeld worked out my lecture ''Formal Languages I'' which I held 
on this topic in the winter semester of 1980/81 to become the foundation of this
book and he made a number of additions at some places.

Dipl.-Math.\ Bernd Becker carefully read the manuscript and contributed with his
proposals to the success of this book.

The publisher as well as the editors of the series earn our thanks for their
patience of waiting for the second edition.

Saarbrücken, August 1981

\vspace{1em}
{\em Günter Hotz}
