\section{Rational sets in $X^*, RAT(X^*)$}

In this section we want to introduce a second characterization of regular sets
over free monoids. We need the following definition.

\begin{definition}[rationally closed]
If $M$ is a monoid, then $K \in \powset{M}$ is called is called {\bf rationally
closed} $\Leftrightarrow$ $K$ is closed unter union $\cup$, complex product $\cdot$ and Kleene-star $^*$.
\end{definition}

\begin{definition}
$RAT(M)$ is the smallest, rationally closed subset of $\powset{M}$ that
contains the single element sets $\{m\}, m \in M$ and the empty set $\emptyset$.
\end{definition}

From main theorem 1 immediately follows:

\begin{lemma}
$RAT(X^*) \subset REG(X^*)$.
\end{lemma}

From the remark following the main theorems it follow that this also holds for
arbitray monoids $M$ because $\{m\} \in REG(M)$ for $m \in M$.

This lemma is the first part of the following theorem by Kleene:

\begin{theorem}[Kleene]
\[ REG(X^*) = RAT(X^*) \]
\end{theorem}

Proof: We still have to proof the inclusion $REG(X^* \subset RAT(X^*))$.

Let $\fa{A}$ be a finite automaton. We show that \lang{A} can be generated
from $\emptyset$ and $\{x\}, x \in X$	using $\cup$, $\cdot$ and $^*$-operations.

We prove this by induction over the number $k$ of edges in the graph of the
automaton when the number of vertices is fixed.

We may assume that the automaton has a single start state, $\card{S} = 1$.

$k = 0$: Without loss of generality, we may assume $S \cap F = \emptyset$. It
follows $\lang{A} = \emptyset$ which is a rational set over $X^*$.

Induction step: Let the claim be true for $i \leq k, k > 1$ and let $G_k$ be a
graph with $k$ edges.

Add a new edge $e : P \edge{x} R$ labeled with $x \in X$ to the graph $G_k$. The
resulting graph shall be $G_{k+1}$. Then the set of accepting paths in $G_{k+1}$
can be written as \begin{eqnarray*}
 \pathcat{G_{k+1}}(S, F) & = & \pathcat{G_k}(S, F) \\
 & \cup & \pathcat{G_k}(S, P) \cdot e \cdot \Big( \pathcat{G_k}(R, P) \cdot
 e \Big)^* \cdot \pathcat{G_k}(R, F)
\end{eqnarray*}

FIGURE

By induction hypothesis it holds:
\[ \alpha(\pathcat{G_k}(S, F)) \in RAT(X^*) \]
\[ \alpha(\pathcat{G_k}(S, P)) \in RAT(X^*) \]
\[ \alpha(\pathcat{G_k}(R, F)) \in RAT(X^*) \]
\[ \alpha(\pathcat{G_k}(R, P)) \in RAT(X^*) \]

From this we get $\alpha(\pathcat{G_{k+1}}(S, F)) \in RAT(X^*)$.

This concludes the proof of Kleene's theorem.



























