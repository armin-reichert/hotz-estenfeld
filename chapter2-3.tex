\section{Sets recognizable by homomorphisms,
\texorpdfstring{$\reclang(X^*)$}{REC(X*)}}

In this section we consider a third possibility for characterizing subsets of
$X^*$.

\begin{definition}
\begin{eqnarray*}
 \reclang(X^*) & := & \{ L \subset X^* \mid \text{ there exists a
finite monoid } H \\
& & \text{ and a monoid homomorphism }\mu : X^* \to H \\
& & \text{ with } L = \inv\mu(T),\ T \subset H \}
\end{eqnarray*}
is the class of {\bf recognizable sets} over $X$.
\end{definition}

\bigskip
\begin{lemma}
\[ \reglang(X^*) \subset \reclang(^*) \]
\end{lemma}

\begin{proof}
Let $L \in \reglang(X^*)$, then there exists a complete, deterministic finite
automaton $\fa{A} = (G, X, S, F, \alpha)$ with graph $G = (V, E)$ accepting $L$.

We define 
\[ H := Map(V, V) := \setof{f : V \to V \mid f \text{ is a mapping}} \]

We define a homomorphism $\mu : X^* \to H$ as follows:

Let $\mu$ be the homomorphic continuation of the mapping $\mu'$ defined by
\begin{eqnarray*}
& \mu'(x)(q) = r \iff & \\
& \text{ there exists an edge }e \in E\text{ with }Q(e) = q, Z(e) = r\text{ and
}\alpha(e) = x \in X
\end{eqnarray*}

We define 
\[ T := \setof{f \in H \mid f(S) \in F} \] 
%TODO: is this correct?

Then it holds $\lang{A} = \inv\mu(T)$ (exercise).
\end{proof}

\bigskip
\begin{lemma}
\[ \reclang(X^*) \subset \reglang(X^*) \]
\end{lemma}

\begin{proof}
Sei $L \in \reclang(X^*)$, let $H$ be a finite monoid and $\mu : X^* \to
H$ a monoid homomorphism and $L = \inv\mu(T),\ T \subset H$.

We construct a graph $G = (V, E)$ with
\begin{itemize}
  \item vertex set $V := H$
  \item edge set $E := \setof{(h, a) \mid h \in H,\ a \in X}$ with $Q((h,a)) =
  h,\ Z((h,a)) = h \cdot \mu(a)$
\end{itemize}

Let
\[ \fa{A} = (G, X, S, F, \alpha) \]
with start states $S = \setof{1_H}$, final states $F = T$ and labelling 
$\alpha((h, a)) = a$.

We show: $L = \lang{A}$.

''$\subset$'': Let $w \in L$, i.e.\ $w \in X^*$ with $\mu(w) \in T$.

If $w = x_1 \cdots w_n$ with $x_i \in X$, then $\mu(x_1) \cdots \mu(x_n) = t
\in T$.

We consider the path
\[ \pi = (1_H, (1_H, x_1), (\mu(x_1), x_2), (\mu(x_1 \cdot x_2),
x_3), \ldots, (\mu(x_1 \cdots x_{n-1}), x_n), \mu(w)) \]

$\pi \in \pathcat{G}$ with label $\alpha(\pi) = x_1 \cdots x_n = w$. 

It even holds $\pi \in \pathcat{G}(S, F)$ because $\mu(w) \in T$. From this
follows $w \in L_{\fa{A}}$.

''$\supset$'': Let $w \in L_{\fa{A}}$, then there exists a path $\pi \in
\pathcat{G}(\setof{1_H}, T)$ with label $\alpha(\pi) = w$. 

For $w = x_1 \cdots x_n$ then obviously holds $\mu(x_1 \cdots x_n) \in T$, thus
$w \in \inv\mu(T) = L$ from which the claim follows.
\end{proof}

From both lemmata immediately follows

\begin{theorem}[Regular sets and recognizable sets coincide over free monoids]
\[ \reglang(X^*) = \reclang(X^*) \]
\end{theorem}

In the first section of this chapter we showed closure properties for regular
sets. We show now closure under inverse homomorphism which is rather easy using
our last theorem.

\begin{theorem}[Closure under inverse homomorphism]
Let $\phi : X^* \to Y^*$ be a monoid homomorphism.
\[ L \in \reclang(Y^*) \Rightarrow \inv\phi(L) \in \reclang(X^*) \]
\end{theorem}

\begin{proof}
Let $L \in \reclang{X^*}$, i.e.$L$ is defined by a monoid homomorphism
$\mu' : Y^* \to H$ as $L = \inv\mu'(T)$ for some subset $T \subset H$.

We set $\mu := \phi \circ \mu'$. Then $\mu : X^* \to H$ is also a monoid
homomorphism and $\inv\phi(L) = \inv\mu(T)$ from which follows $\inv\phi(L)
\in \reclang(X^*)$.
\end{proof}

\bigskip
\begin{exercise}
For the integer numbers $\mathbb{Z}$, the set of recognizable sets
$\reclang(\mathbb{Z})$ shall be defined in analogy to $\reclang(X^*)$.

Characterize $\reclang(\mathbb{Z})$! Are there non-empty, infinite sets in
$\reclang(\mathbb{Z})$?
\end{exercise}
