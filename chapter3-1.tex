\section{The finite automaton with pushdown store}

We investigate now a second example (after the 2-way automaton) that represents
a generalization of the finite automaton in our concept, the {\bf finite
automaton with pushdown store}.

For preparation we remember theorem 2 from chapter II.8

\begin{theorem}
For a finite automaton $\fa{A} = (G, \hgroup{X}, S, F, \alpha)$ it holds:
\[ X^* \cap |L_{\fa{A}}| \in REG(X^*) \]
\end{theorem}

Remark: From this theorem the following claims can be derived (exercise):
\begin{enumerate}
  \item With the same assumptions it holds $|L_{\fa{A}}| \in REG((X\cup
  X^{-1})^*)$
  \item Corresponding theorems hold for the free group $F(X)$ and the polycyclic
  monoid \pocymon{X} instead of \hgroup{X}.
\end{enumerate}
  
We want to introduce now the finite automaton with pushdown store. We define
\begin{definition}
$\mathcal{K} = (X, Y, \delta)$ is called {\bf simple pushdown store} with
input $X^*$ and stack-monoid $Y^*$ if
\begin{itemize}
  \item[(K1)] $\delta: X \times Y^* \to Y^*$ is a mapping with
  \begin{eqnarray*}
  \delta(X \times Y) & \subset & Y^2 \cup \{\epsilon\} \\
  \delta(X \times \{\epsilon\}) & \subset & Y
  \end{eqnarray*} 
  \item[(K2)] $\delta(x, u \cdot v) = u \cdot \delta(x, v)$ for $v \in Y^+,u \in
  Y^*$.
\end{itemize}
\end{definition} 

Obviously the simple pushdown store $\mathcal{K}$ is uniquely determined by
$\delta' = \delta |_{X \times (Y \cup \{\epsilon\})}$. Each such mapping can be
continued using (K2) onto $X \times Y^*$ such that one gets a simple
pushdown store.

We extend $\delta$ onto $X^* \times Y^*$ by defining
\begin{eqnarray*}
\delta(\epsilon, u) & = & u \\
\delta(x v, u) & = & \delta(v,  \delta(x,u))
\end{eqnarray*}

The following lemma holds:
\begin{lemma}
For $u \in \hgroup{X}$ there exists a pushdown store
\[ \mathcal{K} =(\unioninv{X}, \unioninv{X}, \delta) \]
with $|u| = \delta(u, \epsilon)$.
\end{lemma}

\begin{proof}
Define
\begin{eqnarray*}
\delta(x, \epsilon) & = & x,\ x\in \unioninv{X}\\
\delta(x, y) & = & x y,\ x\neq y^{-1}\\
\delta(x^{-1}, x) & = & \epsilon,\ x\in X
\end{eqnarray*}

Let $u\in (\unioninv{X})^*$ and $u = x_1^{\epsilon_1} \cdots x_n^{\epsilon_n}$
with $x_i\in X,\ \epsilon_i\in \{1, -1\}$.

For the sequence $y_1 = \delta(x_1^{\epsilon_1}, \epsilon),\ y_2 =
\delta(x_2^{\epsilon_2}, y_1), \ldots,\ y_n = \delta(x_n^{\epsilon_n}, y_{n-1})$
it holds: $y_n = |u|$ (exercise).
\end{proof}

Remark:
\begin{enumerate}
  \item The lemma holds also for the free group $F(X)$.
  \item The lemma does not hold for the polycyclic monoid $\pocymon{X}$
  (exercise).
\end{enumerate}

The reader should extend the definition of the pushdown store to a
pushdown store with 0 such that the lemma holds also for the polycyclic monoid
$\pocymon{X}$.

We can now define the finite automaton with pushdown store.

\begin{definition}
$\kappa = (\fa{A}, \mathcal{K})$ is called {\bf finite automaton with
pushdown store} or short {\bf PDA}, if $\fa{A} = (G, X, S, F, \alpha)$ is a
finite automaton with graph $G = (V, E)$ and $\mathcal{K} = (E, Y,\delta)$ is a
simple pushdown store.
\end{definition}

Two sets are assigned to the PDA $\kappa$, the {\bf pushdown class}
$KKL(\kappa)$ which is the set of possible words on the pushdown store when
$\fa{A}$ accepts a word, and the {\bf accepted language} $L_\kappa$ which is the
set of input words accepted with empty pushdown store.

Formally:
\begin{eqnarray*}
KKL(\kappa) &:=& \{ \delta(\pi, \epsilon) \mid \pi \in \pathcategory{G}(S, F) \} \\
L_\kappa &:=& \{ \alpha(\pi) \mid \pi \in \pathcategory{G}(S, F),\ \delta(\pi,
\epsilon) = \epsilon \}
\end{eqnarray*}

\begin{definition}
\[ ALG(X_\infty^*) := \{ L \subset X_\infty^* \mid \text{there exists a PDA
$\kappa$ with $L = L_\kappa$} \]
is called the class of {\bf algebraic languages}.
\end{definition}

The pushdown-class $KKL(\kappa)$ is an example for our concept of generalizing
the finite automata by changing the monoid. If we assign to each $x\in X$ a
mapping $\delta_x: Y^* \to Y^*$, then we can represent $\kappa$ by
\[ \fa{B} = (G, M, S, F, \beta) \]
where $M = \{ \delta_x \mid x\in X \}^*$ and $\beta(x) = \delta_x$.

It holds:
\[ L_{\fa{B}}(\epsilon) := \{ \beta(\pi)(\epsilon) \mid \pi \in
\pathcategory{G}(S, F) \} = KKL(\kappa) \]

As our main result we get
\begin{theorem}
\[ KKL(\kappa) \in REG(Y^*) \]
\end{theorem}

\begin{proof}
Before starting with the formal proof we want to explain the construction of the
automaton for $KKL(\kappa)$ using an example.

Let $\pi = (e_1 e_2 e_3 e_4 e_5 e_6 e_7 e_8)$ a path from \pathcategory{G} and the
pushdown store be empty initially.

$\delta$ shall be defined as follows:
\[\begin{array}{r@{\,=\,}l@{\qquad}r@{\,=\,}l@{\qquad}r@{\,=\,}l}
\delta(s_1,\epsilon)&y_1 & \delta(s_2,y_1)&y_1y_2 & \delta(s_3,y_2)&y_2y_3 \\
\delta(s_4,y_3)&\epsilon & \delta(s_5,y_2)&y_2y_4 & \delta(s_6,y_4)&\epsilon \\
\delta(s_7,y_2)&\epsilon & \delta(s_8,y_1)&y_1y_5
\end{array}\]

We assign to the computation of the pushdown store a word $w' \in (Y\cup
Y^{-1})^*$. The reduced word is the current content of the pushdown store,
hereby we have to guess in $w'$ the last symbol of the reduced word
(corresponds to the symbol on top of the stack).

Let a path $\pi \in \pathcategory{G},\ G=(V,E),$ be of the form
\[ \edge{e_1} \edge{e_2} \edge{e_3} \edge{e_4} \edge{e_5} \edge{e_6} \edge{e_7}
\edge{e_8} \]

In the automaton to be constructed we will have the following marking
\[ \edge{y_1} \edge{y_1^{-1} y_1 y_2} \edge{y_2^{-1} y_2 y_3} \edge{y_3^{-1}
y_2^{-1} y_2} \edge{y_2^{-1} y_2 y_4} \edge{y_4^{-1} y_2^{-1} y_2}
\edge{y_2^{-1} y_1^{-1} y_1} \edge{y_1^{-1} y_1 y_5}
\]

The first inverse on each edge label is used to ''guess'' the top symbol on
the stack.

The corresponding stack contents are:
\[ y_1 \qquad y_1 y_2 \qquad y_1 y_2 y_3 \qquad y_1 y_2 \qquad y_1 y_2 y_4
\qquad y_1 y_2 \qquad y_1 \qquad y_1 y_5 \qquad \]

By multiplying the edge labels and applying the cancellation rules one gets the
stack contents.

We want to execute this idea formally:

For the proof we construct an automaton $\fa{C}$ over the H-group $\hgroup{Y_0}$
where $Y_0 := Y \cup \{y_0\},\ y_0 \notin Y$ and the property
\[ Y_0^* \cap |L_{\fa{C}}| = y_0 \cdot KKL(\kappa) \]

From theorem 1 then follows that $y_0 \cdot KKL(\kappa)$ is regular and from
this follows immediately $KKL(\kappa) \in REG(Y^*)$.

\transrem{The proof in the book was completely unreadable without the remarks
and figures which have been added here.}

We define the finite automaton $\fa{C}$ as follows:
\[ \fa{C} = (G', \hgroup{Y_0}, S_0, F, \gamma) \]

The graph $G'=(V',E')$ is defined by $V' = V \cup \{s_0\},\ s_0 \notin V$,
\end{proof}




























