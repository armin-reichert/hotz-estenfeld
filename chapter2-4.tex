\section{Right-linear languages, $r{-}LIN(X^*)$}

In this section we investigate an additional method for defining special subsets
of $X^*$. We use the mechanism introduced in chapter 1.6, the Chomsky grammars.

We consider only a very restricted type of productions and obtain the {\bf
right-linear} grammars.

\begin{definition}[right-linear Chomsky grammar]
A Chomsky grammar $G = (N, X, P, S)$ is called
\begin{eqnarray*}
\mbox{\bf right-linear} & & \mbox{if } P \subset N \times (X \cdot N \cup X) \\
\mbox{\bf left-linear}  & & \mbox{if } P \subset N \times (N \cdot X \cup X)
\end{eqnarray*}
\end{definition}

$X$ is the {\bf terminal alphabet}.

As in chapter 1.6 we have the notions of {\em derivation} and {\em generated
language}.

\begin{definition}[right-linear languages]
\[ r{-}LIN(X^*) = \{ L \subset X^* \mid \mbox{there exists a right-linear
grammar $G$ with } L = L(G) \} \]
is the class of {\bf right-linear languages} in $X^*$.
\end{definition}

In the same way one defines the class of {\bf left-linear languages}.

The following theorem holds:

\begin{theorem}
\[ REG(X^*) = r{-}LIN(X^*) \]
\end{theorem}

Proof:

''$\supset$'': Let $L \in r{-}LIN(X^*)$ with $L = L(G),\ G = (N, X, P, S)$.

We construct a graph $G' = (V, E)$ with vertices $V = N \cup \{F\}, F \notin N$
and we use the productions of the grammar as edge set $E$:
\begin{eqnarray*}
p: v \to x v' \in P & \Rightarrow & e:  v \edge{x} v' \in E \\
p: v \to x \in P & \Rightarrow & e : v \edge{x} F \in E
\end{eqnarray*} 

(Translator remark: $e: v \edge{x} v'$ is just a shorter notation for $Q(e) =
v, \ Z(e) = v',\ \alpha(e) = x$)

This defines a finite acceptor $\fa{A} = (G', X, S, F, \alpha)$.

It is easily seen that
\[ w \in \pathcat{G'} \Rightarrow Q(w) \derives{G} \alpha(w) \cdot Z(w) \mbox{
 , if } Z(w) \in N\]

For paths $w \in \pathcat{G'}(S, F)$ we get $\alpha(w) \in L$.

From this it follows $\lang{A} \subset L(G)$.

Let $w \in L(G)$ be a word generated by the right-linear grammar $G$, then there
exists a sequence of derivation steps
\[ S \dderives{G} x_1 v_1 \dderives{G} x_1 x_2 v_2 \dderives{G} \ldots
\dderives{G} x_1 \cdots x_{n-1} v_{n-1} \dderives{G} w \]

Then there exists a path 
\[ w' = \big(S, (S,x_1 v_1), (v_1, x_2 v_2), \ldots, (v_{n-1}, x_{n-1} v_{n-1}),
(v_{n-1}, x_n), F\big) \in \pathcat{G'}(S, F) \]
and the label of this path is
\[ \alpha(w') = x_1 \cdots x_n = w \]

This means, the word $w$ is accepted by the finite acceptor, $w \in \lang{A}$.

Together, we get $\lang{A} = L(G)$.

''$\subset$'':

Let $L \in REG(X^*),\ L = \lang{A}$ with a finite acceptor with a single start
state $\fa{A} = (G, X, S_{\fa{A}}, F_{\fa{A}}, \alpha),\ G = (V, E)$ 
and $|\alpha(e)| = 1$ for each edge $e \in E$.

We define a right-linear grammar $RLG = (N, X, P, S)$ with $N = E, S =
S_{\fa{A}}$ and production set
\begin{eqnarray*}
P & = & \{ (q, \alpha(e) \cdot r) \mid \exists \mbox{ edge } e : q \to r \in E
\} \\
& \cup & \{ (q, \alpha(e)) \mid \exists \mbox{ edge } e : q \to r \in E,\ r \in
F_{\fa{A}} \}
\end{eqnarray*}

It holds $L(RLG) = \lang{A}$ (exercise) which completes the proof of the
theorem.

Remark: It also holds $l{-}LIN(X^*) = REG(X^*)$. This can be seen by theorem 2
in chapter 2.1 which states that $REG(X^*)$ is closed under the
mirror-operation.

Given a right-linear grammar for $L$ one can directly define a left-linear
grammar for the mirror language $L^R$: Replace each production $X \to t \cdot Y$
by a left-linear production $X \to Y \cdot t$. Because of ${L^R}^R = L$ it then
holds $L \in l{-}LIN(X^*)$.

We can therefore in all of the following results replace $r{-}LIN(X^*)$ by
$l{-}LIN(X^*)$.

We extend now our main theorems from section 1. Let $X_\infty$ again be the
countably infinite alphabet and let $RAT(X_\infty^*)$ and $REC(X_\infty^*)$ be
defined as before.

\begin{maintheorem}\leavevmode
\begin{enumerate}
  \item $ REG(X_\infty^*) = RAT(X_\infty^*) = REC(X_\infty^*) =
  r{-}LIN(X_\infty^*) = l{-}LIN(X_\infty^*) $
  \item $ REG(X_\infty^*)$ is closed under union, intersection,
  complement, Kleene-star, complex product, monoid homomorphisms and inverse
  homomorphisms
\end{enumerate}
\end{maintheorem}

Each of these language classes motivates a different generalization. We want to
shortly present those.

{\bf Regular languages:} Generalization into two directions come to mind: Replace the
free monoid $X^*$ by an arbitrary monoid or by special monoids like free groups or
free commutative monoids. We already pointed that out and we will investigate
two important cases later in this book.

The second generalization concerns the free (path) category. By adding a second
operation we get simply computable, infinite categories. The paths in graphs are
then replaced by trees or nets representing the morphisms of these new free 
category.

{\bf Rational languages:} Instead of the free monoid one can use arbitray monoids. We
already made some statements about this. We now already that this generalization
will coincide with a corresponding generalization of the regular sets.

In an even stronger generalization the monoid can be replaced by other algebras,
that is one adds other operations in addition to the monoid operation. The
question arises if in this case the strong relation between the rattional and
regular sets will be conserved.

{\bf Recognizable languages:} Here also two different directions for generalization
come to mind. First, one can keep monoids but replace finite monoids by simply
computable infinite monoids.

A second generalization concerns the replacement of monoids by other algebras.
Especially the free monoid can be replaced by free algebras and the finite
monoid by finite algebras.

{\bf Right-linear languages:} Here one generalizes by using general Chomsky
grammars (see definition 1, chapter 1.6). This last generalization will be in
our focus later.

Our main theorem motivates yet another generalization:

{\bf Abstract Families of Languages (AFL):} The AFL-theory generalizes the
notion of the family of rational languages. This is done by introducing the
notion of {\bf Kleene-algebra} which is based on the operations of union,
complex product and Kleene-star.

A set $\mathcal{A} \in Pot(X_\infty^*)$ is called a {\bf full AFL}, if
$\mathbb{A}$ is closed under the operations of union, complex product,
Kleene-star, homomorphisms and inverse homomorphisms and intersection with
regular languages.

The idea is the following:

All these operations are simple, therefore using these operations only ''simple
sets'' mya be constructed from some given set of ''simple sets''. It should be
possible to decide the common computer science questions for all elements in an
AFL if one can decide them for the generating basic sets.

In this book we cannot treat all these topics. We will not develop the
AFL-theory.

The interested reader is referred to the book by Jean Berstel \cite{Berstel79}
which handles this theory in great detail.

