\section{Notations, basic terminology}

In this first section we want to define the elementary terminology that is used
throughout the whole book. We use the usual notations
\begin{eqnarray*}
\mathbb{N} &=& \setof{0, 1, 2, \ldots}\text{ for the natural numbers} \\
\mathbb{Z} &=& \setof{\ldots, -2, -1, 0, 1, 2, \ldots}\text{ for the integer
numbers} \\
\mathbb{Q} &=& \setof{\frac{a}{b} \mid a,b \in \mathbb{Z},\ b \neq 0}\text{
for the rational numbers}
\end{eqnarray*}

For the operations on sets we use $\cup$ for the union and $\cap$ for the
intersection. Also $A \subset B$, $a \in A$, $a \not\in A$, $\bar{A}$, $A - B$,
$A \times B$ and $\emptyset$ have their usual meaning.

For the power set of a set $A$ we write $2^A$ or $Pot(A)$. $Card(A)$ denotes the
cardinality of $A$.

Logical implication is denoted by $\Rightarrow$.

{\bf Mappings} are denoted by $f : A \to B$ in which case $f$ is a
total mapping. We write $Q(f) = A, Z(f) = B$. 

\transrem{The letters $Q, Z$ stem from the German words ''Quelle'' (source)
and ''Ziel'' (target).}

If $f: A \to B,\ g : B \to C$ are mappings, then $f \circ g : A
\to C$ is the {\bf composition} which one gets by applying
\boldmath $f$ {\bf first and then} $g$ \unboldmath: 
\[(f \circ g)(a) = g(f(a))\]
If $f:A \to B$ and $C \subset A$, then $f(C) = \setof{f(c) \mid c \in C }$.

\transrem{Note that this is backwards to the usual definition of function
composition but makes perfect sense in the category theoretic framework.}

A subset $R \subset A \times B$ is called a {\bf relation} between $A$ and $B$.
\[R_f = \setof{(a,b) \mid b = f(a) } \subset A \times B\] 
is the relation {\bf induced by} the mapping f or also the {\bf graph} of $f$.

Let $f : A \to B$ be a mapping, $A_1 \subset A$ and $g : A_1 \to
B$ a mapping. 

$f$ is called the {\bf continuation} of $g$ if $f(a_1) = g(a_1),\ a_1 \in A_1$.
In this case we also write $f \mid _{A_1} = g$, in words: $f$ {\bf restricted to} $A_1$.

A {\bf semi-group} consists of a set $M$ and an associative operation on that
set, usually denoted as a multiplication. If a semi-group is commutative, we
also use ''$+$'' instead of ''$\cdot$''.

\index{monoid}
A semi-group is a {\bf monoid} if $M$ contains a neutral element. We often
denote the neutral element with $1_M$ or shortly $1$. In the commutative case we
often write $0$ instead of $1$.

For $A,B \subset M$ we denote by 
\[ A \cdot B = \setof{a \cdot b \mid a \in A,\ b \in B }\]
the {\bf complex product} of $A$ and $B$.

A subset $A \subset M$ is a {\bf submonoid} of $M$ if $1_M \in A$ and $A$ is
closed under the operation of $M$.

For a set $A$, the set $A^*$ is the smallest submonoid of $M$ containing $A$.
More specific, 
\[ A^* = \bigcap_{U \in M(A)} U	\]
where 
\[ M(A) = \setof{U \subset M \mid U \text{ is a submonoid of } M, A \subset U } \]

It is easy to see that
\[ A^* = \bigcup_{n \geq 0} A^n \text{ with }A^0 = \setof{1}\text{ and }A^{n+1}= A^n \cdot A \]

In the same sense the notion $A^+ = A^* - \setof{1}$ is defined for semi-groups.
$A$ is called the {\bf generating system} of $A^*$ and $A^+$ resp.

A special meaning for us has the set of {\bf words} ({\bf strings}) over
a fixed alphabet $A$. We understand as words the finite sequences of elements from
the alphabet $A$, for example $(a,b,d,a,c)$ over the alphabet $A = \setof{a,b,c,d
}$.

We define
\[ WORD(A) := \setof{\epsilon} \cup A \cup (A \times A) \cup (A \times A \times
A) \cup \ldots \]
as the {\bf set of words (strings) over} $A$. 

The symbol $\epsilon$ denotes the {\bf empty word} over $A$, that is $A^0 =
\setof{\epsilon}$.

If $v, w \in WORD(A)$ then $v \cdot w$ is the word which you get by
concatenating $v$ and $w$, formally:
\[ v = (a_1,\ldots, a_k),\ w = (a_{k+1}, \ldots, a_n) \Rightarrow v \cdot
w = (a_1, \ldots, a_n) \]

With this operation, $WORD(A)$ becomes a monoid which is usually also denoted
by $A^*$.

This naming is slightly inconsistent because for the first definition of
the $*$-operator it holds $(A^*)^* = A^*$, but for the second usage of the
$*$-operator it holds $(A^*)^* \neq A^*$.

The following example should clarify this fact: 

Let $A = \setof{a,b,c}$ and let $(a,b,a)$ and $(b,a) \in A^*$.
\[(a,b,a)\cdot(b,a) = (a,b,a,b,a) \in A^* \]
\[((a,b,a),(b,a)) \in (A^*)^*\text{, but }\notin A^* \]

Instead of $(a)$ we just write $a$. In this sense it holds $A \subset A^*$. This
also holds in the sense of the first definition of $A^*$.

If $w = (w_1, \ldots, w_n)$ we call $|w| := n$ the {\bf length} of $w$.
Obviously \[|w \cdot v| = |w| + |v|\text{ and }|\epsilon| = 0\]

\transrem{Later in this book, the notation $|w|$ is also used for reduced
words, but that should not lead to confusion.}

The {\bf reverse} of a word $w = (w_1,\ldots,w_n)$ is the word $w^R =
(w_n,\ldots,w_1)$. It holds:
\[ (w \cdot v)^R = v^R \cdot w^R \text{ and }\epsilon^R = \epsilon \]

In $A^*$ the {\bf cancellation rules} hold:
\begin{enumerate}
  \item $w \cdot x = w \cdot y \Rightarrow x = y$
  \item $x \cdot w = y \cdot w \Rightarrow x = y$
\end{enumerate}

We define {\bf left} and {\bf right quotient}:
\[ X^{-1} \cdot Y = \setof{w \mid \exists x \in X,\ y \in Y\text{ with }x \cdot w
= y } \] 
and 
\[ X \cdot Y^{-1} = \setof{w \mid \exists x \in X,\ y \in Y\text{ with } w \cdot y
= x } \]

Because of the cancellation rules it holds: $\setof{w}^{-1} \cdot \setof{v}$ and
$\setof{w}^{-1} \cdot \setof{v}$ are either empty or contain a single element.

If $\setof{w}^{-1} \cdot \setof{v}$ is not empty, we call $w$ a {\bf prefix} of $v$,
if $\setof{w} \cdot \setof{v }^{-1} \neq \emptyset$, we call $v$ a {\bf suffix}
of $w$.

In the future we will also just write $w$ instead of $\setof{w}$ and $w$ {\bf is
prefix of} $v$, if $w^{-1} \cdot v \neq \emptyset$.
