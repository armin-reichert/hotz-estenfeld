\section{Monoid homomorphisms and congruence relations}

\begin{definition}
A {\bf monoid homomorphism} (short: homomorphism) from a monoid $M$ to a monoid
$S$ is a mapping $\phi : M \to S$ with:
\begin{enumerate}
  \item $\phi(m_1 \cdot m_2) = \phi(m_1) \cdot \phi(m_2), \quad m_1, m_2 \in M$
  \item $\phi(1_M) = 1_S$
\end{enumerate}
\end{definition}

\transrem{Some publications use even shorter just {\em morphism} but here this
would lead to a conflict with the morphisms in categories.}

It can be easily shown: if $U \subset M$ is a submonoid of $M$, then
$\phi(U)$ is a submonoid of $S$. If $V$ is a submonoid of $S$, then
$\phi^{-1}(V)$ is a submonoid of $M$.

A monoid homomorphism $\phi: M \to S$ is called
\begin{eqnarray*}
\text{\bf monomorphism} & & \text{ if $\phi$ is injective} \\
\text{\bf epimorphism} & & \text{ if $\phi$ is surjective} \\
\text{\bf isomorphism} & & \text{ if $\phi$ is bijective} \\
\text{\bf endomorphism} & & \text{ if $\phi: M\to M$} \\
\text{\bf automorphism} & & \text{ if $\phi$ is endomorphism and isomorphism} \\
\end{eqnarray*}
Monoids $M$ and $S$ are called {\bf isomorphic}, if there exists an
isomorphism between $M$ and $S$.

Of course, a homomorphism cannot be defined arbitrarily on a monoid $M$.
Thus the following two questions arise:
\begin{enumerate}
  \item If $U \subset M$ is a submonoid and $\phi_1 : U \to S$ is an arbitrary
mapping. When is it possible to extend $\phi_1$ to a homomorphism $\phi
: M \to S$\ ?
	\item If $\phi_1, \phi_2$ both are homomorphisms from $M$ to $S$ which
	coincide on $U \subset M$. In which way can $\phi_1$ and $\phi_2$ be
	different? 
\end{enumerate}

The answer to this question of course depends on the structure of $U$. If
$U = \{ 1_M \}$ then $\phi$ is determined uniquely on $U$ but there is
little information on the relation between $\phi_1$ and $\phi2$.

The following two simple theorems which can be found in introductory algebra
books hold:
\begin{enumerate}
  \item If $A$ is a generating system of $M$ and $\phi_1, \phi_2 : M \to S$ are
  monoid homomorphisms which coincide on $A$, then $\phi_1 = \phi_2$.
  \item If $A$ is a set, $M = A^*$, and $\phi_1 : A \to S$ is an arbitrary
  mapping, then there exists a unique continuation $\phi$ of $\phi_1$ which is
  a monoid homomorphism from $A^*$ to $S$.
\end{enumerate}

\begin{definition}
A subset $A \subset M$ is called a {\bf free generating system} of $M$, if each
mapping $\phi_1 : A \to S$, where $S$ is an arbitrary monoid, can be continued
to a monoid homomorphims in a unique way. 

A monoid with a free generating system is called a {\bf free monoid}.
\end{definition}

$A^*$ therefore is a free monoid and $A$ is a free generating system of $A^*$.

It holds too: If $A$ is a free generating system of $M$ and $A^*$ is the monoid
of words (string) over $A$, then $A^*$ and $M$ are isomorphic.

A free monoid has at most one free generating system (exercise. From that one
can see that the length $|w|$ of a word $w \in A^*$ can be defined in a unique way for any
free monoid.

The length function $L$ is an example for a monoid homomorphism $L : A^* \to
\mathbb{N}$.

If $\phi : M \to S$ is a monoid homomorphism, then the sets 
\[\{ \phi^{-1}(s) \mid s \in S \} \subset Pot(M) \]
form a monoid isomorphic to $\phi(M)$ (exercise).

We want to investigate now the following problem:

Let $M$ be a monoid, $L \subset M$ be any subset of $M$. Does there exist a
monoid $S$ and a homomorphism $\phi : M \to S$ with the following property:
There exists an $s \in S$ with $L \subset \phi^{-1}(s)$?

Of course, there always exists such an $S$: Choose $S = \{ 1 \}$ and $\phi(M) =
\{1\}$. Therefore we strengthen our task: 

Find $S$ and $\phi$ such that $L \subset \phi^{-1}(S)$ and for each other
homomorphism $\Psi$ with that property holds:
\[ L \subset \Psi^{-1}(S') \Rightarrow \phi^{-1}(S) \subset \Psi^{-1}(S') \]

We want to describe $L$ as close as possible by a monoid homomorphism.

Such an $S$ and $\phi$ exists for each $L \subset M$ (see Algebra text), it is
named $synt_M(L)$ an is constructed as follows:

\begin{definition}
Let $M$ be a monoid and $L \subset M$. For $a, b \in M$ we define
\[ a \equiv b (L) \iff \text{for all }u, v \in M: u \cdot a \cdot v \in L
\Leftrightarrow u \cdot b \cdot v \in L \]
\end{definition}

$\equiv (L)$ is a congruence relation, it holds:
\begin{enumerate}
  \item Let $[a]_L = \{ b \in M \mid a \equiv b\ (L) \}$\ then\ $b \in [a]_L
  \Rightarrow [a]_L = [b]_L $
  \item If we define $[a]_L \cdot [b]_L := [a \cdot b]_L$\ (complex product),
  then \[synt_M(L) = \{ [a]_L \mid a \in M \}\] becomes a monoid under this
  operation and the mapping \[\phi_L : M \to synt_M(L),\ \phi_L(a) = [a]_L\]
   is a monoid epimorphism.
\end{enumerate}

We call $\equiv (L)$ the {\bf syntactic congruence} of $L$ and $synt_M(L)$ the
{\bf syntactic monoid} of $L$ with respect to $M$.

To motivate the name {\em syntactic monoid} we give an example from German
language.

Let $A$ be the alphabet of German and $L$ the set of sentences in German. One can
denote two words $w_1$ and $w_2$ as congruent if they can always be exchanged in
each german sentence. There exist words that cannot always be exchanged. In the
sentence ''Apfel ist eine Kernfrucht'' the word ''Apfel'' can be exchanged by
''Birne'' but this is not possible in the sentence ''Apfel schreibt sich A p f e
l''.

The difficulty is of semantic nature. If you don't consider semantic correctness
of sentences you get a classification of words wrt. their syntactic meaning.

\transrem{I fear this example makes not a lot of sense in this translation}

The important notion of ''syntactic congruence'' has been introduced by M.\ P.\
Schützenberger in the context of coding problems.

{\bf Exercise:} Prove the following 
\begin{theorem}
It is decidable if a monoid homomorphism $\alpha$ is injective.
\end{theorem}

Hint: Let $\alpha: A^* \to B^*$ be a given monoid homomorphism, $A =
\{a_1,\ldots,a_n\},\ E = \{\alpha(a_1),\ldots,\alpha(a_n)\}$.

Consider $A_1 := \{ u^{-1} v \in B^* \mid u, v \in E,\ u\neq v \}$ and then
inductively $A_{k+1} = A_k^{-1} E \cup E A_k$.

Prove that
\begin{enumerate}
  \item The construction of the $A_k$ halts
  \item $1 \in \cup A_k \iff \alpha\text{ is not injective}$.
\end{enumerate}


\begin{corollary}
In a free monoid it is decidable if a finite set is a free generating system.
\end{corollary}
