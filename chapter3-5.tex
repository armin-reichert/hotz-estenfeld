\section{The deterministic finite automaton with storage}

In this section we will prove a very important theorem about the relation
between non-determinism and determinism of finite automata with storage.

To be able to construct the deterministic automaton we need some definitions.

\begin{definition}
A finite automaton with storage $\fa{P}=(\fa{A},\pdstore{P})$ is called
\begin{itemize}
  \item {\bf $\epsilon$-free} if $\fa{A}$ is $\epsilon$-free,
  \item {\bf deterministic} if $\fa{A}$ is deterministic,
  \item {\bf complete} if $\fa{A}$ is complete.
\end{itemize}
\end{definition}

\bigskip
\begin{definition}
$(S, +, \cdot, 0, 1)$ is called a {\bf semi-ring} if
\begin{enumerate}
  \item $(S, +, 0)$ is a commutative monoid
  \item $(S, \cdot, 1)$ is a monoid
  \item The distributive laws hold:
  \begin{eqnarray*}
  a \cdot (b + c) &=& a \cdot b + a \cdot c\\
  (a + b) \cdot c &=& a \cdot c + b \cdot c\\
  \end{eqnarray*}
  \item $0 \cdot a = a \cdot 0 = 0$ for all $a \in S$
\end{enumerate}
\end{definition}

\bigskip
\begin{definition}
Let $M$ be a monoid and $S$ a semi-ring.
\[ S(M) := \setof{\lambda: M \to S \mid \card{\setof{m \mid \lambda(m) \neq 0}}
< \infty} \]
is called the {\bf monoid-ring} of the monoid $M$ over the semi-ring $S$.
\end{definition}

Addition and multiplication on $S(M)$ are defined as follows: For $\lambda_1,
\lambda_2 \in S(M)$
\begin{eqnarray*}
(\lambda_1 + \lambda_2)(m) &=& \lambda_1(m) + \lambda_2(m) \\
(\lambda_1 \cdot \lambda_2)(m) &=& \sum_{m_1 \cdot m_2 = m} 
\lambda_1(m) \cdot \lambda_2(m)
\end{eqnarray*}

One can see that $(S(M), +, \cdot)$ again is a semi-ring. We also write
\[ \lambda := \sum_{\substack{m \in M\\\lambda(m)\neq 0}} m \cdot \lambda(m) \]
or also
\[ \sum_{\substack{m \in M\\\lambda(m)\neq 0}} m \cdot {<}\lambda, m{>} \]

The operations are defined in a way that one can calculate with these
expressions as usual.

In the following we will most often consider the semi-ring $S = \mathbb{B}$, the
{\bf Boolean semi-ring}.

\bigskip
Before proving our main theorem we will show some other results on finite
automata with storage.

\bigskip
First we want to give a representation for graphs in the polycyclic monoid
$\pocymon{Y}$.

Let $G=(V, E)$ be a finite graph with $\card{E} = n$, let $\card{Y} = 2$ and
$2^k \geq n$.

Let $\beta: V \to Y^*$ be an injective mapping and let $\strlen{\beta(v)} = k$
for each vertex $v \in V$.

Further let $G' = (\setof{v_0}, E')$ be a single-vertex graph and $\phi: E \to
E'$ a bijection on the edges, and let $Q(e') = Z(e') = v_0$ for each edge $e'
\in E'$.

The mapping $\phi$ can be uniquely continued to a functor $\phi : \pathcat{G}
\to \pathcat{G'}$. As a functor, $\phi$ is also injective.

We complement $G'$ by a storage $\gamma: E' \to \pocymon{Y}$ by setting
\[ \gamma(e') := u^{-1} \cdot v\text{ where }u = \beta(Q(\phi^{-1}(e')))\text{
and }v = \beta(Z(\phi^{-1}(e'))) \]

In this way the storage ''characterizes'' the paths $\phi(\pathcat{G})$.

Obviously it then holds
\begin{lemma}
\begin{eqnarray*}
\pi \in \phi(\pathcat{G}) &\Leftrightarrow& \gamma(\pi) \neq 0 \\
\gamma(\pi) \neq 0 &\Rightarrow& \gamma(\pi) = u^{-1} \cdot v \\
&& \text{with }u = \beta(Q(\phi^{-1}(e'))) \\
&& \text{and }v = \beta(Z(\phi^{-1}(e'))) \\
&& \text{and }\pi = e'_1 \circ \pi' \circ e'_k
\end{eqnarray*}
\end{lemma}

\begin{proof}
''$\Rightarrow$'':

Let $\pi \in \pathcat{G}$ be a path, $\pi = e'_1 \ldots e'_k \Rightarrow Z(e'_i)
= Q(e'_{i+1})$.

There exists a path $\psi \in \pathcat{G}$ such that $\psi = e_1 \ldots e_k$
with $\phi(\psi) = \pi$.
\begin{eqnarray*}
&\Rightarrow& \beta(Z(\phi^{-1}(e'_i))) = \beta(Q(\phi^{-1}(e'_{i+1}))) \\
&\Rightarrow& \beta(Z(\phi^{-1}(e'_i))) = \beta(Q(\phi^{-1}(e'_{i+1}))) \\
\end{eqnarray*}
\end{proof}






















