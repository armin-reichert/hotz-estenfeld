\section{The finite 2-way automaton}

We start now with generalizing the concept of the finite automaton as described
in chapter II.4.

We consider a finite automaton where the reading head can traverse the input
tape in two directions as indicated by the following figure:

\begin{center}
\begin{tikzpicture}
	\begin{pgfonlayer}{nodelayer}
		\node [style=none] (0) at (-1, -0.5) {};
		\node [style=none] (1) at (0, -0.5) {};
		\node [style=none] (2) at (-3, -0) {};
		\node [style=none] (3) at (-3, -0.5) {};
		\node [style=none] (4) at (2, -0) {};
		\node [style=none] (5) at (2, -0.5) {};
		\node [style=none] (6) at (-1, -0.5) {};
		\node [style=none] (7) at (0, -0.5) {};
		\node [style=none] (8) at (-1.5, -1.75) {};
		\node [style=none] (9) at (0.5, -1.75) {};
		\node [style=none] (10) at (-1.5, -3.25) {};
		\node [style=none] (11) at (0.5, -3.25) {};
		\node [style=none] (12) at (-0.5, -1) {};
		\node [style=none] (13) at (-0.5, -1.75) {};
		\node [style=none] (14) at (-0.5, -2.5) {$v$};
		\node [style=none] (15) at (3, -0.25) {Input tape};
		\node [style=none] (16) at (3, -1) {Two-way reading head};
		\node [style=none] (17) at (3, -2.5) {Finite state control in state $v$};
		\node [style=none] (18) at (-1.5, -1) {};
		\node [style=none] (19) at (0.5, -1) {};
		\node [style=none] (20) at (-2, -1) {};
		\node [style=none] (21) at (-2, -0) {};
		\node [style=none] (22) at (-2, -0.5) {};
		\node [style=none] (23) at (-1, -0) {};
		\node [style=none] (24) at (0, -0) {};
		\node [style=none] (25) at (1, -0) {};
		\node [style=none] (26) at (1, -0.5) {};
	\end{pgfonlayer}
	\begin{pgfonlayer}{edgelayer}
		\draw [style=simple] (3.center) to (0.center);
		\draw [style=simple] (0.center) to (1.center);
		\draw [style=simple] (1.center) to (5.center);
		\draw [style=simple, bend right=90, looseness=1.50] (6.center) to (7.center);
		\draw [style=simple] (6.center) to (7.center);
		\draw [style=simple] (8.center) to (9.center);
		\draw [style=simple] (9.center) to (11.center);
		\draw [style=simple] (11.center) to (10.center);
		\draw [style=simple] (10.center) to (8.center);
		\draw [style=simple] (12.center) to (13.center);
		\draw [style=transition] (12.center) to (18.center);
		\draw [style=transition] (12.center) to (19.center);
		\draw [style=simple] (2.center) to (4.center);
		\draw [style=simple] (21.center) to (22.center);
		\draw [style=simple] (23.center) to (0.center);
		\draw [style=simple] (24.center) to (1.center);
		\draw [style=simple] (25.center) to (26.center);
		\draw [style=simple] (4.center) to (5.center);
		\draw [style=simple] (2.center) to (3.center);
	\end{pgfonlayer}
\end{tikzpicture}
\end{center}

The representation of such an automaton by a state graph is not trivial. For
this reason, we  define this type of automaton as usual at first.

Afterwards, we will show how to represent this automaton by a state graph.

\begin{definition}
A {\bf finite two-way automaton} is defined by
\[ \fa{B} = (X, Z, \delta, S, F, \setof{l, r}) \]
where 
\begin{itemize}
  \item $X$ is the input alphabet with $\setof{l, r} \cap X = \emptyset$
  \item $Z$ is the state set with $\setof{l, r} \cap X = \emptyset$
  \item $S \subset Z$ is the set of start states
  \item $F \subset Z$ is the set of final states of $\fa{B}$
  \item $\delta \subset (X \times Z) \times (Z \times \setof{l, r})$ is a
  relation describing the functioning of $\fa{B}$
\end{itemize}

If $\delta : (X \times Z) \to Z \times \setof{l, r}$ is a mapping, then $\fa{B}$
is called a {\bf complete, deterministic finite two-way automaton}.
\end{definition}

\bigskip
\begin{definition}
A {\bf computation} of $\fa{B}$ on the input tape with label $v \in X^*$ is a
sequence
\[ f_k = (z_0 v \to u_1 z_1 v_1 \to u_2 z_2 v_2 \to \ldots \to u_k z_k v_k) \]
with $u_i, v_i \in X^*,\ z_i \in Z$ for $i = 1, \ldots, k$ and $z_0 \in S$. 
\end{definition}

$u_i z_i v_i$ transfers in a single step into $u_{i+1} z_{i+1} v_{i+1}$ if the
following holds:

\begin{enumerate}
  \item If $v_i = a v'_i$, then $u_{i+1} = u_i a,\ v_{i+1} = v'_i$, if $(z_{i+1}, r)
\in \delta(a, z_i)$, i.e.\ the reading head moves to the right and the
automaton changes into state $z_{i+1}$, if it reads in state $z_i$ a symbol $a$,
or
\item if $u_i = u'_i b,\ v_i = a v'_i$, then $u_{i+1} = u'_i$ and $v_{i+1} = b
v_i$, if $(z_{i+1}, l) \in \delta(a, z_i)$, i.e.\ the reading head moves to the
left and the automaton changes into state $z_{i+1}$, if it reads in state $z_i$
a symbol $a$.
\end{enumerate}

\bigskip
\begin{definition}
A word $w \in X^*$ is {\bf accepted by} $\fa{B}$ if there exists a computation
$f_k$ with $v_k = \epsilon$ and $z_k \in F$.
\end{definition}

We set
\[ L_{\fa{B}} = \setof{w \in X^* \mid w\text{ is accepted by }\fa{B}} \]

We want to consider again the monoid $\hgroup{X}$, see chapter I.3.

We construct an automaton
\[ \fa{D} = (G, \hgroup{X}, S, F, \alpha) \]
with graph $G=(V, E)$ as follows:
\begin{itemize}
  \item $V = Z \times \setof{l_1, l_2, r_1, r_2}$
  \item $E$ and $\alpha: E^* \to \hgroup{X}$ is defined as follows:
  
  $E$ contains an edge $e: (z, y) \edge{a} (z', y') \iff$ one of the following
  cases occurs:
  \begin{enumerate}
    \item $y = r_2,\ y' = r_2,\ (z', r) \in \delta(a, z)$
    \item $y = r_2,\ y' = l_1,\ (z', l) \in \delta(a, z)$
    \item $y = r_1,\ y' = r_2,\ z = z'$ and there exists $\bar{z}$ with $(z, r)
    \in \delta(a, \bar{z})$
  \end{enumerate}

  $E$ contains an edge $e: (z, y) \edge{\inv{a}} (z', y') \iff$ one of the
  following cases occurs:
  \begin{enumerate}
    \item $y = l_2,\ y' = l_2,\ (z', l) \in \delta(a, z)$
    \item $y = l_2,\ y' = r_1,\ (z', r) \in \delta(a, z)$
    \item $y = l_1,\ y' = l_2,\ z = z'$ and there exists $\bar{z}$ with $(z, l)
    \in \delta(a, \bar{z})$
  \end{enumerate}
\end{itemize}

\bigskip
To clarify this definition we use the following visualization:

The input tape of the automaton is divided alternatingly into rectangular and
oval cells.

\missingfigure

The ovals are ''idle-positions'' for the reading head, the rectangles
each contain a symbol of the input alphabet.

The reading head moves from a ''idle-position'' to the ''idle-position'' to the
left or to the right.

If it moves from left to right over the cell $k$ with symbol $a$, then the
automaton reads $a$, if to the opposite it moves from right to left over that
cell, it reads $\inv{a}$. (This kind of reading when a cell is traversed
corresponds to the physical process when for example using a magnetic disk).

In some cases, our geometric model has to perform two computation steps, when
the automaton $\fa{B}$ performs only a single step.

The following cases may occur:

\begin{enumerate}
  \item The automaton is in idle position to the left of a cell $k$. It
  remembers that it should move to the right, i.e.\ it is in some state $(z,
  r_2)$. It traverses now the $k$th cell, reads the symbol $a$, reaches the
  subsequent idle position and finds that it should have gone to the left. This
  is desribed by the state $(z', l_1)$. Therefore it moves back to its previous
  idle position and changes into state $(z', l_2)$. Together it has read $a
  \inv{a}$.
  
  This kind of movement can be expressed using the following diagram:
  \[ (z, r_2) \edge{a} (z', l_1) \edge{\inv{a}} (z', l_2) \]
  
  \item In the same way, the following diagram is to be understood:
  \[ (z, l_2) \edge{\inv{a}} (z', r_1) \edge{a} (z', r_2) \]
\end{enumerate}

We complete the definition of $\fa{D}$ by setting
\[ S = S_{\fa{B}} \times \setof{r_2},\quad F = F_{\fa{B}} \times \setof{r_2} \]

We denote again by $|u| \in (\unioninv{X})^*$ the reduced word for an element
$[u] \in \hgroup{X}$.

Then we can define:

\begin{definition}
\begin{eqnarray*}
\genbyone{L_{\fa{D}}} &:=& \{ |\alpha(\pi)| \mid  |\alpha(\pi)| \in
X^*,\ \pi \in \pathcat{G}(S, F) \\
&& \text{and for all paths prefixes }\omega \prefix \pi \text{ holds: }
|\alpha(\omega)| \prefix |\alpha(\pi)| \}
\end{eqnarray*}

Here the relation $\prefix \subset \pathcat{G} \times \pathcat{G}$ denotes the
path prefix relation, and the relation $\prefix \subset (\unioninv{X})^* \times
(\unioninv{X})^*$ denotes the word prefix relation.
\end{definition}

\bigskip
First, we want to give an example to clarify the construction given above.

We have a finite two-way automaton which accepts the language
\[ L = \setof{w \in \setof{a, b}^+ c \setof{a, b}^+ \mid w = w_1 \cdots w_n
\text{ with } w_i = c \Rightarrow w_{i-1} = w_{i+1},\ 1 \leq i \leq n} \]

The transition function shall be defined as follows:
\begin{eqnarray*}
\delta(a, z_0) &=& \delta(a, z_1) = (z_1, r) \\
\delta(b, z_0) &=& \delta(b, z_1) = (z_1, r) \\
\delta(c, z_1) &=& (z_2, l) \\
\delta(a, z_2) &=& (z_a, r) \\
\delta(b, z_2) &=& (z_b, r) \\
\delta(c, z_a) &=& (z_{a'}, r) \\
\delta(c, z_b) &=& (z_{b'}, r) \\
\delta(a, z_{a'}) &=& (z_e, r) \\
\delta(b, z_{b'}) &=& (z_e, r) \\
\delta(a, z_{e}) &=& \delta(b, z_e) = (z_e, r)
\end{eqnarray*}

The two-way automaton $\fa{A}$ is defined by
\[ \fa{A} = (\setof{z_0, z_1, z_2, z_a, z_b, z_{a'}, z_{b'}, z_e}, \setof{a,
b, c}, \setof{z_0}, \setof{z_e}, \delta) \]

From this we want to construct our automaton 
\[ \fa{D} = (G, \hgroup{\setof{a, b, c}}, \setof{(z_0, r_2)}, \setof{(z_e,
r_2)}, \alpha) \]

The graph $G = (V, E)$ has the following form (the markings at the edges are
the edge labels):

\missingfigure

We begin with proving the equivalence between both automata models. To do so, we
prove the following lemma.

\begin{lemma}
\[ L_{\fa{B}} \subset \genbyone{L_{\fa{D}}} \]
\end{lemma}

\begin{proof}
We assign to each computation $f_k,\ k \in \mathbb{N}$ of the automaton $\fa{B}$
a path $\pi \in \pathcat{G}(S, V)$. We do that by induction over the length of
the computations.

{\em Induction base:}

To the empty computation $f_0$ we assign the empty path $1_{(z_0, r_2)}$ where
$z_0 \in S_{\fa{B}}$ is a start state of $\fa{B}$.

It holds: $\alpha(1_{(z_0, r_2)}) = \epsilon = u_0$.

{\em Induction hypothesis:}

For all computations $f_k$ of length $k$ we assume to have a path $\pi_k \in
\pathcat{G}(S, V)$ such that $Z(\pi_k) \in Z \times \setof{(r_2, l_2)}$ and
\[ |\alpha(\pi_k)| = \begin{cases}
u_k & \text{for }Z(\pi_k) \in Z \times \setof{r_2} \\
u_k \cdot First(v_k) & \text{for }Z(\pi_k) \in Z \times \setof{l_2}
\end{cases} \]

Here $First(w)$ denotes the first character of the word $w$.

Let $f_{k+1}$ be a computation of length $k+1$ which ends with the computation
step
\[ u_k z_k v_k \to u_{k+1} z_{k+1} v_{k+1} \]

\begin{itemize}
  \item[Case 1:] $Z(\pi_k) = (z_k, r_2),\ v_k = a \bar{v_k},\ (z_{k+1}, r) \in
  \delta(a, z_k)$
  
  By construction, $G$ contains an edge $e: (z_k, r_2) \edge{a} (z_{k+1}, r_2)$.
  
  We set $\pi_{k+1} := \pi_k \cdot e$ and get
  \[ |\alpha(\pi_{k+1})| = |\alpha(\pi_k)| \cdot a = u_{k+1} \]
  
  \item[Case 2:] $Z(\pi_k) = (z_k, r_2),\ u_k = \bar{u}_k b,\ v_k = a
  \bar{v_k},\ (z_{k+1}, l) \in \delta(a, z_k)$
  
  By construction, in $G$ there exists a path
  \[ (z_k, r_2) \edge{e_1 / a} (z_{k+1}, l_1) \edge{e_2 / \inv{a}} (z_{k+1},
  l_2),\ e_1, e_2 \in E \]
  
  We set $\pi_{k+1} = \pi_k \cdot e_1 \cdot e_2$, then it is
  \[ |\pi_{k+1}| = |\pi_k| = u_{k+1} \cdot b = u_{k+1} \cdot First(v_{k+1}) \]
  and $Z(\pi_{k+1}) \in Z \times \setof{l_2}$.
  
  \item[Case 3:] $Z(\pi_k) = (z_k, l_2),\ v_k = a \bar{v_k},\ (z_{k+1}, r) \in
  \delta(a, z_k)$
  
  In $G$ there exists a path 
  \[ (z_k, l_2) \edge{e_1 / \inv{a}} (z_{k+1}, r_1) \edge{e_2 / a} (z_{k+1},
  r_2),\ e_1, e_2 \in E \]
  
  We set $\pi_{k+1} = \pi_k \cdot e:1 \cdot e_2$, then it is $Z(\pi_{k+1}) \in
  Z \times \setof{r_2}$ and
  \[ |\alpha(\pi_{k+1})| = |u_k a \inv{a} a| = u_k a = u_{k+1} \]
  
  \item[Case 4:] $Z(\pi_k) = (z_k, l_2),\ u_k = \bar{u_k} b,\ v_k = a
  \bar{v_k},\ (z_{k+1}, l) \in \delta(a, z_k)$
  
  In $G$ there exists the edge $e: (z_k, l_2) \edge{\inv{a}} (z_{k+1}, l_2)$.
  
  We set $\pi_{k+1} = \pi_k \cdot e$, so $Z(\pi_{k+1}) = (z_{k+1}, l_2)$ and it
  is
  \[ |\alpha(\pi_{k+1})| = |\alpha(\pi_k)| \cdot \inv{a} = |u_k \cdot a \inv{a}|
  = |u_k| = u_{k+1} \cdot First(v_{k+1}) \]
\end{itemize}

These are all possible cases for $\pi_{k+1}$ and $\pi_{k+1}$ fulfills the
induction hypothesis for $k+1$.

Let now $f_n$ be an accepting computation for $v$, so $v_n = \epsilon,\ u_n = v$
and $z_n \in F$.

The last move of $\fa{B}$ was a right-move. It holds $Z(\pi_n) = (z_n, r_2)$ and
therefore it holds $|\alpha(\pi_n)| = u_n = v$.

For each path $\pi_n$ constructed in that way it holds
\[ \forall \omega \prefix \pi_n \text{ holds }|\alpha(\omega)| \prefix
|\alpha(\pi_n)| \]

Because $f_n$ is an accepting computation, i.e.\ the reading head is placed to
the right of the right-most input symbol, each input symbol must have been
traversed one time more often to the right than to the left.

This means, if a subpath $e \cdot e'$ with label $\alpha(e \cdot e') = a
\inv{a}$ exists in $\pi_n$, then it must be followed by an edge $e''$ with
$\alpha(e'') = a$ in $\pi_n$.

The prefix condition of $\pi_n$ therefore cannot be violated by cancellation of
an edge labelled with $\inv{a}$.

It holds therefore $|\alpha(\pi_n)| = u_n = v \in \genbyone{L_{\fa{D}}}$.
\end{proof}

\bigskip
\begin{lemma}
\[ \genbyone{L_{\fa{D}}} \subset L_{\fa{B}} \]
\end{lemma}

\begin{proof}
Let $G = (V, E)$ be the graph of $\fa{D}$, $v \in \genbyone{L_{\fa{D}}}$ and
$\pi \in \pathcat{G}(S, F)$ be a path with label $|\alpha(\pi) = v$ and for any
path prefix $\omega \prefix \pi$ it should hold $|\alpha(\omega)| \prefix
|\alpha(\pi)$.

We want to construct a computation $f$ which corresponds to the path $\pi$ in
the sense of the previous lemma.

We consider the sequence of subpaths
\[ 1_{(z_0, r_2)} \prefix \pi_1 \prefix \pi_2 \ldots \prefix \pi_n = \pi \]
with $\len{\pi_i} \leq \len{\pi_{i-1}} + 2$ and $Z(\pi_i) \in Z \times
\setof{r_2, l_2}$ for $i = 1, \ldots, n$ and for $\pi_i \prefix \omega \prefix
\pi_{i+1} \Rightarrow Z(\omega) \in Z \times \setof{r_1, l_1}$ if $\pi_i
\neq \omega \neq \pi_{i+1}$.

We run the automaton $\fa{B}$ on the input $v$ and show that there exists a
computation $f$ which corresponds to the path $\pi$.

\medskip
To the empty path $\pi_0 = 1_{(z_0, r2)}$ corresponds $z_0 v$.

Let $f_k$ be ending with $u_k z_k v_k$.

Here, we also have to consider 4 cases:

\begin{itemize}
  \item[Case 1:] $Z(\pi_k) = (z_k, r_2),\ \pi_{k+1} = \pi_k \cdot e$ with $Z(e)
  = (z_{k+1}, r_2)$ and $\alpha(e) = a$.
  
  It then holds $v_k = a \bar{v_k}$ and by construction of $\fa{D}$ it holds
  $(z_{k+1, r}) \in \delta(a, z_k)$, therefore $u_{k+1} z_{k+1} v_{k+1}$ is a
  configuration following $u_{k} z_{k} v_{k}$.
   
  \item[Case 2:] $Z(\pi_k) = (z_k, r_2),\ \pi_{k+1} = \pi_k \cdot e_1 \cdot
  e_2$.
  
  We then have $Z(e_1) = (z_{k+1}, l_1),\ Z(e_2) = (z_{k+1}, l_2)$ and
  $\alpha(e_1) = a$ and $\alpha(e_2) = \inv{b}$.
  
  Because of $|\alpha(\pi)| \in X^*$ it follows $b = a$ and we have
  \[ (z_{k+1}, l) \in \delta(a, z_k)\text{ and }u_{k+1} = u \inv{a},\ v_{k+1} =
  a v_k \]
  and $u_{k+1} z_{k+1} v_{k+1}$ is a configuration following $u_{k} z_{k} v_{k}$.
  
  \item[Case 3:] $Z(\pi_k) = (z_k, l_2),\ \pi_{k+1} = \pi_k \cdot e$.
  
  It then holds $Z(e) = (z_{k+1}, l_2)$ and $\alpha(e) \in \inv{X}$. Let
  $\alpha(e) = \inv{a}$. We have then $|\alpha(\pi_{k+1})| = |\alpha(\pi_k)
  \cdot \inv{a}|$. Because of $|\alpha(\pi| \in X^*,\ \pi \in \pathcat{G}(S, V)$
  and $|\alpha(\pi_k| = u_k \cdot First(v_k)$ it follows $First(v_k) = a$.
  
  If we have $u_k = \bar{u}_k b$, then for $u_{k+1} = \bar{u}_k$ and $v_{k+1} =
  b a v_k$ it holds: $u_{k+1} z_{k+1} v_{k+1}$ is a configuration following
  $u_{k} z_{k} v_{k}$.
  
  \item[Case 4:] $Z(\pi_k) = (z_k, l_2),\ \pi_{k+1} = \pi_k \cdot e_1 \cdot
  e_2$.
  
  It follows $Z(e_1) = (z_{k+1}, r_1)$ and $Z(e_2) = (z_{k+1}, r_2)$ and
  $\alpha(e_1) = \inv{a} \in \inv{X},\ \alpha(e_2) = b \in X$.
  
  Because of $\alpha(\pi) \in X^*,\ \pi \in \pathcat{G}(S, V)$, it follows
  $|\alpha(\pi_k)| = u_k a$ and because $u_k a \inv{a} b \prefix |\alpha(\pi)|$
  and $u_k a \prefix |\alpha(\pi_{k+1})|$ it follows $b = a$.
  
  Therefore $\alpha(\pi_{k+1}) = u_k a$. If we set $u_{k+1} = u_k a$ and
  $v_{k+1} = \inv{a} v_k\ (v_k = a \bar{v}_k)$, then by construction of $\fa{D}$
  $u_{k+1} z_{k+1} v_{k+1}$ is a configuration following $u_{k} z_{k} v_{k}$ and
  the path $\pi_{k+1}$ corresponds to the computation $f_k$.
\end{itemize}

Let $v \in \genbyone{L_{\fa{D}}}$ and $\pi_n \in \pathcat{G}(S, F)$ with label
$|\alpha(\pi_n)| = v$ and let for $\omega \prefix \pi_n$ hold $|\alpha(\omega)|
\prefix |\alpha(\pi_n)|$, then there exists a computation $f_n = (z_0 v \to
\ldots \to v z_n)$ with $z_0 \in S_{\fa{B}},\ z_n \in F_{\fa{B}}$ and therefore
$v \in L_{\fa{B}}$.
\end{proof}

Both lemmata are summarized in the following theorem:

\begin{theorem}
\[ L_{\fa{B}} = \genbyone{L_{\fa{D}}} \]
\end{theorem}

\bigskip
We will show now that for all finite two-way automata $\fa{B}$ it holds
\[ L_{\fa{B}} \in \reglang(X^*)\ (= \ratlang(X^*)) \]

Before doing that, we show a relationship between $\ratlang(X^*)$ and
$\ratlang{\hgroup{X}}$.

For the automaton $\fa{D}$ constructed above we set
\[ L_{\fa{D}} := \setof{|\alpha(\pi)| \mid \pi \in \pathcat{G}(S, F)} \]
where $G = (V, E)$ is the graph of $\fa{D}$.

In this way, $\ratlang(\hgroup{X})$ is embedded in a natural way into
$(\unioninv{X})^*$, see chapter II.7.

We set
\[ |\ratlang(\hgroup{X})| := \setof{|L| \mid L \in \ratlang(\hgroup{X}}) \]

Then the following theorem holds:

\begin{theorem}
\[ |L| \cap X^* \in \ratlang(X^*) \]
\end{theorem}

\begin{proof}
We only sketch the proof of this theorem, the reader may give the proof
formally.

Let $v_1, v_2$ be vertices of the graph $G = (V, E)$ of the automaton $\fa{D}$,
and let $\pi: v_1 \to v_1$ be a path with label $\alpha(\pi) \equiv 1
\pmod{\hgroup{X}}$.

In this case, a new edge $e: v_1 \edge{\epsilon} v_2$ will be added to the
graph. This will be done for all pairs $(v_1, v_2) \in V \times V$ which are
connected by such a path.

We get a new automaton $\fa{D}'$ with $|L_{\fa{D}}| = |L_{\fa{D}'}|$.

If one removes all edges $e$ from the graph of $\fa{D}'$ where the label is
$\alpha(e) \in \inv{X}$, then one gets a finite automaton $\fa{A}$ accepting the
language $|L| \cap X^*$.
\end{proof}

Remark: A similar result can be proved for the free group $F(X)$
(exercise).


































