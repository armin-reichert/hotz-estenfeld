\section{Regular sets in \pocymon{X}, \reglang(\pocymon{X})}

In the first section of this chapter we looked at $\reglang(X^*)$, the class of
regular languages over the free monoid $X^*$.

We replace now the free monoid by an important special case of a monoid, namely
the polycyclic monoid \pocymon{X} which is the syntactic monoid of the Dyck
language of the alphabet $\unioninv{X}$.

We represent the elements of \pocymon{X} by their reduced words (see chapter
1.3).

\begin{definition}
Let $|w|$ be the reduced word of $w \in (\unioninv{X})^*$ with respect to the
polycyclic monoid \pocymon{X} and let $L \subset (\unioninv{X})^*$ be a
language. Then
\begin{eqnarray*}
|L| &:=& \setof{|w| \mid w \in L,\ |w| \not\equiv 0}
\end{eqnarray*}
is the {\bf set of reduced words} of $L$.
\end{definition}

\bigskip
\begin{definition}
\begin{eqnarray*}
|\reglang|(\pocymon{X}) & := & \{ L \subset (X\cup X^{-1})^* \mid \mbox{there
exists }L' \in \reglang(\pocymon{X}) \\
& & \mbox{ with } L = \setof{|w| \mid [w] \in L'} \}
\end{eqnarray*}
\end{definition}

Our goal is to show the inclusion
\[ |\reglang|(\pocymon{X}) \subset \reglang((\unioninv{X})^*) \]

This inclusion does not tell that the word problem of \pocymon{X} is rational,
or in other words that $[w]$ is a rational set. But the structure of
$\reglang(\pocymon{X})$ is determined for the biggest part by $\reglang(
(\unioninv{X})^*)$.

Of course it is
\[ |\reglang|(\pocymon{X}) \neq \reglang((\unioninv{X})^*) \]

But we can completely characterize $|\reglang|(\pocymon{X})$ as a subset of $\reglang((X
\cup X^{-1})^*)$.

To do so, we first prove some lemmata.

\begin{lemma}
\[ L \subset \reglang((\unioninv{X})^*) \Rightarrow |L| \in \reglang((\unioninv{X})^*)
\]
\end{lemma}

\transrem{Because the following proof was hard to read in the original book,
some names have been changed and figures been added.}

\begin{proof}
Let $\fa{A} = (G, \unioninv{X}, S, F, \alpha)$ be a finite acceptor of $L$. We
construct in several steps an acceptor $\fa{A}'$ for $|L|$.

This construction will play an exceptional role in later chapters.

Without loss of generality we may assume that for each edge $e$ of our graph $G
= (V, E)$ it holds $\alpha(e) \in \unioninv{X} \cup \setof{\epsilon}$.

{\bf Step 1:} If there is a path $\pi \in \pathcat{G}(v, v')$ with
$|\alpha(\pi)| = \epsilon$, then we add a new edge $e: v \to v'$ to $E$ and set
$\alpha(e) = \epsilon$. Any $\epsilon$-loops are removed.

\begin{center}
\begin{tikzpicture}
	\begin{pgfonlayer}{nodelayer}
		\node [style=state] (0) at (-2, -0) {$v$};
		\node [style=state] (1) at (2, -0) {$v'$};
	\end{pgfonlayer}
	\begin{pgfonlayer}{edgelayer}
		\draw [style=path] (0) to node[auto]{$|\alpha(\pi)|=\epsilon$} (1);
		\draw [style=transition, bend left=45, looseness=1.25] (0) to node[auto]{$\epsilon$} (1);
	\end{pgfonlayer}
\end{tikzpicture}
\end{center}

We do this for each such pair $(v, v')\in V \times V$ and get a graph $G_1$ with
corresponding labelling $\alpha$.

{\bf Step 2:} For consecutive edges $e: v_1 \edge{\epsilon} v_2,\ e': v_2
\edge{x} v_3 \in E_1,\ x \in \unioninv{X}$, we add a ''bridge'' $b: v_1 \edge{x}
v_3$ with label $x$.

For each edge $e_{final}: v_3 \edge{\epsilon} f$, where $f \in F$ is a final
state, we add a ''bridge'' $b_{final}: v_1 \edge{x} f$ with label $x$.

\begin{center}
\begin{tikzpicture}
	\begin{pgfonlayer}{nodelayer}
		\node [style=state] (0) at (-2, -0) {$v_1$};
		\node [style=state] (1) at (0, -0) {$v_2$};
		\node [style=state] (2) at (2, -0) {$v_3$};
		\node [style=state] (3) at (4, -0) {$f$};
		\node [style=none] (4) at (1, -0) {$e'$};
		\node [style=none] (5) at (3, -0) {$e_{final}$};
		\node [style=none] (6) at (-1, -0) {$e$};
		\node [style=none] (7) at (1.5, -1) {$b$};
		\node [style=none] (8) at (2.75, -2) {$b_{final}$};
	\end{pgfonlayer}
	\begin{pgfonlayer}{edgelayer}
		\draw [style=transition, bend left, looseness=1.25] (0) to node[auto]{$\epsilon$} (1);
		\draw [style=transition, bend left, looseness=1.25] (1) to node[auto]{$x$} (2);
		\draw [style=transition, bend left, looseness=1.25] (2) to node[auto]{$\epsilon$} (3);
		\draw [style=transition, bend right=60, looseness=1.00] (0) to node[auto]{$x$} (2);
		\draw [style=transition, bend right=60, looseness=1.25] (0) to node[auto]{$x$} (3);
	\end{pgfonlayer}
\end{tikzpicture}
\end{center}

If we do this for all such pairs of edges $(e, e')$, we get a graph
$G_2=(V_2,E_2)$ with labelling $\alpha$ as described.

{\bf Step 3:} We remove all edges $e \in E_2$ where $\alpha(e) = \epsilon$. The
resulting graph shall be $G_3=(V_3,E_3)$ with labelling $\alpha$.

Because of our construction we get:
\[ L_{\fa{A}_3} = \setof{v \mid v \mbox{ is reduced word of } w \in
L_{\fa{A}} \mbox{ wrt. } x \cdot x^{-1} = 1} \]

{\bf Step 4:} We cut all paths whose label contains the word $a \cdot
d^{-1},\ a, d \in X,\ a \neq d$.

\begin{center}
\begin{tikzpicture}
	\begin{pgfonlayer}{nodelayer}
		\node [style=state] (0) at (0, -0) {$v$};
		\node [style=state] (1) at (-2, 1) {};
		\node [style=state] (2) at (2, 1) {};
		\node [style=state] (3) at (-2, -1) {};
		\node [style=state] (4) at (2, -1) {};
	\end{pgfonlayer}
	\begin{pgfonlayer}{edgelayer}
		\draw [style=transition] (1) to node[auto]{$a$} (0);
		\draw [style=transition] (0) to node[auto]{$b$} (2);
		\draw [style=transition] (0) to node[auto]{$d^{-1}$} (4);
		\draw [style=transition] (3) to node[auto]{$c^{-1}$} (0);
	\end{pgfonlayer}
\end{tikzpicture}
\end{center}

To do so, we duplicate vertices and edges of $G_3$. The graph structure
above will be transformed into this structure:

\begin{center}
\begin{tikzpicture}
	\begin{pgfonlayer}{nodelayer}
		\node [style=state] (0) at (0, 1) {$v_{+}$};
		\node [style=state] (1) at (0, -1) {$v$};
		\node [style=state] (2) at (-3, 1) {};
		\node [style=state] (3) at (-3, -1) {};
		\node [style=state] (4) at (3, 1) {};
		\node [style=state] (5) at (3, -1) {};
		\node [style=none] (6) at (2.75, -0) {};
		\node [style=none] (7) at (1.5, 1.75) {};
	\end{pgfonlayer}
	\begin{pgfonlayer}{edgelayer}
		\draw [style=transition] (2) to node[auto]{$a$} (0);
		\draw [style=transition] (0) to node[auto]{$b\ (e_{+})$} (4);
		\draw [style=transition] (3) to node[auto]{$c^{-1}$} (1);
		\draw [style=transition] (1) to node[auto]{$d^{-1}$} (5);
		\draw [style=transition] (1) to node[auto]{$b\ (e_{-})$} (4);
	\end{pgfonlayer}
\end{tikzpicture}
\end{center}

Let $v$ be a vertex which is the target of an edge with label $a \in X$ and the
source of an edge with label $d^{-1} \in X^{-1}$.

For each vertex $v$ we add a copy $v_{+}$ to the graph and add new edges as
follows:
\begin{itemize}
  \item All edges entering $v$ with label $a \in X$ are rerouted to the copy
  $v_{+}$.
  \item All edges entering $v$ with label $c^{-1} \in X^{-1}$ are left
  unchanged.
  \item All edges $e$ leaving $v$ with label $b \in X$ will be
  duplicated into edges $e_{+}$ and $e_{-}$. Edge $e_{+}$ leads from $v_{+}$
  to $Z(e)$ and the other edge $e_{-}$ from $v$ to $Z(e)$. The label for
  both new edges is $\alpha(e) = b$.
	\item All edges leaving $v$ with label $d^{-1} \in X^{-1}$ stay unchanged.
\end{itemize}
We do this with every such vertex of $G_3$ and get a graph $G'$.

With start and final state sets $S' := S$ and $F' := F \cup \setof{v^+ \mid v
\in F}$ we get a finite automaton
\[ \fa{A}' = (G', \unioninv{X}, S', F', \alpha') \]

It then holds: $L_{\fa{A}'} \stackrel{!}{=} |L|$

We observe that $L_{\fa{A}'}$ contains only reduced words which are different
from $0$. Additionally, the reversal of the construction of $G'$ from $G_3$ is a
functor which preserves the edge labelling.

Therefore $L_{\fa{A}'} \subset |L|$.

\bigskip
Let to the opposite $\pi$ be a path in $\pathcat{G_3}$ with label $\alpha(\pi)
= |\alpha(\pi)|$, then there is no $(+, -)$-change in the label of $\pi$.

Therefore the path $\pi$ stays unchanged in $G'$ as long as the label
of the edges is in $X^{-1}$.

If the label changes, then the path switches to the $+$-vertices and stays
''positive'' until its target vertex. Let $\pi' \in \pathcat{G'}$ denote this
path belonging to $\pi$, then $\alpha(\pi) = \alpha'(\pi')$.

Therefore it holds $|L| \subset L_{\fa{A}'}$.

Together we get $|L| = L_{\fa{A}'}$.
\end{proof}

\bigskip
In this proof we showed even more, we summarize this in the following corollary.

\begin{corollary}
To each path
\[ \pi = e_1 \cdots e_k \in \pathcat{G}(S, -),\ e_i \in E \]
with $\alpha(e_k) \neq \epsilon$ and $\alpha(\pi) \neq \epsilon$ there is
assigned a unique $\epsilon$-free path $\pi' \in \pathcat{G_3}(S, -)$ with $\alpha(\pi) =
\alpha(\pi')$.

Additionally, we can define a functor $\nu : \pathcat{G_3} \to \pathcat{G}$ such
that the following holds:

For each accepted word $w \in L$ there exists an accepting path
\[ \pi = e_1 \cdots e_n \in \pathcat{G_3}(S, F) \]
with $e_i \in E_3,\ \alpha(e_i) \neq \epsilon$ such that $\alpha(\nu(\pi)) = w$.
\end{corollary}

\begin{proof}
Exercise for the reader.
\end{proof}

Our construction shows even more: the number of paths $\pi$ with label
$\alpha(\pi) = w$ for a given $w \in |L|$ is the same in $\pathcat{G_3}$ as in
$\pathcat{G'}$. Therefore this number is in $\pathcat{G'}$ less or equal to
the number in $\pathcat{G_2}$. A similar statement holds for $\pathcat{G_1}$ and
$\pathcat{G}$.

This fact, which will be of importance later in this book, is summarized in the
following corollary. We define

\begin{definition}
\[ \mathrm{INDEX}(|\fa{A}|, w) := \card{\setof{\pi \in \pathcat{G}(S, F) \mid
|\alpha(\pi)| = w}} \]
\end{definition}

\bigskip
\begin{corollary}
\[ \mathrm{INDEX}(|\fa{A}|, w) \geq \mathrm{INDEX}(|\fa{A}'|, w),\ w \in
L_{\fa{A}'} \]
\end{corollary}

Remark: Because of $L_{\fa{A}'} = |L_{\fa{A}'}|$ it holds
\[ \mathrm{INDEX}(|\fa{A}'|, w) = \mathrm{INDEX}(\fa{A}', w) \]

Now we can prove the announced theorem.

\bigskip
\begin{theorem}
\[ |\reglang|(\pocymon{X}) = \setof{|L| \mid L \in \reglang((X\cup X^{-1})^*)} \]
\end{theorem}

\begin{proof}
 The inclusion ''$\supset$'' is clear.
 
 ''$\subset$'': Every functor $\alpha : \pathcat{G} \to \pocymon{X}$ can be
 extended to a functor $\alpha' : \pathcat{G} \to (\unioninv{X})^*$ by
 defining $\alpha'(e) := |w|$ for each edge $e \in E$ with label $\alpha(e) =
 [w]$.
 
 $\alpha'$ is continued to a functor and for the finite automaton
 \[ \fa{A}' = (G', \unioninv{X}, S', F', \alpha') \]
 it holds:
 \[ L_{\fa{A}} = \setof{[w] \mid w \in L_{\fa{A}'}} \]
 and with lemma 1 we get:
 \[ L_{\fa{A}} = \setof{[w] \mid w \in |L_{\fa{A}'}|} \]
 which means
 \[ |L_{\fa{A}}| = \setof{|w| \mid w \in L_{\fa{A}'}} \]
\end{proof}

The next theorem gives an even more precise characterization for
$|\reglang|(\pocymon{X})$.

\begin{theorem}
Let $\fa{A}$ be a finite automaton
\[ \fa{A} = (G, \unioninv{X}, S, F, \alpha) \]
where $G = (V, E)$ and $\card{E} = c \in \mathbb{N}$.

Then there exist $c$ pairs $(L_i^+, L_i^-)$ with 
\[ L_i^+ \in \reglang(X^*),\qquad L_i^- \in \reglang((X^{-1})^*) \]
such that
\[ |L_{\fa{A}}| = \bigcup_{i=1}^c L_i^- \cdot L_i^+ \]
\end{theorem}

\begin{proof}
The proof follows directly from the construction of the automaton $\fa{A}'$.

If $V'$ is the set of those vertices $v$ which have an associated vertex $v^+$
in $G'$, then every path $\pi \in \pathcat{G'}(S', F')$ can be uniquely split into
 a product $\pi_1 \cdot \pi_2$ where the target $Z(\pi_1)$ is such a vertex $v$.

Therefore one can write
\[ |L_{\fa{A}}| = \bigcup_{v \in V} \left( L_{\fa{A}(S, v)}^- 
\cdot L_{\fa{A}(v, F)}^+ \right)
\]
\end{proof}

\begin{corollary}
\[ \mathrm{INDEX}(\fa{A}', w) = \sum_{v \in V' \atop w_1 \cdot w_2 = w} \left(
\mathrm{INDEX}(\fa{A}'(S, v), w_1) \cdot \mathrm{INDEX}(\fa{A}'(v, F), w_2)
\right) \]
\end{corollary}

\bigskip
\begin{lemma}
Let $L = L_{\fa{A}}^+ \subset X^*$ and $L' = L_{\fa{A}}^- \subset (X^{-1})^*$.

Then there exist 
\[ L_1^+, \ldots, L_p^+ \in \reglang(X^*),\quad L_1^-, \ldots, L_q^- \in
\reglang((X^{-1})^*) \]
with
\[ |L \cdot L'| = \bigcup_{i=1}^{p} L_i^+ \ \cup\ \bigcup_{j=1}^{q} L_j^- \]
Here $p, q \leq \card{F(\fa{A'})}$.
\end{lemma}

\begin{proof}
We construct $\fa{A} = \fa{A}^+ \circ \fa{A}^-$ with $L_{\fa{A}} = L \cdot L'$ and
thereafter $\fa{A'}$ with $L_{\fa{A}'} = |L_{\fa{A}}|$.

Because for every reduced word $w \in (X\cup X^{-1})^*$ it holds $w = w^- \cdot
w^+$ with $w^- \in (X^{-1})^*,\ w^+ \in X^*$, the first part of the lemma is
proved.

From our construction of $\fa{A'}$ it follows that the final state set
$F(\fa{A'})$ is split into states for sets from $(X^{-1})^*$ and those
for sets from $X^*$. From that follows the rest of the lemma.
\end{proof}

\bigskip
In the following we want to investigate the set of regular sets which can be
defined with a fixed graph and prove two simple but important properties of
these sets.

Let $G = (V, E)$ be a graph and $\alpha : E \to (\unioninv{X})^*$. We may
assume again that $\alpha(e) \in \unioninv{X} \cup \setof{\epsilon}$.

We define
\[ \mathcal{R}(G) := \setof{L_{\fa{A}(\tilde{S},\tilde{F})} \mid \fa{A} = (G, X
\cup X^{-1}, S, F, \alpha),\ \tilde{S} \subset V,\ \tilde{F} \subset V} \]

For $c := \card{V}$ it trivially holds 
\[ \card{\mathcal{R}(G)} \leq 2^{2 \cdot c} \]

Remark: The set $\mathcal{R}(G)$ is in general not closed under union but it
contains a simple base for the union-closure ${<}\mathcal{R}(G){>}$.

For shorter notation we set $\mathcal{R}(\tilde{S},\tilde{F}) :=
L_{\fa{A}(\tilde{S},\tilde{F})}$.

\begin{lemma}
\[ \mathcal{R}(\tilde{S},\tilde{F}) = \bigcup_{\substack{v \in \tilde{S}\\v'
\in \tilde{F}}} \mathcal{R}(v, v') \]
\end{lemma}

\begin{proof}
Exercise for the reader.
\end{proof}

\bigskip
Let $\fa{A'}$ be the automaton constructed as in lemma 1. We construct the graph
$G' = (V', E')$ from the graph $G = (V, E)$. Define
\[ \mathcal{B}(G') := \setof{L_{\fa{A'}(v, v')} \mid v, v' \in V'} \]

\begin{lemma}
Let $L^+ \subset X^*, L^- \subset (X^{-1})^*$ and $L^+, L^- \in
\mathcal{B}(G')$. Then $|L^+ \cdot L^-|$ is contained in the union-closure
${<}\mathcal{B}(G'){>}$ of $\mathcal{B}(G')$.
\end{lemma}

\begin{proof}
Let $L^+ = L_{\fa{A'}(v, v')}$ and $L^- = L_{\fa{A'}(q, q')}$. 

We create an automaton
\[ \fa{A}'' := \fa{A'}(v, v') \circ \fa{A'}(q, q') \]
with $L_{\fa{A}''} = L^+ \cdot L^-$.

We get $\fa{A}''$ by connecting an instance of $\fa{A'}(q, q')$ using an
$\epsilon$-edge which leads from $v'$ to $q$. Now all words $w \in L_{\fa{A}''}$
have the form $w = w^+ \cdot w^-$. 

But we are only interested in reduced words.

There are three cases:
\[ |w^+ \cdot w^-| = \begin{cases} w_1^+ \\ 0 \\ w_1^- \end{cases} \]

Because every path $\pi$ of $\fa{A}''$ leading from $v$ to $q'$ is divided into
two sections
\[ \pi = \pi^+ \circ e \circ \pi^- \text{ where }\pi^+: v \to v' \in \fa{A'}(v,
v')\text{ and }\pi^-: q \to q' \in \fa{A'}(q, q') \]
there exists in case $|w^+ \cdot w^-| = w_1^+$ a decomposition $\pi^+ = \pi_1^+
\cdot \pi_2^+$ with labels $\alpha(\pi_1^+) = w_1^+$ and $|\alpha(\pi_2^+ \circ
e \circ \pi^-)| = \epsilon$.

TODO: add figure

\bigskip
Opposite direction:

If $\bar{\pi}_1$ is a path with $Q(\bar{\pi}_1) = Q(\pi_1^+),\ Z(\bar{\pi}_1) =
Z(\pi_1^+)$, then $\alpha(\bar{\pi}_1) \in |L^+ \cdot L^-|$.

\medskip
The case for $|w^+ \cdot w^-| = w_1^-$ is symmetrical.

\bigskip
Therefore there exist sets $V^+,\ V^- \subset V'$ with
\[ |L^+ \cdot L^-| = \bigcup_{v^+ \in V^+} \mathcal{R}(v, v^+) \ \cup\ 
\bigcup_{q^- \in V^-} \mathcal{R}(q^-, q') \]
\end{proof}

Remark: These facts will be useful in connection to the normal-form
theorems for context-free grammars (chapter IV). This is especially true for the
following

\begin{corollary}
If 
\[ \mathcal{B}(G', S, F) = \setof{L^+_{\fa{A'}(S, v)},\ L^-_{\fa{A'}(q, F)}
\mid v, q \in V} \]
Then $\mathcal{B}(G', S, F)$ is a union-base for
\[ \setof{|L^+ \cdot L^-| \mid L^+, L^- \in \mathcal{B}(G', S, F)} \]
\end{corollary}

This corollary gives us a smaller base. For fixed $S$ and $F$ we can represent
every set $|L^+ \cdot L^-|$ as the union of at most $2 \cdot \card{V}$ sets of
type $L^+_{\fa{A}(S, v)}$ resp.\ $L^-_{\fa{A}(q, F)}$.

We assign now to $\mathcal{B}(G)$ a system 
\begin{eqnarray*} 
\mathcal{P}(G) & \subset & \mathcal{B}(G) \times \mathcal{B}(G) \cdot \setof{;}
\cdot \mathcal{B}(G) \\
& \cup & \setof{\epsilon} \times \mathcal{B}(G)
\end{eqnarray*}
in the following way:
\begin{eqnarray*}
L \to L^+ ; L^- \in \mathcal{P}(G) & \iff & \text{in the representation } L^+
  \cdot L^- = \cup L_i \text{ from lemma 2} \\
  & & \text {there exists an index $i$ with $L = L_i$} \\
\epsilon \to L \in \mathcal{P}(G) & \iff & \epsilon \in L
\end{eqnarray*}

\bigskip
\begin{lemma}
$\epsilon \in |L_1 L_2 \cdots L_n|$ with $L_i \in \mathcal{B}(G) \iff
L_1;L_2;\cdots ;L_n$ can be reduced to $\epsilon$ using the rules from
$\mathcal{P}(G)$.
\end{lemma}

\begin{proof}

''$\Rightarrow$'':
\begin{eqnarray*}
\epsilon \in |L_1 L_2 \cdots L_n| & \Rightarrow & \text{ there exists } w_i \in
L_i,\ i = 1, \ldots, n,\text{ with }|w_1 \cdots w_n| = \epsilon \\
& \Rightarrow & \text{there exists an index $j$ with }L_j \subset X^*,\ L_{j+1}
\subset (X^{-1})^*
\end{eqnarray*}
Because $|w_1 \cdots w_n| = \epsilon$ it holds $|w_j \cdot w_{j+1}| \neq 0$.

Therefore $w_j \cdot w_{j+1}$ can be reduced to a word $\bar{w}_j$ with
$\bar{w}_j \in X^*$ or $\bar{w}_j \in (X^{-1})^*$.

In the decomposition $L_j \cdot L_{j+1} = \cup L'_i$ there exists an $L'_j$ with
$\bar{w}_j \in L'_j$ and it holds $L'_j \to L_j;L_{j+1} \in \mathcal{P}(G)$.

If we replace $L_j;L_{j+1}$ by $L'_j$ then we have again
\[ \epsilon \in |L_1 \cdots L_{j-1} \cdot L'_j \cdot L_{j+2} \cdots L_n| \]

Inductively we may assume that $L_1 \cdots L_n$ can be reduced using
$\mathcal{P}(G)$ to some $L'$ and that $\epsilon \in |L'|$ holds.

Because $|L'| = L$ it follows $\epsilon \in L$. Therefore $L_1; \cdots ;L_n$ can
indeed be reduced to $\epsilon$.

''$\Leftarrow$'':

Let $L_1; \cdots ;L_n$ be reducible to $\epsilon$ using $\mathcal{P}(G)$ and let
$L'_j \to L_j;L_{j+1} \in \mathcal{P}(G)$.

Then for each $w_i \in L_i,\ i = 1, \ldots, j-1, j+2, \ldots, n$ and $\bar{w}_i
\in L_j'$ there exist $w_j \in L_j,\ w_{j+1} \in L_{j+1}$ with
\[ |w_1 \cdots w_n| = |w_1 \cdots w_{j-1} \cdot \bar{w}_j \cdot w_{j+2} \cdots
w_n|
\]
If $\epsilon \in L_j$ then $\epsilon \to L_j \in \mathcal{P}(G)$ and it holds
$\bar{w}_j = \epsilon$. Therefore
\[ |w_1 \cdots w_{j-1} \cdot w_{j+1} \cdots w_n| = |w_1 \cdots w_n| \]

By induction it follows: if $L_1; \cdots ;L_n$ can be derived in
$\mathcal{P}(G)$ from $\epsilon$, then $\epsilon \in |L_1 \cdots L_n|$.
\end{proof}

