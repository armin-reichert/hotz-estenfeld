\section{The deterministic finite automaton with storage}

In this section we will prove a very important theorem about the relation
between non-determinism and determinism of finite automata with storage.

To be able to construct the deterministic automaton we need some definitions.

\begin{definition}
A finite automaton with storage $\fa{P}=(\fa{A},\store{P})$ is called
\begin{itemize}
  \item {\bf $\epsilon$-free} if $\fa{A}$ is $\epsilon$-free,
  \item {\bf deterministic} if $\fa{A}$ is deterministic,
  \item {\bf complete} if $\fa{A}$ is complete.
\end{itemize}
\end{definition}

\bigskip
\begin{definition}
$(S, +, \cdot, 0, 1)$ is called a {\bf semi-ring} if
\begin{enumerate}
  \item $(S, +, 0)$ is a commutative monoid
  \item $(S, \cdot, 1)$ is a monoid
  \item The distributive laws hold:
  \begin{eqnarray*}
  a \cdot (b + c) &=& a \cdot b + a \cdot c\\
  (a + b) \cdot c &=& a \cdot c + b \cdot c\\
  \end{eqnarray*}
  \item $0 \cdot a = a \cdot 0 = 0$ for all $a \in S$
\end{enumerate}
\end{definition}

\bigskip
\begin{definition}
Let $M$ be a monoid and $S$ a semi-ring.
\[ S(M) := \setof{\lambda: M \to S \mid \card{\setof{m \mid \lambda(m) \neq 0}}
< \infty} \]
is called the {\bf monoid-ring} of the monoid $M$ over the semi-ring $S$.
\end{definition}

Addition and multiplication on $S(M)$ are defined as follows: For $\lambda_1,
\lambda_2 \in S(M)$
\begin{eqnarray*}
(\lambda_1 + \lambda_2)(m) &=& \lambda_1(m) + \lambda_2(m) \\
(\lambda_1 \cdot \lambda_2)(m) &=& \sum_{m_1 \cdot m_2 = m} 
\lambda_1(m) \cdot \lambda_2(m)
\end{eqnarray*}

One can see that $(S(M), +, \cdot)$ again is a semi-ring. We also write
\[ \lambda := \sum_{\substack{m \in M\\\lambda(m)\neq 0}} m \cdot \lambda(m) \]
or also
\[ \sum_{\substack{m \in M\\\lambda(m)\neq 0}} m \cdot {<}\lambda, m{>} \]

The operations are defined in a way that one can calculate with these
expressions as usual.

In the following we will most often consider the semi-ring $S = \mathbb{B}$, the
{\bf Boolean semi-ring}.

\bigskip
Before proving our main theorem we will show some other results on finite
automata with storage.

\bigskip
First we want to give a representation for graphs in the polycyclic monoid
$\pocymon{Y}$.

Let $G=(V, E)$ be a finite graph with $\card{E} = n$, let $\card{Y} = 2$ and
$2^k \geq n$.

Let $\beta: V \to Y^*$ be an injective mapping and let $\len{\beta(v)} = k$
for each vertex $v \in V$.

Further let $G' = (\setof{v_0}, E')$ be a single-vertex graph and $\phi: E \to
E'$ a bijection on the edges, and let $Q(e') = Z(e') = v_0$ for each edge $e'
\in E'$.

The mapping $\phi$ can be uniquely continued to a functor $\phi : \pathcat{G}
\to \pathcat{G'}$. As a functor, $\phi$ is also injective.

We complement $G'$ by a storage $\gamma: E' \to \pocymon{Y}$ by setting
\[ \gamma(e') := u^{-1} \cdot v\text{ where }u = \beta(Q(\phi^{-1}(e')))\text{
and }v = \beta(Z(\phi^{-1}(e'))) \]

In this way the storage ''characterizes'' the paths $\phi(\pathcat{G})$.

Obviously it then holds
\begin{lemma}
\begin{eqnarray*}
\pi \in \phi(\pathcat{G}) &\Leftrightarrow& \gamma(\pi) \neq 0 \\
\gamma(\pi) \neq 0 &\Rightarrow& \gamma(\pi) = u^{-1} \cdot v \\
&& \text{with }u = \beta(Q(\phi^{-1}(e'))) \\
&& \text{and }v = \beta(Z(\phi^{-1}(e'))) \\
&& \text{and }\pi = e'_1 \circ \pi' \circ e'_k
\end{eqnarray*}
\end{lemma}

\begin{proof}
''$\Rightarrow$'':

Let $\pi \in \pathcat{G}$ be a path, $\pi = e'_1 \ldots e'_k \Rightarrow Z(e'_i)
= Q(e'_{i+1})$.

There exists a path $\psi \in \pathcat{G}$ such that $\psi = e_1 \ldots e_k$
with $\phi(\psi) = \pi$.
\begin{eqnarray*}
&\Rightarrow& \beta(Z(\phi^{-1}(e'_i))) = \beta(Q(\phi^{-1}(e'_{i+1}))) \\
&\Rightarrow& \beta(Z(\phi^{-1}(e'_i)))^{-1} \cdot \beta(Q(\phi^{-1}(e'_{i+1})))
= 1 \\
&\Rightarrow& \gamma(e'_1 \ldots e'_k) = \beta(Q(\phi^{-1}(e'_1)))^{-1} \cdot
\beta(Z(\phi^{-1}(e'_k)))
\end{eqnarray*}

''$\Leftarrow$'':

Let $\pi \notin \phi(\pathcat{G}),\ \pi = e'_1 \ldots e'_k$ and there exists an
index $i$ such that $Z(\phi^{-1}(e'_i)) \neq Q(\phi^{-1}(e'_{i+1}))$
\begin{eqnarray*}
&\Rightarrow& \beta(Z(\phi^{-1}(e'_i))) \neq \beta(Q(\phi^{-1}(s'_{i+1}))) \\
&\Rightarrow& \beta(Z(\phi^{-1}(s'_i)))^{-1} \cdot \beta(Q(\phi^{-1}(e'_{i+1})))
= 0
\end{eqnarray*}
\end{proof}

We want to clarify the construction using an example. Let $G = (V, E)$ be the
following graph:

\begin{center}
\begin{tikzpicture}
	\begin{pgfonlayer}{nodelayer}
		\node [style=state] (0) at (-1.5, -0) {$1$};
		\node [style=state] (1) at (1.5, -0) {$3$};
		\node [style=state] (2) at (0, 1.5) {$4$};
		\node [style=state] (3) at (0, -1.5) {$2$};
	\end{pgfonlayer}
	\begin{pgfonlayer}{edgelayer}
		\draw [style=transition] (0) to node[auto]{$3$} (1);
		\draw [style=transition] (1) to node[auto]{$4$} (2);
		\draw [style=transition] (0) to node[auto]{$5$} (2);
		\draw [style=transition] (0) to node[auto]{$2$} (3);
		\draw [style=transition] (3) to node[auto]{$1$} (1);
		\draw [style=transition, bend left=75, looseness=2.50] (3) to node[auto]{$6$} (2);
	\end{pgfonlayer}
\end{tikzpicture}
\end{center}

It is $\card{V} = 4 \Rightarrow k = 2\ (= \log_2(4))$.

The mapping $\beta: V \to Y^*,\ Y = \setof{(, [}$ shall be defined by
\[ \beta(1) = [[,\quad \beta(2) = [(,\quad \beta(3) = ([,\quad \beta(4) = (( \]

Let $\phi : E \to E'$ be a bijection and $\gamma: E' \to \pocymon{Y}$ with for
example $\gamma(1) = )]([$.

As a consequence to our construction we get

\begin{theorem}
For each regular language $L \in \reglang(X^*)$ there exists a homomorphism $h:
X^* \to \mathbb{B}(\pocymon{Y})$ with
\[ L = h^{-1}(u^{-1} \cdot v + \sum),\quad \sum \text{ finite sum in
}\mathbb{B}(\pocymon{Y}) \]
$h(x),\ x\in X$ contains only monomials of length $2 \cdot \log_2(n)$ where $n
= \card{V}$.
\end{theorem}

\begin{proof}
$L \in \reglang(X^*)$, then there exists a deterministic finite automaton
\[ \fa{A} = (G, X, S, F, \alpha) \]
with $\card{S} = \card{F} = 1$ which accepts $L$.

To $G = (V, E)$ we construct the single-vertex graph $G' = (V', E')$ and the
mapping $\phi:
E \to E'$ as above. The labelling $\alpha$ is carried over to $E'$ by the
bijection $\phi$. $\gamma$ shall also be defined as above.

We merge all edges with the same label into a single edge $e'$ an define
\[ \gamma'(e') := \sum_{s} \gamma(s) \quad\text{the edges $e$ have the same
label and are merged to $s'$} \]
and continue $\gamma'$ to a homomorphism.

We define our homomorphism $h$ as
\[ h: X^* \to \mathbb{B}(\pocymon{Y})\text{ defined by }h(x) := \gamma'(e'),\
\alpha'(e') = x \]

Then it holds: $w \in L_{\fa{A}} \Leftrightarrow h(w) = u^{-1} \cdot v + \sum$
with $u = \beta(S)$ and $v = \beta(F)$ (exercise) and the length-condition is
also fulfilled.
\end{proof}

\bigskip
We want to use now our graph embedding construction to transform special finite
automata with storage and to approach our goal, the deterministic finite
automaton with storage.

\begin{theorem}
Let $\fa{P} = (\fa{A}, \store{P})$ be an $\epsilon$-free automaton with
storage monoid $\pocymon{Y}$, let the domain and co-domain be given by $D = C
= \setof{\epsilon}$.

Then there exists a deterministic finite automaton $\fa{P}' = (\fa{A}', \store{P}')$
with storage $\mathbb{B}(\pocymon{Y})$ (if $\card{Y} \geq 2$) and with
\[ D' = \setof{\epsilon},\qquad C' = \setof{\beta(s)^{-1} \cdot \beta(f) +
\mathbb{B}(\pocymon{Y}) \mid s\in S,\ f\in F} \]
such that $L_{\fa{P}} = L_{\fa{P}'}$.

If $\card{Y} = 1$ we add a second element to $Y$.
\end{theorem}

\begin{proof}
In the first step we make $\fa{A} = (G, X, S, F, \alpha)$ deterministic as we
have seen in chapter II.1. We also can assume without loss of generality that
$\alpha(e) \in X$ and $\card{S} = 1$ (exercise).

Let now
\[ \fa{A}' = (G', X, S', F', \alpha') \]
be the deterministic finite automaton where
\[ G'=(V', E'),\quad V' = \powset{V} \]
\end{proof}








































