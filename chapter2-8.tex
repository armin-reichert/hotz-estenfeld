\section{The finite 2-way automaton}

We start now with generalizing the concept of the finite automaton as described
in chapter II.4.

We consider a finite automaton where the reading head can traverse the input
tape in two directions as indicated by the following figure:

\begin{center}
\begin{tikzpicture}
	\begin{pgfonlayer}{nodelayer}
		\node [style=none] (0) at (-1, -0.5) {};
		\node [style=none] (1) at (0, -0.5) {};
		\node [style=none] (2) at (-3, -0) {};
		\node [style=none] (3) at (-3, -0.5) {};
		\node [style=none] (4) at (2, -0) {};
		\node [style=none] (5) at (2, -0.5) {};
		\node [style=none] (6) at (-1, -0.5) {};
		\node [style=none] (7) at (0, -0.5) {};
		\node [style=none] (8) at (-1.5, -1.75) {};
		\node [style=none] (9) at (0.5, -1.75) {};
		\node [style=none] (10) at (-1.5, -3.25) {};
		\node [style=none] (11) at (0.5, -3.25) {};
		\node [style=none] (12) at (-0.5, -1) {};
		\node [style=none] (13) at (-0.5, -1.75) {};
		\node [style=none] (14) at (-0.5, -2.5) {$v$};
		\node [style=none] (15) at (3, -0.25) {Input tape};
		\node [style=none] (16) at (3, -1) {Two-way reading head};
		\node [style=none] (17) at (3, -2.5) {Finite state control in state $v$};
		\node [style=none] (18) at (-1.5, -1) {};
		\node [style=none] (19) at (0.5, -1) {};
		\node [style=none] (20) at (-2, -1) {};
		\node [style=none] (21) at (-2, -0) {};
		\node [style=none] (22) at (-2, -0.5) {};
		\node [style=none] (23) at (-1, -0) {};
		\node [style=none] (24) at (0, -0) {};
		\node [style=none] (25) at (1, -0) {};
		\node [style=none] (26) at (1, -0.5) {};
	\end{pgfonlayer}
	\begin{pgfonlayer}{edgelayer}
		\draw [style=simple] (3.center) to (0.center);
		\draw [style=simple] (0.center) to (1.center);
		\draw [style=simple] (1.center) to (5.center);
		\draw [style=simple, bend right=90, looseness=1.50] (6.center) to (7.center);
		\draw [style=simple] (6.center) to (7.center);
		\draw [style=simple] (8.center) to (9.center);
		\draw [style=simple] (9.center) to (11.center);
		\draw [style=simple] (11.center) to (10.center);
		\draw [style=simple] (10.center) to (8.center);
		\draw [style=simple] (12.center) to (13.center);
		\draw [style=transition] (12.center) to (18.center);
		\draw [style=transition] (12.center) to (19.center);
		\draw [style=simple] (2.center) to (4.center);
		\draw [style=simple] (21.center) to (22.center);
		\draw [style=simple] (23.center) to (0.center);
		\draw [style=simple] (24.center) to (1.center);
		\draw [style=simple] (25.center) to (26.center);
		\draw [style=simple] (4.center) to (5.center);
		\draw [style=simple] (2.center) to (3.center);
	\end{pgfonlayer}
\end{tikzpicture}
\end{center}

The representation of such an automaton by a state graph is not trivial. For
this reason, we first define this type of automaton as usual.

Afterwards, we will show how to represent this automaton by a state graph.

(The definition in the original book has been slightly adapted to conform to
the common literature.)

\begin{definition}
A {\bf (non-deterministic) two-way finite automaton} is defined by
\[ \fa{B} = (X, Q, \delta, S, F, \setof{L, R}) \]
where 
\begin{itemize}
  \item $X$ is the input alphabet with $\setof{L, R} \cap X = \emptyset$
  \item $Q$ is the state set with $\setof{L, R} \cap Q = \emptyset$
  \item $S \subset Q$ is the set of start states
  \item $F \subset Q$ is the set of final states of $\fa{B}$
  \item $\delta \subset (Q \times X) \to (Q \times \setof{L, R})$ is a
  relation describing the functioning of $\fa{B}$
\end{itemize}

If $\delta : (Q \times X) \to Q \times \setof{L, R}$ is a map, then
$\fa{B}$ is called a {\bf complete, deterministic two-way finite automaton}.
\end{definition}

\bigskip
\begin{definition}
A {\bf computation} of $\fa{B}$ on the input word $v \in X^*$ is a sequence of
single computation steps
\[ f_k = (z_0 v \to u_1 z_1 v_1 \to u_2 z_2 v_2 \to \ldots \to u_k z_k v_k) \]
with $u_i, v_i \in X^*,\ z_i \in Z$ for $i = 1, \ldots, k$ and $z_0 \in S$. 
\end{definition}

A single computation step $u_i z_i v_i \to u_{i+1} z_{i+1} v_{i+1}$ is defined
by the following alternatives:

\begin{itemize}
  \item If $v_i = a v'_i$, then $u_{i+1} = u_i a,\ v_{i+1} =
  v'_i$, if $(z_{i+1}, R) \in \delta(z_i, a)$,\\
  i.e.\ the reading head moves to the right and the automaton changes into state
  $z_{i+1}$, if it reads $a$ in state $z_i$.

	\item If $u_i = u'_i b,\ v_i = a v'_i$, then $u_{i+1} = u'_i$ and
	$v_{i+1} = b v_i$, if $(z_{i+1}, L) \in \delta(z_i, a)$,\\
	i.e.\ the reading head moves to the left and the automaton changes into state
	$z_{i+1}$, if it reads $a$ in state $z_i$.
\end{itemize}

\bigskip
\begin{definition}
A word $w \in X^*$ is {\bf accepted by} $\fa{B}$ if there exists a computation
$f_k$ with $v_k = \epsilon$ and $z_k \in F$.
\end{definition}

The language of $\fa{B}$ is
\[ L_{\fa{B}} = \setof{w \in X^* \mid w\text{ is accepted by }\fa{B}} \]

\bigskip
We want to consider again the monoid $\hgroup{X}$ (H-group, involutive monoid),
see chapter I.3, and construct an automaton 
\[ \fa{D} = (G, \hgroup{X}, S, F, \alpha) \]
with graph $G=(V, E)$ as follows:
\begin{itemize}
  \item $V = Q \times \setof{L_1, L_2, R_1, R_2}$
  
  \item Edge set $E$ and labelling $\alpha: E \to \hgroup{X}$ are defined as
  follows:
  
  $E$ contains an edge $e: (q, y) \edge{a} (q', y') \iff$ one of the following
  cases occurs:
  \begin{enumerate}
    \item $y = R_2,\ y' = R_2,\ (q', R) \in \delta(q, a)$
    \item $y = R_2,\ y' = L_1,\ (q', L) \in \delta(q, a)$
    \item $y = R_1,\ y' = R_2,\ q = q'$ and there exists $\bar{q} \in Q$ with
    $(q, R) \in \delta(\bar{q}, a)$
  \end{enumerate}

  $E$ contains an edge $e: (q, y) \edge{\inv{a}} (q', y') \iff$ one of the
  following cases occurs:
  \begin{enumerate}
    \item $y = L_2,\ y' = L_2,\ (q', L) \in \delta(q, a)$
    \item $y = L_2,\ y' = R_1,\ (q', R) \in \delta(q, a)$
    \item $y = L_1,\ y' = L_2,\ q = q'$ and there exists $\bar{q} \in Q$ with
    $(q, L) \in \delta(\bar{q}, a)$
  \end{enumerate}
\end{itemize}

\bigskip
To clarify this definition, we imagine the following model:

The input tape of the automaton is alternatingly divided into rectangular and
oval cells.

\begin{center}
\begin{tikzpicture}
	\begin{pgfonlayer}{nodelayer}
		\node [style=box] (0) at (0, -0) {$a$};
		\node [style=state] (1) at (-1, -0) {};
		\node [style=state] (2) at (1, -0) {};
		\node [style=state] (3) at (3, -0) {};
		\node [style=box] (4) at (4, -0) {$$};
		\node [style=state] (5) at (5, -0) {};
		\node [style=state] (6) at (-5, -0) {};
		\node [style=box] (7) at (-4, -0) {$$};
		\node [style=state] (8) at (-3, -0) {};
		\node [style=none] (9) at (-2, -0) {$\ldots$};
		\node [style=none] (10) at (2, -0) {$\ldots$};
		\node [style=none] (11) at (-4, 0.5) {$1$};
		\node [style=none] (12) at (0, 0.5) {$k$};
		\node [style=none] (13) at (4, 0.5) {$n$};
		\node [style=none] (14) at (-0.5, -1.5) {};
		\node [style=none] (15) at (0.5, -1.5) {};
		\node [style=none] (16) at (-0.5, -2.5) {};
		\node [style=none] (17) at (0.5, -2.5) {};
		\node [style=none] (18) at (0, -2) {$\fa{D}$};
		\node [style={triangle_down}] (19) at (1, -0.75) {};
		\node [style=none] (20) at (0, -1.5) {};
		\node [style=none] (21) at (1, -1) {};
	\end{pgfonlayer}
	\begin{pgfonlayer}{edgelayer}
		\draw [style=simple] (14.center) to (15.center);
		\draw [style=simple] (15.center) to (17.center);
		\draw [style=simple] (17.center) to (16.center);
		\draw [style=simple] (16.center) to (14.center);
		\draw [style=simple] (21.center) to (20.center);
		\draw [style=simple] (6) to (7);
		\draw [style=simple] (7) to (8);
		\draw [style=simple] (1) to (0);
		\draw [style=simple] (0) to (2);
		\draw [style=simple] (3) to (4);
		\draw [style=simple] (4) to (5);
	\end{pgfonlayer}
\end{tikzpicture}
\end{center}

The ovals represent ''idle-positions'' for the reading head, the rectangles
contain the single symbols of the input word.

The reading head moves from an ''idle-position'' to the neighbor
''idle-position'' to the left or to the right.

If it moves from left to right over the cell $k$ with symbol $a$, then the
automaton reads $a$, if it moves from right to left over that cell, it reads
$\inv{a}$. (This kind of reading when traversing a cell corresponds
to the physical model of a magnetic disc).

In some cases, our model has to perform two computation steps when the
automaton $\fa{B}$ performs only a single step.

The following cases may occur:

\begin{enumerate}
  \item The automaton is in idle position to the left of a cell $k$. It
  remembers that it should move to the right, i.e.\ it is in some state $(q,
  R_2)$. It traverses now the $k$th cell, reads the symbol $a$, reaches the
  subsequent idle position and finds that it should have gone to the left. This
  is desribed by the state $(q', L_1)$. Therefore it moves back to its previous
  idle position and changes into state $(q', L_2)$. Together it has read $a
  \inv{a}$.
  
  This kind of movement can be expressed using the following path:
  \[ (q, R_2) \edge{a} (q', L_1) \edge{\inv{a}} (q', L_2) \]
  
  \item In the same way, the following path is to be understood:
  \[ (q, L_2) \edge{\inv{a}} (q', R_1) \edge{a} (q', R_2) \]
\end{enumerate}

The definition of $\fa{D}$ is completed by setting
\[ S = S_{\fa{B}} \times \setof{R_2},\quad F = F_{\fa{B}} \times \setof{R_2} \]

We write again $|u| \in (\unioninv{X})^*$ for the reduced word for an element
$[u] \in \hgroup{X}$ and define:

\begin{definition}
\begin{eqnarray*}
\genbyone{L_{\fa{D}}} &:=& \{ |\alpha(\pi)| \mid  |\alpha(\pi)| \in
X^*,\ \pi \in \pathcat{G}(S, F) \\
&& \text{and for all paths prefixes }\omega \prefix \pi \text{ holds: }
|\alpha(\omega)| \prefix |\alpha(\pi)| \}
\end{eqnarray*}

Here, the symbol $\prefix$ denotes the path-prefix relation
\[ \prefix\ \subset \pathcat{G} \times \pathcat{G} \]
as well as the word-prefix relation
\[ \prefix\ \subset (\unioninv{X})^* \times (\unioninv{X})^* \]
\end{definition}

First, we want to give an example to clarify the construction given above.

The following two-way finite automaton accepts the language
\begin{eqnarray*}
L &=& \{ w \in \setof{a, b}^+ \cdot c \cdot \setof{a, b}^+ \mid w = w_1 \cdots
w_n,\ w_i \in X \\
&& \text{ with } w_i = c \Rightarrow w_{i-1} = w_{i+1},\ 1 < i < n \}
\end{eqnarray*} 

The transition function shall be defined as follows:
\begin{eqnarray*}
\delta(q_0, a) &=& \delta(q_1, a) = (q_1, R) \\
\delta(q_0, b) &=& \delta(q_1, b) = (q_1, R) \\
\delta(q_1, c) &=& (q_2, L) \\
\delta(q_2, a) &=& (q_a, R) \\
\delta(q_2, b) &=& (q_b, R) \\
\delta(q_a, c) &=& (q_{a'}, R) \\
\delta(q_b, c) &=& (q_{b'}, R) \\
\delta(q_{a'}, a) &=& (q_e, R) \\
\delta(q_{b'}, b) &=& (q_e, R) \\
\delta(q_{e}, a) &=& \delta(q_e, b) = q_e, R)
\end{eqnarray*}

The two-way finite automaton $\fa{A}$ is defined by
\[ \fa{A} = (\setof{q_0, q_1, q_2, q_a, q_b, q_{a'}, q_{b'}, q_e}, \setof{a,
b, c}, \setof{q_0}, \setof{q_e}, \delta) \]

From this we want to construct our automaton 
\[ \fa{D} = (G, \hgroup{\setof{a, b, c}}, \setof{(q_0, R_2)}, \setof{(q_e,
R_2)}, \alpha) \]

The graph $G = (V, E)$ has the following form (the markings at the edges are
the edge labels): 

\begin{center}
\begin{tikzpicture}
	\begin{pgfonlayer}{nodelayer}
		\node [style={oval_state}] (0) at (0, 6) {$(q_0, R_2)$};
		\node [style={oval_state}] (1) at (0, 4) {$(q_1, R_2)$};
		\node [style={oval_state}] (2) at (0, 2) {$(q_2, L_1)$};
		\node [style={oval_state}] (3) at (0, -0) {$(q_2, L_2)$};
		\node [style={oval_state}] (4) at (2, 1) {$(q_a, R_1)$};
		\node [style={oval_state}] (5) at (2, -1) {$(q_b, R_1)$};
		\node [style={oval_state}] (6) at (4, 2) {$(q_a, R_2)$};
		\node [style={oval_state}] (7) at (6, 3) {$(q_{a'}, R_2)$};
		\node [style={oval_state}] (8) at (6, -0) {$(q_e, R_2)$};
		\node [style={oval_state}] (9) at (4, -2) {$(q_b, R_2)$};
		\node [style={oval_state}] (10) at (6, -3) {$(q_{b'}, R_2)$};
	\end{pgfonlayer}
	\begin{pgfonlayer}{edgelayer}
		\draw [style=transition, bend right=60, looseness=1.25] (0) to node[auto]{$a$} (1);
		\draw [style=transition] (1) to node[auto]{$c$} (2);
		\draw [style=transition] (2) to node[auto]{$\inv{c}$} (3);
		\draw [style=transition] (3) to node[auto]{$\inv{a}$} (4);
		\draw [style=transition] (4) to node[auto]{$a$} (6);
		\draw [style=transition] (6) to node[auto]{$c$} (7);
		\draw [style=transition] (7) to node[auto]{$a$} (8);
		\draw [style=transition] (3) to node[auto]{$\inv{b}$} (5);
		\draw [style=transition] (5) to node[auto]{$b$} (9);
		\draw [style=transition] (9) to node[auto]{$c$} (10);
		\draw [style=transition] (10) to node[auto]{$b$} (8);
		\draw [style=transition, in=-150, out=150, loop] (1) to node[auto]{$a$} ();
		\draw [style=transition, in=30, out=-30, loop] (1) to node[auto]{$b$} ();
		\draw [style=transition, bend left=45, looseness=1.25] (0) to node[auto]{$b$} (1);
		\draw [style=transition, in=90, out=0, loop] (8) to node[auto]{$a$} ();
		\draw [style=transition, in=0, out=-90, loop] (8) to node[auto]{$b$} ();
	\end{pgfonlayer}
\end{tikzpicture}
\end{center}

We begin with proving the equivalence between both automata models. To do so, we
prove the following lemma.

\begin{lemma}
\[ L_{\fa{B}} \subset \genbyone{L_{\fa{D}}} \]
\end{lemma}

\begin{proof}
To each computation $f_k,\ k \in \mathbb{N}$, of the automaton
$\fa{B}$ we assign a path $\pi \in \pathcat{G}(S, V)$. We use induction over the
length of the computations.

{\em Induction base:}

To the empty computation $f_0$ we assign the empty path $1_{(q_0, R_2)}$ where
$q_0 \in S_{\fa{B}}$ is a start state of $\fa{B}$.

It holds: $\alpha(1_{(q_0, R_2)}) = \epsilon = u_0$.

{\em Induction step:}

For all computations $f_k$ of length $k$ we assume to have a path $\pi_k \in
\pathcat{G}(S, V)$ such that $Z(\pi_k) \in Q \times \setof{(R_2, L_2)}$ and
\[ |\alpha(\pi_k)| = \begin{cases}
u_k & \text{for }Z(\pi_k) \in Q \times \setof{R_2} \\
u_k \cdot First(v_k) & \text{for }Z(\pi_k) \in Q \times \setof{L_2}
\end{cases} \]

Here $First(w)$ denotes the first symbol of the word $w$.

Let $f_{k+1}$ be a computation of length $k+1$ which ends with the computation
step
\[ u_k z_k v_k \to u_{k+1} z_{k+1} v_{k+1} \]

\begin{itemize}
  \item[Case 1:] $Z(\pi_k) = (q_k, R_2),\ v_k = a \bar{v}_k,\ (q_{k+1}, R) \in
  \delta(q_k, a)$
  
  By construction, $G$ contains an edge $e: (q_k, R_2) \edge{a} (q_{k+1}, R_2)$.
  
  We set $\pi_{k+1} := \pi_k \cdot e$ and get
  \[ |\alpha(\pi_{k+1})| = |\alpha(\pi_k)| \cdot a = u_{k+1} \]
  
  \item[Case 2:] $Z(\pi_k) = (q_k, R_2),\ u_k = \bar{u}_k b,\ v_k = a
  \bar{v}_k,\ (q_{k+1}, L) \in \delta(q_k, a)$
  
  By construction, in $G$ there exists a path
  \[ (q_k, R_2) \edge{e_1 / a} (q_{k+1}, L_1) \edge{e_2 / \inv{a}} (q_{k+1},
  L_2),\ e_1, e_2 \in E \]
  
  We set $\pi_{k+1} := \pi_k \cdot e_1 \cdot e_2$, then it is
  \[ |\pi_{k+1}| = |\pi_k| = u_{k+1} \cdot b = u_{k+1} \cdot First(v_{k+1}) \]
  and $Z(\pi_{k+1}) \in Q \times \setof{L_2}$.
  
  \item[Case 3:] $Z(\pi_k) = (q_k, L_2),\ v_k = a \bar{v}_k,\ (q_{k+1}, R) \in
  \delta(q_k, a)$
  
  In $G$ there exists a path 
  \[ (q_k, L_2) \edge{e_1 / \inv{a}} (q_{k+1}, R_1) \edge{e_2 / a} (q_{k+1},
  R_2),\ e_1, e_2 \in E \]
  
  We set $\pi_{k+1} := \pi_k \cdot e_1 \cdot e_2$, then it is $Z(\pi_{k+1}) \in
  Q \times \setof{R_2}$ and
  \[ |\alpha(\pi_{k+1})| = |u_k a \inv{a} a| = u_k a = u_{k+1} \]
  
  \item[Case 4:] $Z(\pi_k) = (q_k, L_2),\ u_k = \bar{u}_k b,\ v_k = a
  \bar{v}_k,\ (q_{k+1}, L) \in \delta(q_k, a)$
  
  In $G$ there exists an edge $e: (q_k, L_2) \edge{\inv{a}} (q_{k+1}, L_2)$.
  
  We set $\pi_{k+1} := \pi_k \cdot e$, then $Z(\pi_{k+1}) = (q_{k+1}, L_2)$ and
  it is
  \[ |\alpha(\pi_{k+1})| = |\alpha(\pi_k)| \cdot \inv{a} = |u_k \cdot a \inv{a}|
  = |u_k| = u_{k+1} \cdot First(v_{k+1}) \]
\end{itemize}

These are all possible cases for $\pi_{k+1}$ and $\pi_{k+1}$ fulfills the
induction claim for $k+1$.

Let now $f_n$ be an accepting computation for a word $v$, so $v_n = \epsilon,\
u_n = v$ and $z_n \in F$.

The last move of $\fa{B}$ was a right-move. It holds $Z(\pi_n) = (q_n, R_2)$ and
therefore it holds $|\alpha(\pi_n)| = u_n = v$.

For each subpath $\omega$ of a path $\pi_n$ constructed in that way we get
\[ \forall \omega \prefix \pi_n \text{ it holds: }|\alpha(\omega)| \prefix
|\alpha(\pi_n)| \]

Because $f_n$ is an {\em accepting} computation, i.e.\ the reading head is
placed to the right of the right-most input symbol, each input symbol must have 
been traversed one time more often to the right than to the left.

This means, if a subpath $e \cdot e'$ with label $\alpha(e \cdot e') = a
\inv{a}$ exists in $\pi_n$, then it must be followed by an edge $e''$ with
$\alpha(e'') = a$.

The prefix condition of $\pi_n$ therefore cannot be violated by cancellation of
an edge labelled with $\inv{a}$.

It therefore holds: $|\alpha(\pi_n)| = u_n = v \in \genbyone{L_{\fa{D}}}$.
\end{proof}

\bigskip
\begin{lemma}
\[ \genbyone{L_{\fa{D}}} \subset L_{\fa{B}} \]
\end{lemma}

\begin{proof}
Let $G = (V, E)$ be the graph of $\fa{D}$, $v \in \genbyone{L_{\fa{D}}}$ and
$\pi \in \pathcat{G}(S, F)$ be a path with label $|\alpha(\pi)| = v$, and for
any path prefix $\omega \prefix \pi$ it should hold $|\alpha(\omega)| \prefix
|\alpha(\pi)|$.

We want to construct a computation $f$ which corresponds to the path $\pi$ in
the sense of the previous lemma.

We consider the sequence of subpaths
\[ 1_{(q_0, R_2)} \prefix \pi_1 \prefix \pi_2 \ldots \prefix \pi_n = \pi \]
with $\len{\pi_i} - \len{\pi_{i-1}} \leq 2$ and $Z(\pi_i) \in Q \times
\setof{R_2, L_2}$ for $i = 1, \ldots, n$ and for $\pi_i \prefix \omega \prefix
\pi_{i+1},\ \pi_i \neq \omega \neq \pi_{i+1}, \Rightarrow Z(\omega) \in Q \times
\setof{R_1, L_1}$.

We apply the automaton $\fa{B}$ on the input $v$ and show that there exists a
computation $f$ which corresponds to the path $\pi$.

\medskip
To the empty path $\pi_0 = 1_{(q_0, R_2)}$ corresponds the computation $q_0 v$.

Let $f_k$ be a computation ending with $u_k z_k v_k$.

Here, we also have to consider 4 cases:

\begin{itemize}
  \item[Case 1:] $Z(\pi_k) = (q_k, R_2),\ \pi_{k+1} = \pi_k \cdot e$ with $Z(e)
  = (q_{k+1}, R_2)$ and $\alpha(e) = a$.
  
  It then holds $v_k = a \bar{v}_k$ and by construction of $\fa{D}$ it holds
  $(q_{k+1, R}) \in \delta(q_k, a)$, therefore $u_{k+1} q_{k+1} v_{k+1}$ is
  a configuration following $u_k q_k v_k$.
   
  \item[Case 2:] $Z(\pi_k) = (q_k, R_2),\ \pi_{k+1} = \pi_k \cdot e_1 \cdot
  e_2$.
  
  We then have $Z(e_1) = (q_{k+1}, L_1),\ Z(e_2) = (q_{k+1}, L_2)$ and
  $\alpha(e_1) = a$ and $\alpha(e_2) = \inv{b}$.
  
  Because of $|\alpha(\pi)| \in X^*$ it follows $b = a$ and we have
  \[ (q_{k+1}, L) \in \delta(q_k, a)\text{ and }u_{k+1} = u \inv{a},\
  v_{k+1} = a v_k \]
  and $u_{k+1} q_{k+1} v_{k+1}$ is a configuration following $u_k q_k v_k$.
  
  \item[Case 3:] $Z(\pi_k) = (q_k, L_2),\ \pi_{k+1} = \pi_k \cdot e$.
  
  It then holds $Z(e) = (q_{k+1}, L_2)$ and $\alpha(e) \in \inv{X}$. Let
  $\alpha(e) = \inv{a}$. We then have $|\alpha(\pi_{k+1})| = |\alpha(\pi_k)
  \cdot \inv{a}|$. Because of $|\alpha(\pi| \in X^*,\ \pi \in \pathcat{G}(S, V)$
  and $|\alpha(\pi_k| = u_k \cdot First(v_k)$ it follows $First(v_k) = a$.
  
  If we have $u_k = \bar{u}_k b$, then for $u_{k+1} = \bar{u}_k$ and $v_{k+1} =
  b a v_k$ it holds: $u_{k+1} q_{k+1} v_{k+1}$ is a configuration following
  $u_k q_k v_k$.
  
  \item[Case 4:] $Z(\pi_k) = (q_k, L_2),\ \pi_{k+1} = \pi_k \cdot e_1 \cdot
  e_2$.
  
  It follows $Z(e_1) = (q_{k+1}, R_1)$ and $Z(e_2) = (q_{k+1}, R_2)$ and
  $\alpha(e_1) = \inv{a} \in \inv{X},\ \alpha(e_2) = b \in X$.
  
  Because of $\alpha(\pi) \in X^*,\ \pi \in \pathcat{G}(S, V)$ it follows
  $|\alpha(\pi_k)| = u_k a$ and because $u_k a \inv{a} b \prefix |\alpha(\pi)|$
  and $u_k a \prefix |\alpha(\pi_{k+1})|$ it follows $b = a$.
  
  Therefore $\alpha(\pi_{k+1}) = u_k a$. If we set $u_{k+1} = u_k a$ and
  $v_{k+1} = \inv{a} v_k\ (v_k = a \bar{v}_k)$, then by construction of
  $\fa{D}$, $u_{k+1} q_{k+1} v_{k+1}$ is a configuration following $u_k q_k v_k$
  and the path $\pi_{k+1}$ corresponds to the computation $f_k$.
\end{itemize}

Let $v \in \genbyone{L_{\fa{D}}}$ and $\pi_n \in \pathcat{G}(S, F)$ with label
$|\alpha(\pi_n)| = v$ and let for $\omega \prefix \pi_n$ hold $|\alpha(\omega)|
\prefix |\alpha(\pi_n)|$, then there exists a computation $f_n = (q_0 v \to
\ldots \to v q_n)$ with $q_0 \in S_{\fa{B}},q_n \in F_{\fa{B}}$ and therefore
$v \in L_{\fa{B}}$.
\end{proof}

Both lemmata are summarized in the following theorem:

\begin{theorem}
\[ L_{\fa{B}} = \genbyone{L_{\fa{D}}} \]
\end{theorem}

\bigskip
We will show now that for all finite two-way automata $\fa{B}$ it holds
\[ L_{\fa{B}} \in \reglang(X^*)\ (= \ratlang(X^*)) \]

Before doing that, we show a relationship between $\ratlang(X^*)$ and
$\ratlang(\hgroup{X})$.

For the automaton $\fa{D}$ constructed above we set
\[ L_{\fa{D}} := \setof{|\alpha(\pi)| \mid \pi \in \pathcat{G}(S, F)} \]
where $G = (V, E)$ is the graph of $\fa{D}$.

In this way, $\ratlang(\hgroup{X})$ is embedded in a natural way into
$(\unioninv{X})^*$, see chapter II.7.

We set
\[ |\ratlang(\hgroup{X})| := \setof{|L| \mid L \in \ratlang(\hgroup{X})} \]

Then the following theorem holds:

\begin{theorem}
\[ |L| \cap X^* \in \ratlang(X^*) \]
\end{theorem}

\begin{proof}
We only sketch the proof of this theorem, the reader may give the proof
formally.

Let $q_1, q_2$ be vertices of the graph $G = (V, E)$ of the automaton $\fa{D}$,
and let $\pi: q_1 \to q_2$ be a path with label $\alpha(\pi) \equiv 1
\pmod{\hgroup{X}}$.

In this case, a new edge $e: q_1 \edge{\epsilon} q_2$ will be added to the
graph. This will be done for all pairs $(q_1, q_2) \in V \times V$ which are
connected by such a path.

We get a new automaton $\fa{D}'$ with $|L_{\fa{D}}| = |L_{\fa{D}'}|$.

If one removes all edges $e$ from the graph of $\fa{D}'$ with label 
$\alpha(e) \in \inv{X}$, one gets a finite automaton $\fa{A}$ accepting the 
language $|L| \cap X^*$ (exercise).
\end{proof}

Remark: A similar result can be proved for the free group $F(X)$
(exercise).

\bigskip
\begin{theorem}[Rabin, Scott]
\[ \genbyone{L_{\fa{D}}} \in \reglang(X^*) \]
\end{theorem}

\begin{proof}
The proof gets clearer if we don't use the state graph but the automaton model
used in the construction of $\fa{D}$.

\missingfigure

On the input tape we have $w = x_1 \cdots x_n,\ x_i \in X$, the automaton shall
be in state $(q, y),\ y \in \setof{L_2, R_2})$ and the reading head shall be
located to the right of the $k$th cell.

We set $w_k := x_1 \cdots x_k$ and ask in which states the automaton will return
to its current position for $y = L_2$.

We denote the set of these states as the {\bf echo} of state $q$ in $w_k$.

We don't assume that the automaton has read the tape from the left, but think of
the head to be placed arbitrarily at this position in state $(q, L_2)$.

We define the echo formally:
\newcommand{\echo}[2]{\mathrm{Echo}_{#1}(#2)}

Define for $v \in X^*,\ a \in X$ the relation $\mathrm{Echo}_v \subset V \times
\powset{V}$ by
\begin{enumerate}
  \item $\echo{\epsilon}{(q, L_i)} = \emptyset$ for $i = 1, 2$.
  
  \item $\echo{v}{(q, R_i)} = (q, R_i)$ for $i = 1, 2$.
  
  \item $\echo{v \cdot a}{(q, L_1)} = a \cdot \echo{v}{(\inv{a}(q, L_1))}$
  
  \item $\echo{v \cdot a}{(q, L_2)} = (a \cdot \echo{v}{\inv{a}})^+ (q L_2)$
\end{enumerate}

\end{proof}






























