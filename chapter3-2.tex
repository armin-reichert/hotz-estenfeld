\section{Closure properties of $ALG(X^*)$}

In the last section we defined $ALG(X^*)$ as the class of languages over an
alphabet $X$ accepted by a pushdown automaton.

The name ''algebraic languages'' remembers the fact that these languages can be
defined as the solution of equation systems with non-commutative variables using
formal power series. See for example \cite{SaSo} or \cite{Berstel77}.

We will prove for $ALG(X^*)$ similar closure properties as for $REG(X^*)$.

In this section we assume without loss of generality that the pushdown automata
have the following properties:

$card(S) = card(F) = 1$ and no edge ends in a start state and no edge leaves a
final state. Additionally, all edges labels have length 0 or 1, $\alpha(e) \leq
1,\ e \in E$.

In a very easy way one gets

\begin{theorem}
Let $\phi : X^* \to Y^*$ be a monoid homomorphism.
\[ L \in ALG(X^*) \Rightarrow \phi(L) \in ALG(Y^*) \]
\end{theorem} 

\begin{proof}
$L \in ALG(X^*) \Rightarrow$ there exists a PDA $\kappa = (\fa{A}, \pdstore{K})$
with $L_{\kappa} = L$. Replace $\alpha$ by $\alpha' := \alpha \circ \phi$ in
the definition of $\fa{A}$. 

For
\[ \fa{A'} = (G, Y, S, F, \alpha') \]
the PDA $\kappa' = (\fa{A'}, \pdstore{K})$ accepts $\phi(L_{\kappa}) = \phi(L)$.
\end{proof}

\begin{theorem}
$ALG(X^*)$ is closed under union, concatenation and Kleene-star operation.
\end{theorem}

{\em Closure under union:}
\[ L,L' \in ALG(X^*) \Rightarrow L \cup L' \in ALG(X^*) \]
\begin{proof}
Let
\[ \kappa = (\fa{A},\pdstore{K}),\quad \fa{A} = (G = (V, E), X, \setof{s},
\setof{f}, \alpha),\quad \pdstore{K} = (X, Y, \delta)
\]
and
\[ \kappa' = (\fa{A'}, \pdstore{K'}), \quad \fa{A'} = (G' = (V', E'), X,
\setof{s'}, \setof{f'}, \alpha'),\quad \pdstore{K'} = (X, Y', \delta')
\]
be pushdown automata accepting $L$ and $L'$.

We may assume that $Y \cap Y' = \emptyset,\ E \cap E' =
\emptyset,\ V \cap V' = \setof{s, f}$ where $s = s'$ and $f = f'$.

Define the PDA $\kappa \cup \kappa'$ by
\[ \fa{A} \cup \fa{A'} := (G \cup G', X \cup X', \setof{s}, \setof{f}, \alpha
\cup \alpha')
\]
and pushdown store
\[ \pdstore{K} \cup \pdstore{K'} := (E \cup E', Y \cup Y' \cup
\setof{\%}, \delta \cup \delta')
\]
where
\[ (\delta \cup \delta')(e, y) = \begin{cases}
\delta(e, y) & e \in E,\ y \in Y \cup \setof{\epsilon} \\ 
\delta'(e, y) & e \in E',\ y \in Y' \cup \setof{\epsilon} \\
\%\% & \text{else} 
\end{cases}
\]

The symbol $\%$ is only introduced to make $\delta \cup \delta'$
complete. If $\%$ is on-top of the pushdown store, then this cannot become
empty anymore. In that way it will become impossible to ''mix'' words from both
languages.

It holds: $L_{\kappa \cup \kappa'} = L_{\kappa} \cup L_{\kappa'}$ (exercise).

The following figure shows our constructions:

FIGURE

\end{proof}

{\em Closure under complex product:}
\[ L,L' \in ALG(X^*) \Rightarrow L \cdot L' \in ALG(X^*) \]
\begin{proof}
We construct the PDA $\kappa'' = (\fa{A''}, \pdstore{K}'')$ with 
\[\fa{A''} = (G'', X, \setof{s}, \setof{f'}, \alpha'')\]
where
\begin{eqnarray*}
G'' &=& (V'', E'') \\
V'' &=& V \cup V' \cup \setof{p},\ p \notin V \cup V' \\
E'' &=& E \cup E' \cup \setof{e_1, e_2},\ e_i \notin E \cup E',\ i = 1, 2
\end{eqnarray*}

For the new edges $e_1, e_2$ it is $e_1: f \edge{\epsilon} p$ and $e_2: p
\edge{\epsilon} s'$, that is $e1$ and $e_2$ form a bridge between $G$ and $G'$.

The labeling of the graph is defined by 
\[ \alpha''(e) = \begin{cases} 
\alpha(e) & e \in E \\
\alpha'(e) & e \in E' \\
\epsilon & e \in \setof{e_1, e_2} 
\end{cases}\]

FIGURE

The pushdown store $\pdstore{K''} = (E'', Y'', \delta'')$ is defined by
\[ Y'' = Y \cup Y' \cup \setof{\$, \%} \text{ with }\setof{\$, \%} \cap (Y \cup
Y') = \emptyset \]
\[ \delta''(e, y) = \begin{cases}
\delta(e, y) & e \in E,\ y\in Y \cup \setof{\epsilon} \\
\delta'(e, y) & e \in E',\ y\in Y' \cup \setof{\epsilon} \\
y \cdot \$ & e = e_1,\ y \in Y \cup \setof{\epsilon} \\
\epsilon & e = e_2,\ y = \$ \\
\%\% & \text{else}
\end{cases}\]

We prove now that the so defined pushdown automaton $\kappa''$ accepts the
complex product $L \cdot L'$.

''$L \cdot L' \subset L_{\kappa''}$'':




\end{proof}



































