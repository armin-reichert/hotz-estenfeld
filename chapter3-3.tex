\section{The theorem of Chomsky-Schützenberger}

The Chomsky-Schützenberger theorem is one of the fundamental theorems in the
theory of formal languages, especially the context-free languages \cite{ChSch}.

In the literature, the proof of this theorem is always based on grammars.

We will give an automata-theoretic access to this theorem. We present this
theorem at this point in the book because from a proof technical point of
view it can be very well be appended to the section on pushdown languages
(chapter III.1). A similar access to this theorem has been given by Goldstine
 \cite{Goldstine77,Goldstine79,Goldstine80}.

The Chomsky-Schützenberger theorem characterizes the context-free languages
by a very simple subclass of context-free languages with the use of a
homomorphism.

In this respect the theorem is an analogon to lemma 7 in chapter II.1 where we
proved that each regular language can be presented as the homomorphic image of
a local language.

We want to formulate the theorem now.

Let $D(\Sigma)$ be the Dyck language over the alphabet $\Sigma$ (see chapter
I.3).

Remark: In the literature, often the language \[ \setof{w\in
(\unioninv{\Sigma})^* \mid |w|= \epsilon\text{ in the free group }F(\Sigma)} \]
is called Dyck language. We denote that language as the {\em symmetric} Dyck
language.

With our notations the following theorem holds:
\begin{theorem}[Chomsky-Schützenberger] For each language $L\in \alglang(X^*)$
there exists
\begin{itemize}
  \item an alphabet $\Sigma \subset X_\infty$
  \item a regular set $R\in \reglang((\unioninv{\Sigma})^*)$
  \item and a monoid homomorphism $\phi: (\unioninv{\Sigma})^* \to X^*$ such
  that
\end{itemize}
\[ L = \phi(D(\Sigma)\cap R) \]
\end{theorem}

(The proof in the original book was hard to read, it has been reformulated
therefore.)

\begin{proof}
$L\in \alglang(X^*) \Rightarrow$ it exists a PDA $\kappa=(\fa{A},\storage{K})$ with
$L = L_\kappa$, where $\fa{A}=(G, X, S, F, \alpha)$ is the corresponding finite
automaton and $G=(V, E)$ the graph of $\fa{A}$.

For the proof we use the graph $G'=(V',E'),\ V' = V \cup \setof{s_0}$, which we
constructed in theorem 2 of section III.1 and the mapping $\gamma: E' \to 
\hgroup{Y}$.

Reminder: The graph $G'$ was constructed from $G$ by adding the following kinds
of edges:
\begin{itemize}
  \item A start edge $e_0 \in \setof{s_0} \times S$ of the form 
  \[e_0: s_0 \edge{y_0} s\] 
  for each start vertex $s \in V$.
  
  \item ''PUSH''-edges $e' \in E \times Y$ of the form
  \[e': Q(e) \edge{y^{-1} y z} Z(e)\]
  for each edge $e \in E$ where the pushdown store computation ''pushes'' $z$,
  i.e.\ $\delta(e, y) = z$. 
  
  The prefix $y^{-1} y$ realizes the non-deterministic guess of the topmost
  stack symbol $y$.
  
  \item ''Empty-stack''-edges $e' \in E \times Y$ of the form
  \[e': Q(e) \edge{y_0^{-1} y_0 \cdot \delta(e, \epsilon)} Z(e)\]
  for each edge $e \in E$.
  
  The prefix $y_0^{-1} y_0$ realizes the non-deterministic guess of the
  empty-stack symbol $y_0$.
  
  \item ''POP''-edges $e' \in E \times Y \times Y_0$ of the form
  \[e': Q(e) \edge{y^{-1} z^{-1} z} Z(e)\]
  for each edge $e \in E$ where the pushdown store ''pops'' $y$ from the stack,
  i.e.\ $\delta(e, y) = \epsilon$.
  
  The suffix $z^{-1} z$ realizes the non-deterministic guess of the topmost
  stack symbol $z$ after the ''pop'' has been executed.
\end{itemize}

Our goal is to construct a graph $\tilde{G}$ whose edges constitute the alphabet
$\Sigma \cup \Sigma^{-1}$ of corresponding brackets.

We define a relation on the edge set of $E'$ which relates two edges if they
constitute corresponding push/pop-operations in computation paths of $G'$.

Formally define the relation $\rho \subset E' \times E'$ as
\[ (e_1, e_2) \in \rho \iff e_1 = (e, y),\ e_2 = (e', z, y),\ \delta(e, y) = y
z,\quad e \in E,\ y,z \in Y \]

What does that mean? Consider a path in the graph $G$:

\begin{center}
\begin{tikzpicture}
	\begin{pgfonlayer}{nodelayer}
		\node [style=state] (0) at (0, -0) {};
		\node [style=state] (1) at (2, -0) {};
		\node [style=state] (2) at (3, -0) {};
		\node [style=state] (3) at (5, -0) {};
		\node [style=state] (4) at (6, -0) {};
		\node [style=none] (5) at (2.5, 0.75) {$\delta(e, y )= yz$};
		\node [style=none] (6) at (5.5, 0.75) {$\delta(f, z)=\epsilon$};
		\node [style=state] (7) at (8, -0) {};
		\node [style=none] (8) at (2.5, -0.5) {PUSH $z$};
		\node [style=none] (9) at (5.5, -0.5) {POP $z$};
		\node [style=none] (10) at (2.5, -1) {};
		\node [style=none] (11) at (5.5, -1) {};
	\end{pgfonlayer}
	\begin{pgfonlayer}{edgelayer}
		\draw [style=transition] (1) to node[auto]{$e$} (2);
		\draw [style=transition] (3) to node[auto]{$f$} (4);
		\draw [style=simple] (0) to (1);
		\draw [style=simple] (2) to (3);
		\draw [style=simple] (4) to (7);
		\draw [style=simple, bend right=15, looseness=1.00] (10.center) to node[auto]{corresponding} (11.center);
	\end{pgfonlayer}
\end{tikzpicture}
\end{center}

In the graph $G'$ there exists a corresponding path

\begin{center}
\begin{tikzpicture}
	\begin{pgfonlayer}{nodelayer}
		\node [style=state] (0) at (0, -0) {};
		\node [style=state] (1) at (2, -0) {};
		\node [style=state] (2) at (3, -0) {};
		\node [style=state] (3) at (5, -0) {};
		\node [style=state] (4) at (6, -0) {};
		\node [style=none] (5) at (2.5, 0.75) {$\delta(e, y = yz$};
		\node [style=none] (6) at (5.5, 0.75) {$\delta(e', z)=\epsilon$};
		\node [style=state] (7) at (8, -0) {};
		\node [style=none] (8) at (2.5, -0.5) {PUSH};
		\node [style=none] (9) at (5.5, -0.5) {POP};
		\node [style=none] (10) at (2.5, -1) {};
		\node [style=none] (11) at (5.5, -1) {};
	\end{pgfonlayer}
	\begin{pgfonlayer}{edgelayer}
		\draw [style=transition] (1) to node[auto]{$e$} (2);
		\draw [style=transition] (3) to node[auto]{$e'$} (4);
		\draw [style=simple] (0) to (1);
		\draw [style=simple] (2) to (3);
		\draw [style=simple] (4) to (7);
		\draw [style=simple, bend right=15, looseness=1.00] (10.center) to node[auto]{corresponding} (11.center);
	\end{pgfonlayer}
\end{tikzpicture}
\end{center}

Edges that are corresponding with respect to push and pop-operations are related
to each other using the relation $\rho$. This relation (between edges) can be
visualized as a bipartite graph:

\begin{center}
\begin{tikzpicture}
	\begin{pgfonlayer}{nodelayer}
		\node [style=state] (0) at (-1, 2) {$e'_1$};
		\node [style=state] (1) at (-1, 1) {$e'_2$};
		\node [style=state] (2) at (-1, -1) {$e'_m$};
		\node [style=state] (3) at (1, 2) {$f'_1$};
		\node [style=state] (4) at (1, 1) {$f'_2$};
		\node [style=state] (5) at (1, -1) {$f'_k$};
		\node [style=none] (6) at (-1, -0) {$\vdots$};
		\node [style=none] (7) at (1, -0) {$\vdots$};
	\end{pgfonlayer}
	\begin{pgfonlayer}{edgelayer}
		\draw [style=transition] (0) to (3);
		\draw [style=transition] (0) to (4);
		\draw [style=transition] (1) to (3);
		\draw [style=transition] (1) to (5);
		\draw [style=transition] (2) to (5);
		\draw [style=transition] (2) to (7.center);
	\end{pgfonlayer}
\end{tikzpicture}
\end{center}

Edges in this relation graph go from ''opening brackets'' to ''closing
brackets''. To each opening bracket a set of corresponding closing brackets is
related. 

To get a one-to-one relation the graph is transformed by multiplying edges
as needed:

\begin{center}
\begin{tikzpicture}
	\begin{pgfonlayer}{nodelayer}
		\node [style=state] (0) at (-1, 3) {$e_1^{(1)}$};
		\node [style=state] (1) at (-1, 1) {$e_1^{(j_1)}$};
		\node [style=state] (2) at (1, 3) {$f_{1}^{(1)}$};
		\node [style=state] (3) at (1, 1) {$f_{1}^{(j_1)}$};
		\node [style=none] (4) at (-1, -3.5) {$\vdots$};
		\node [style=none] (5) at (1, -3.5) {$\vdots$};
		\node [style=state] (6) at (-1, -0.5) {$e_{2}^{(1)}$};
		\node [style=state] (7) at (1, -0.5) {$f_{2}^{(1)}$};
		\node [style=state] (8) at (-1, -2.5) {$e_{2}^{(j_2)}$};
		\node [style=state] (9) at (1, -2.5) {$f_{2}^{(j_2)}$};
		\node [style=none] (10) at (1, -1.5) {$\vdots$};
		\node [style=none] (11) at (-1, -1.5) {$\vdots$};
		\node [style=none] (12) at (1, 2) {$\vdots$};
		\node [style=none] (13) at (-1, 2) {$\vdots$};
	\end{pgfonlayer}
	\begin{pgfonlayer}{edgelayer}
		\draw [style=bijection] (0) to (2);
		\draw [style=bijection] (1) to (3);
		\draw [style=bijection] (6) to (7);
		\draw [style=bijection] (8) to (9);
	\end{pgfonlayer}
\end{tikzpicture}
\end{center}

In this graph, each edge on the left representing a ''push''-operation
(''opening bracket'') has exactly one edge on the right representing the
corresponding ''pop''-operation (''closing bracket''). The edges on the right
are called {\em parallel edges} and the new relation is denoted by $\rho'$.

For edges $(e_1, e_2) \in \rho'$ we also write $e_2 = \inv{e_1}$ and call $e_2$
the {\em inverse} edge to $e_1$.

In the graph $\tilde{G}$ to be constructed we replace the edges of $G'$ by
parallel edges as shown in the following figure:

\begin{center}
\begin{tikzpicture}
	\begin{pgfonlayer}{nodelayer}
		\node [style={filled_vertex}] (0) at (-5, -0) {};
		\node [style={filled_vertex}] (1) at (-3, -0) {};
		\node [style={filled_vertex}] (2) at (0, -0) {};
		\node [style={filled_vertex}] (3) at (2, -0) {};
		\node [style=none] (4) at (1, 0.25) {$\vdots$};
		\node [style=none] (5) at (-1.5, -0) {is replaced by};
	\end{pgfonlayer}
	\begin{pgfonlayer}{edgelayer}
		\draw [style=transition] (0) to node[auto]{$e \in E'$} (1);
		\draw [style=transition, bend left=60, looseness=2.00] (2) to
		node[auto]{$e^{(1)}$} (3); 
		\draw [style=transition, bend right=75,looseness=1.75] (2) to
		node[auto]{$e^{(j_1)}$} (3);
	\end{pgfonlayer}
\end{tikzpicture}
\end{center}

The parallel edges $e^{i}$ have the same source and target as the original
vertex from $G'$, i.e.\ $Q(e^{i}) = Q(e),\ Z(e^{i}) = Z(e)$.

By our construction it holds:
If $(e_1, e_2) \in \rho$ then there exists exactly one pair of parallel edges $(e'_1, e'_2)  \in \rho'$ such that $e'_2 =
\inv{e'_1}$.

From the graph $G'$ we construct a new graph $\tilde{G} = (\tilde{V},
\tilde{E})$ by
\begin{eqnarray*}
\tilde{V} &=& V' \ (=V \cup \setof{s_0}) \\
\tilde{E} &=& \setof{e_2 \mid e_2\text{ is the parallel edge for some $e_1 \in
E'$}}
\end{eqnarray*}

If $e_2 \in \tilde{E}$ is the parallel edge for some $e_1 \in E'$ we write $e_1
= \proj{e_2}$ (projection). For each path $\tilde{\pi} \in \pathcat{\tilde{G}}$ it
holds $\proj{\tilde{\pi}} \in \pathcat{G'}$.

We define the set of {\em valid} paths in $\tilde{G}$ by
\begin{eqnarray*}
R &=& \{ e_1 \ldots e_k \in \pathcat{\tilde{G}} \mid e_i \in \tilde{E} \\
& & \text{and for each pair }(e_i, e_{i+1})\text{ of consecutive edges it
holds:}\\
& & \gamma(\proj{e_i} \circ \proj{e_{i+1}}) \not\equiv 0\text{ in the
polycyclic monoid }\pocymon{Y} \}
\end{eqnarray*} 

The path set $R$ contains exactly those paths in $\tilde{G}$ whose projections
in $G'$ have valid pushdown store computations.

We define the set of {\em valid accepting} paths in $\tilde{G}$ by
\begin{equation*}
R' = \setof{\pi \in R \mid Q(\pi) \in S,\ Z(\pi) \in F}
\end{equation*}

Note that both, $R$ and $R'$ are local sets.

We define the alphabet $\Sigma$ as 
\begin{equation*}
\Sigma := \tilde{E}^+ := \setof{e \in \tilde{E} \mid \proj{e} =
\underbrace{(e,y)}_{\text{''PUSH''}-edge}}
\end{equation*}

Then the edge set of $\tilde{G}$ is divided into $\tilde{E} = \tilde{E}^+ \cup
\tilde{E}^-$.

The proof uses the following two lemmata which describe the relation between
accepting paths in $\tilde{G}$ and accepting paths in $G'$.

\begin{lemma}
If $\tilde{\pi} \in R$ and $\pi \equiv \epsilon \pmod{{<}e\circ \inv{e} =
\epsilon{>}}$ then for the projection $\pi = \proj{\tilde{\pi}}$ holds:
\[ |\gamma(\pi)| = y \inv{y}\text{ for suitable }y \in Y \]
\end{lemma}

\begin{proof}
TODO
\end{proof}

\begin{lemma}
For each path $\pi \in \pathcat{G'}$ with
\[ |\gamma(\pi)| = \inv{y} y \quad\text{for some }y \in Y \]
there exists exactly one path $\tilde{\pi} \in \pathcat{\tilde{G}} \cap R$ with
\[ |\tilde{\pi}| = \epsilon\text{ and }\pi = \proj{\tilde{\pi}} \] 
\end{lemma}

\begin{proof}
TODO
\end{proof}

Similar as after lemma 1 one can conclude that for each path $\pi \in
\pathcat{G'}(S, F)$ with $|\gamma(\pi)| = \inv{y_0}y_0$ there exists exactly one
path $\tilde{\pi} \in \pathcat{\tilde{G}}(S, F) \cap R'$ with
$\proj{\tilde{\pi}} = \pi$ and $|\tilde{\pi}| = \epsilon$.

From both lemmata the proof of the theorem follows:

By definition
\[ L_\kappa = \setof{\alpha(\pi) \mid \pi \in \pathcat{G}(S, F),\ \delta(\pi,
\epsilon) = \epsilon} \]

By theorem 2 from chapter III.1 we get
\begin{eqnarray*}
L_\kappa &=& \setof{\alpha(\pi) \mid \pi \in \pathcat{G'}(s_0, F),\
|\gamma(\pi)| = y_0} \\
&=& \setof{\alpha(\pi) \mid \pi \in \pathcat{G'}(S, F),\
|\gamma(\pi)| = \inv{y_0} y_0}\text{ (by construction of $G'$)}
\end{eqnarray*}

If we define $\tilde{R} := \pathcat{G}(S, F) \cap R'$ we get
\[ \tilde{R} \in \reglang(\unioninv{\Sigma})^* \]
because the intersection of regular languages is again regular language.

\medskip
Together with lemma 1 and lemma 2 it follows
\[ \proj{\tilde{R} \cap D(\underbrace{\tilde{E}^+}_{\Sigma})} = \setof{\pi \in
\pathcat{G'}(s_0, F) \mid |\gamma(\pi)| = y_0} \]

With homomorphism $\phi := \mathrm{proj} \circ \alpha : (\unioninv{\Sigma})^*
\to X^*$ we get
\[ L_\kappa = \phi(D(\Sigma) \cap \tilde{R}) \]
\end{proof}

\bigskip
We even proved slightly more:

Let \[ {<}\kappa, w{>} := \card{\setof{\pi \in \pathcat{G}(S,F) \mid \alpha(\pi)
= w,\ \delta(\pi, \epsilon) = \epsilon}} \]

We get the following
\begin{corollary}
For each PDA $\kappa$ there exists a Dyck language $D(\Sigma)$ and a local set
$R$ over $\unioninv{\Sigma}$ and a monoid homomorphism $\phi:
(\unioninv{\Sigma})^* \to X^*$ such that
\begin{enumerate}
  \item $\phi(D(\Sigma\cap R)) = L_\kappa$
  \item For $w\in X^*$ it holds ${<}\kappa, w{>} = \card{\phi^{-1}(w)}$
\end{enumerate}
\end{corollary}

\begin{proof}\ 

\begin{enumerate}
  \item Chomsky-Schützenberger theorem
  \item Follows from the construction in the proof.
\end{enumerate}
\end{proof}

Problem: Is it possible to always find a PDA $\kappa$ accepting an algebraic
language $L \in \alglang(X^*)$ such that for each word $w$ it holds 
${<}\kappa, w{>} < \infty$?

If the definition of the pushdown store is generalized such that $\delta(X
\times Y) \subset Y*$ is allowed, is then the condition ${<}\kappa, w{>} <
\infty$ always satisfiable?

At the end of this section we want to mention the following for the interested
reader:

Because the Dyck language $D(\Sigma) \in \alglang(X_\infty^*)$ and by theorem 1
in section III.2 it follows that the class of algebraic languages is closed
under monoid homomorphism and by theorem 3 in chapter III.2 under intersection 
with regular sets, from the theorem of Chomsky-Schützenberger it follows that
the class of algebraic languages over the alphabet $X_\infty$ equals the class 
of context-free languages.

The problem above will be investigated in a later chapter.
