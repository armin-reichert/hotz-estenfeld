\section{The finite automaton with pushdown store}

We investigate now a second example (after the 2-way automaton) that represents
a generalization of the finite automaton in our concept, the {\bf finite
automaton with pushdown store}.

For preparation we remember theorem 2 from chapter II.8

\begin{theorem}
For a finite automaton $\fa{A} = (G, \hgroup{X}, S, F, \alpha)$ it holds:
\[ X^* \cap |L_{\fa{A}}| \in REG(X^*) \]
\end{theorem}

Remark: From this theorem the following claims can be derived (exercise):
\begin{enumerate}
  \item With the same assumptions it holds $|L_{\fa{A}}| \in REG((X\cup
  X^{-1})^*)$
  \item Corresponding theorems hold for the free group $F(X)$ and the polycyclic
  monoid \pocymon{X} instead of \hgroup{X}.
\end{enumerate}
  
We want to introduce now the finite automaton with pushdown store. We define
\begin{definition}
$\pdstore{K} = (X, Y, \delta)$ is called {\bf simple pushdown store} with
input $X^*$ and stack-monoid $Y^*$ if
\begin{itemize}
  \item[(K1)] $\delta: X \times Y^* \to Y^*$ is a mapping with
  \begin{eqnarray*}
  \delta(X \times Y) & \subset & Y^2 \cup \{\epsilon\} \\
  \delta(X \times \{\epsilon\}) & \subset & Y
  \end{eqnarray*} 
  \item[(K2)] $\delta(x, u \cdot v) = u \cdot \delta(x, v)$ for $v \in Y^+,u \in
  Y^*$.
\end{itemize}
\end{definition} 

Obviously the simple pushdown store $\pdstore{K}$ is uniquely determined by
$\delta' = \delta |_{X \times (Y \cup \{\epsilon\})}$. Each such mapping can be
continued using (K2) onto $X \times Y^*$ such that one gets a simple
pushdown store.

We extend $\delta$ onto $X^* \times Y^*$ by defining
\begin{eqnarray*}
\delta(\epsilon, u) & = & u \\
\delta(x v, u) & = & \delta(v,  \delta(x,u))
\end{eqnarray*}

The following lemma holds:
\begin{lemma}
For $w \in \hgroup{X}$ there exists a pushdown store
\[ \pdstore{K} =(\unioninv{X}, \unioninv{X}, \delta) \]
with $|w| = \delta(w, \epsilon)$.
\end{lemma}

\begin{proof}
Define
\begin{eqnarray}
\delta(x, \epsilon) & = & x,\ x\in \unioninv{X} \label{delta:1} \\
\delta(x, y) & = & yx,\ x\neq y^{-1} \label{delta:2}\\
\delta(x^{-1}, x) & = & \epsilon,\ x\in X \label{delta:3}
\end{eqnarray}

Let $w\in (\unioninv{X})^*$ and $w = x_1^{\epsilon_1} \cdots x_n^{\epsilon_n}$
with $x_i\in X,\ \epsilon_i\in \{1, -1\}$.

For the sequence $y_1 = \delta(x_1^{\epsilon_1}, \epsilon),\ y_2 =
\delta(x_2^{\epsilon_2}, y_1), \ldots,\ y_n = \delta(x_n^{\epsilon_n}, y_{n-1})$
it holds: $y_n = |w|$ (exercise).
\end{proof}

\transrem{The following example was not in the original book and has been
inserted because the definition \eqref{delta:2} of $\delta$ in the book had an
error.}

Example: Let $X=\setof{a,b}$ and $w=abb^{-1}baa^{-1}\in \hgroup{X}$. It is
easy to see that the reduced word $|w|=ab$. The computation of $\delta(w,\epsilon)$ is
\begin{eqnarray*}
\delta(abb^{-1}baa^{-1}, \epsilon)
&=&\delta(bb^{-1}baa^{-1}, \delta(a,\epsilon)) \\
&=&\delta(bb^{-1}baa^{-1}, a) \\
&=&\delta(b^{-1}baa^{-1}, \delta(b, a)) \\
&=&\delta(b^{-1}baa^{-1}, ab)\\
&=&\delta(baa^{-1}, \delta(b^{-1}, ab)) \\
&=&\delta(baa^{-1}, a \delta(b^{-1}, b)) \\
&=&\delta(baa^{-1}, a \epsilon) \\
&=&\delta(baa^{-1}, a) \\
&=&\delta(aa^{-1}, \delta(b,a)) \\
&=&\delta(aa^{-1}, a b) \\
&=&\delta(a^{-1}, \delta(a,ab)) \\
&=&\delta(a^{-1}, a \delta(a,b)) \\
&=&\delta(a^{-1}, a b a) \\
&=&\delta(\epsilon, \delta(a^{-1}, aba)) \\
&=&\delta(\epsilon, a b \delta(a^{-1}, a)) \\
&=&\delta(\epsilon, a b \epsilon) \\
&=&\delta(\epsilon, a b) \\
&=&a b \\
\end{eqnarray*}

Remark:
\begin{enumerate}
  \item The lemma holds also for the free group $F(X)$.
  \item The lemma does not hold for the polycyclic monoid $\pocymon{X}$
  (exercise).
\end{enumerate}

The reader should extend the definition of the pushdown store to a
pushdown store with 0 such that the lemma holds also for the polycyclic monoid
$\pocymon{X}$.

We can now define the finite automaton with pushdown store.

\begin{definition}[finite automaton with pushdown store, PDA]
$\kappa = (\fa{A}, \pdstore{K})$ is called {\bf finite automaton with
pushdown store} ({\bf PDA}) if $\fa{A} = (G, X, S, F, \alpha)$ is a
finite automaton with graph $G = (V, E)$ and $\pdstore{K} = (E, Y,\delta)$ is a
simple pushdown store.
\end{definition}

\transrem{In the original book, the pushdown store language ($PSL(\kappa)$) was
called ''Kellerklasse'' (cellar class) and denoted by $KKL(\kappa)$.}

Two sets are assigned to the PDA $\kappa$, the {\bf pushdown store language}
$PSL(\kappa)$ which is the set of possible words on the pushdown store when
$\fa{A}$ accepts a word, and the {\bf accepted language} $L_\kappa$ which is the set of
 input words accepted with empty pushdown store.

Formally:
\begin{eqnarray*}
PSL(\kappa) &:=& \{ \delta(\pi, \epsilon) \mid \pi \in \pathcat{G}(S, F) \} \\
L_\kappa &:=& \{ \alpha(\pi) \mid \pi \in \pathcat{G}(S, F),\ \delta(\pi,
\epsilon) = \epsilon \}
\end{eqnarray*}

\begin{definition}
\[ ALG(X_\infty^*) := \setof{L \subset X_\infty^* \mid \text{there exists a PDA
$\kappa$ with $L = L_\kappa$}}
\]
is called the class of {\bf algebraic languages}.
\end{definition}

The pushdown store language $PSL(\kappa)$ is an example for our concept of
generalizing finite automata by changing the monoid. 

Let $\pdstore{K}=(X, Y, \delta)$ be a pushdown store. If to each $x\in X$ we
assign a mapping $\delta_x: Y^* \to Y^*,\ \delta_x(u) := \delta(x, u)$, then we can 
represent a PDA $\kappa$ by the finite automaton
\[ \fa{B} = (G, M, S, F, \beta),\quad G=(V,E) \]
with monoid $M = (\setof{\delta_e \mid e\in E}, \circ)$ and labelling $\beta(e)
= \delta_e$.

It holds:
\[ L_{\fa{B}}(\epsilon) := \setof{\beta(\pi)(\epsilon) \mid \pi
\in \pathcat{G}(S, F)} = PSL(\kappa)
\]

\bigskip
As our main result we get
\begin{theorem}[Pushdown store languages are regular]
\[ PSL(\kappa) \in REG(Y^*) \]
\end{theorem}

Before starting with the formal proof we want to explain the construction of the
automaton for $PSL(\kappa)$ using an example.

Let $\pi = (e_1 e_2 e_3 e_4 e_5 e_6 e_7 e_8)$ a path from \pathcat{G} and the
pushdown store be empty initially.

$\delta$ shall be defined as follows:
\[\begin{array}{r@{\,=\,}l@{\qquad}r@{\,=\,}l@{\qquad}r@{\,=\,}l}
\delta(s_1,\epsilon)&y_1 & \delta(s_2,y_1)&y_1y_2 & \delta(s_3,y_2)&y_2y_3 \\
\delta(s_4,y_3)&\epsilon & \delta(s_5,y_2)&y_2y_4 & \delta(s_6,y_4)&\epsilon \\
\delta(s_7,y_2)&\epsilon & \delta(s_8,y_1)&y_1y_5
\end{array}\]

We assign to the computation of the pushdown store a word $w' \in (Y\cup
Y^{-1})^*$. The reduced word is the current content of the pushdown store,
hereby we have to guess in $w'$ the last symbol of the reduced word
(corresponds to the symbol on top of the stack).

Let a path $\pi \in \pathcat{G},\ G=(V,E),$ be of the form
\[ \edge{e_1} \edge{e_2} \edge{e_3} \edge{e_4} \edge{e_5} \edge{e_6} \edge{e_7}
\edge{e_8} \]

In the automaton to be constructed we will have the following marking
\[ \edge{y_1} \edge{y_1^{-1} y_1 y_2} \edge{y_2^{-1} y_2 y_3} \edge{y_3^{-1}
y_2^{-1} y_2} \edge{y_2^{-1} y_2 y_4} \edge{y_4^{-1} y_2^{-1} y_2}
\edge{y_2^{-1} y_1^{-1} y_1} \edge{y_1^{-1} y_1 y_5}
\]

The first inverse on each edge label is used to ''guess'' the top symbol on
the stack.

The corresponding stack contents are:
\[ y_1 \qquad y_1 y_2 \qquad y_1 y_2 y_3 \qquad y_1 y_2 \qquad y_1 y_2 y_4
\qquad y_1 y_2 \qquad y_1 \qquad y_1 y_5 \qquad \]

By multiplying the edge labels and applying the cancellation rules one gets the
stack contents.

We want to execute this idea formally:

For the proof we construct an automaton $\fa{C}$ over the H-group $\hgroup{Y_0}$
where $Y_0 := Y \cup \setof{y_0},\ y_0 \notin Y$ and the property
\[ Y_0^* \cap |L_{\fa{C}}| = y_0 \cdot KKL(\kappa) \]

From theorem 1 then follows that $y_0 \cdot KKL(\kappa)$ is regular and from
this follows immediately $KKL(\kappa) \in REG(Y^*)$.

\transrem{The proof in the book was completely unreadable to me. Therefore I
used new names, added comments and figures.}

\begin{proof}
Let $\kappa = (\fa{A}, \pdstore{K})$ be a PDA with finite automaton
\[ \fa{A} = (G, X, S, F, \alpha) \]
with graph $G = (V, E)$ and pushdown store $\pdstore{K}$.

We define a finite automaton $\fa{C}$ with the H-group over $Y_0$ as its storage
monoid:
\[ \fa{C} = (G', \hgroup{Y_0}, \setof{s_0}, F, \gamma) \]

The graph $G'=(V',E')$ of $\fa{C}$ is defined as follows:

The vertices are the same as in $G$ plus a new vertex $s_0$ which becomes the
start state of $\fa{C}$:
\[ V' = V \cup \setof{s_0},\ s_0 \notin V \]

The edge set $E'$ contains three different types of edges:
\[ E' = \underbrace{\setof{s_0} \times S}_{\text{start edges}} \quad\cup\quad
\underbrace{E \times Y_0}_{\text{push-edges}} \quad\cup\quad \underbrace{E
\times Y \times Y_0}_{\text{pop-edges}}
\]

{\bf Start edges:} These edges provide the initial
stack content (bottom-of-stack symbol $y_0$).

For each start state $s \in S$ of the PDA $\kappa$ we define an edge
\[ e': s_0 \edge{y_0} s\]
formally:
\[ Q(e') = s_0,\quad Z(e') = s,\quad \gamma(e') = y_0 \]

\bigskip
{\bf Edges for simulating push-operations:} These edges simulate a
push-operation on the stack.
The top-most stack symbol $y$ is ''guessed'' and if the guess was correct, $y$ is
restored and $z$ is pushed on the stack. A wrong guess results in a combination of symbols on 
the stack that cannot be reduced to the empty word (no accepting computation
possible) anymore.
 
For each pair $(e, y) \in E \times Y$ there is an edge
\[ e': Q(e) \edge{y^{-1} y z} Z(e)\]
but only if the pushdown store does not compute the monoid unit for the edge,
formally $|\delta(e, y)| \neq \epsilon$. That means, the pushdown store indeed
puts a symbol on the stack when executing the transition $e$.

\bigskip
{\bf Edges for handling empty stack:} These edges simulate computations with
empty stack. The bottom-of-stack symbol is guessed and if the guess was correct, the
corresponding stack operation is simulated.

For each edge $e \in E$ we define an edge
\[ e': Q(e) \edge{y_0^{-1} y_0 \delta(e, \epsilon)} Z(e) \] 

\bigskip
{\bf Edges for simulating pop-operations:} These edges simulate popping of
symbols from the stack.
The top-most stack symbol $y$ is guessed and if the guess was correct, each
possible symbol $z$ that could be on the stack after popping $y$ is provided as
a possible continuation. If some symbol can never be on the stack after
popping $y$ the computation can never be successful. 

For each triple $(e, y, z) \in E \times Y
\times Y_0$ we define an edge
\[ e': Q(e) \edge{y^{-1} z^{-1} z} Z(e) \]
but only if the pushdown store pops the symbol $y$ when executing the transition
$e$, formally $\delta(e, y) = \epsilon$.

\bigskip
This completes the definition of the finite automaton $\fa{C}$.

We prove first that each accepting computation of the pushdown store may be
simulated with the finite automaton $\fa{C}$:
\[ y_0 \cdot KKL(\kappa) \subset |L_{\fa{C}}| \]

To prove this, we construct for each accepting path $\pi \in
\pathcat{G}(S,F)$ an accepting path $\pi' \in
\pathcat{G'}(\setof{s_0}, F)$ with $|y_0 \cdot \delta(\pi, \epsilon)| = |\gamma(\pi')|$ 
which means that the computation of the pushdown store is equivalent modulo the
H-group to the label of the computation path in $\fa{C}$.

Given an accepting computation $\pi$ in the PDA, we consider all prefixes
$\psi$ of $\pi$ by increasing length. The corresponding prefix of path $\pi'$ in
graph $G'$ will be denoted by $\psi'$.

{\em Induction base:} For prefix $\psi$ of length 0, that is $\psi =
(Q(\pi), Q(\pi))$, we set $\psi' := s_0 \edge{y_0} Q(\pi)$, that
is an edge from the start state $s_0$ to the start vertex of $\pi$. 

We get $\delta(\psi, \epsilon) = \delta(\epsilon, \epsilon) = \epsilon$ and
$\gamma(\psi') = y_0$ which proves the claim for paths of zero length.

{\em Induction step:} We assume that the two prefixes $\psi$ and $\psi'$ have
the same target vertex, $Z(\psi) = Z(\psi')$, and that the computations for
these prefixes fulfill the induction assumption, namely $|\gamma(\psi')| =
|y_0 \cdot \delta(\psi, \epsilon)|$.

Consider the path $\pi = \psi \circ e$ where $e$ is an edge of $G$ and let
$|y_0 \cdot \delta(\psi, \epsilon)| = u \cdot y$ for some symbol $y \in Y_0$.

We have two cases.

{\em Case 1}: $y \in Y$ and $\delta(e, y) = y z$:

By construction of $G'$ there exists an edge $e': Q(e) \edge{y^{-1} y z} Z(e)
\in E \times Y$.

Therefore we have $|y_0 \cdot \delta(\psi \circ e, \epsilon)| = |y_0 \cdot u
\cdot y \cdot z = |\gamma(\psi') \cdot e'|$.

The case $y=y_0$ and $\delta(e, \epsilon) = y$ can be handled similarly.

{\em Case 2}: $y \in Y$ and $\delta(e, y) = |y y^{-1}| = \epsilon$:

In this case it is $u = u' z \neq \epsilon$ and we can write $|y_0 \cdot
\delta(\psi, \epsilon)| = u'zy$.

In $E'$ there exists an edge $e' : Q(e) \edge{y^{-1}z^{-1}z} Z(e) \in E
\times Y \times Y_0$.

We define $\psi' \circ e'$ as the path corresponding to $\psi \circ e$ and get
$|y_0 \cdot \delta(\psi \circ e, \epsilon)| = |\gamma(\psi' \circ e')| = u$.

This proves inductively the inclusion $y_0 \cdot KKL(\kappa) \subset
|L_{\fa{C}}|$.


We prove now:
\[ y_0 \cdot Y^* \cap |L_{\fa{C}}| \subset y_0 \cdot KKL(\kappa) \]

Let $\pi' \in \pathcat{G'}(\setof{s_0}, F)$ and $|\gamma(\pi')| \in y_0 \cdot
Y^*$.

From the definition of $\gamma$ on $E'$ it follows that also for each path
prefix $\psi'$ of $\pi$ it holds: $|\gamma(\psi')| \in y_0 \cdot Y^*$.

We construct for each path $\psi'$ a path $\psi$ with $|\gamma(\psi)| = y_0
\cdot \delta(\pi, \epsilon) \in \pathcat{G}$.

To a path of length 1 we assign the empty path $1_{Z(\psi')}$.

For a path $\psi' = e_1 \circ \psi''$ where $e_1 \in E'$ we assign to $\psi''$
the projection $\pi''$ of $\psi''$ onto $\pathcat{G}$. This projection is a
functor that maps paths into paths.

For $\psi'$ of length 1, our claim $|\gamma(\psi')| = y_0 \cdot \delta(\pi,
\epsilon)$ is obviously true.

Assume the claim to be true for $\psi'$ and $\psi' \circ e'$ to be a subpath of
$\pi'$.

{\em Case 1}:

$\gamma(e') = y^{-1}yz$ so $e' = (e, y) \in E'$. From $|\gamma(\psi' \cdot
e')| \in y_0 \cdot Y^*$ it follows $|\gamma(\psi')| = u \cdot y$, so we have
$\delta(\psi' \circ e')| = y_0 \cdot |\delta(\psi \circ e, \epsilon)|$.

{\em Case 2}:

$\gamma(e') = y^{-1} z^{-1} z$, so $e' = (e, y, z) \in E'$.

As before it follows $\gamma(\psi')| = uyz = \delta(\pi, \epsilon)$. Therefore
it holds also in this case $|\gamma(\psi' \circ e')| = y_0 \cdot |\delta(\psi
\circ e, \epsilon)|$.

By induction follows the inclusion $y_0 \cdot Y^* \cap |L_{\fa{C}}| \subset y_0
\cdot KKL(\kappa)$.

Both inclusions give the proof of our theorem.
\end{proof}

{\bf Exercise:}

Let $L = \setof{ w \cdot w^R \mid w \in X^*}$. Define a PDA accepting $L$ and
prove its correctness.
