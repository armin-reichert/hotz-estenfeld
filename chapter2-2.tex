\section{Rational sets in \texorpdfstring{$X^*,\ \ratlang(X^*)$}{X*,
RAT(X*)}}

In this section we want to introduce a second characterization of the regular
sets over free monoids. We need the following definition.

\begin{definition}
If $M$ is a monoid, then $K \subset \powset{M}$ is called is called {\bf
rationally closed} $\iff K$ is closed unter union $\cup$, complex product
$\cdot$ and Kleene-star $^*$.
\end{definition}

\bigskip
\begin{definition}
$\ratlang(M)$ is the smallest rationally closed subset of $\powset{M}$ which
contains the single element sets $\setof{m},\ m \in M$, and the empty set
$\emptyset$.
\end{definition}

From main theorem 1 it immediately follows:

\begin{lemma}
$\ratlang(X^*) \subset \reglang(X^*)$.
\end{lemma}

From the remark following the main theorems it follow that this also holds for
arbitrary monoids $M$ because $\setof{m} \in \reglang(M)$ for $m \in M$.

This lemma is the first part of the following theorem by Kleene:

\begin{theorem}[Kleene]
\[ \reglang(X^*) = \ratlang(X^*) \]
\end{theorem}

\begin{proof}
We only have to proof the inclusion $\reglang(X^*) \subset \ratlang(X^*)$.

Let $\fa{A} = (G, X, S, F, \alpha),\ G=(V, E)$ be a finite automaton. We show
that $L_{\fa{A}}$ can be generated from $\emptyset$ and $\setof{x},\ x \in X$ 
using the $\cup,\ \cdot$ and $^*$-operations.

We prove this by induction over the number $k$ of edges in the graph of the
automaton when the number of vertices is fixed.

By lemma 1 from chapter II.1, we may assume that the automaton has a single
start state, $\card{S} = 1$.

{\em Induction base:}\\
$k = 0$: Without loss of generality, we may assume $S
\cap F = \emptyset$. It follows $\lang{A} = \emptyset \in \ratlang{X^*}$.

{\em Induction step:}\\
Let the claim be true for $i \leq k,\ k > 1$, and let $G_k$ be a graph with $k$
edges.

Add a new edge $e : q \edge{x} r$ labeled with $x \in X$ to the graph $G_k$. The
resulting graph shall be $G_{k+1}$. 

The set of accepting paths in $G_{k+1}$ can be written as
\begin{eqnarray*}
\pathcat{G_{k+1}}(S, F) & = & \pathcat{G_k}(S, F) \\
& \cup & \pathcat{G_k}(S, q) \cdot e \cdot \Big( \pathcat{G_k}(r, q) \cdot e 
\Big)^* \cdot \pathcat{G_k}(r, F)
\end{eqnarray*}

\begin{center}
\begin{tikzpicture}
	\begin{pgfonlayer}{nodelayer}
		\node [style=state] (0) at (-1, 2) {$q$};
		\node [style=state] (1) at (1, 2) {$r$};
		\node [style=state] (2) at (-3, -0) {$S$};
		\node [style=state] (3) at (3, -0) {$F$};
		\node [style=none] (4) at (0, 3.5) {$\pathcat{G_k}(r, q)$};
	\end{pgfonlayer}
	\begin{pgfonlayer}{edgelayer}
		\draw [style=transition] (2) to node[auto]{$\pathcat{G_k}(S, F)$} (3);
		\draw [style=transition] (2) to node[auto]{$\pathcat{G_k}(S, q)$} (0);
		\draw [style=transition] (0) to node[auto]{$x$} (1);
		\draw [style=transition] (1) to node[auto]{$\pathcat{G_k}(r, F)$} (3);
		\draw [style=transition, bend right=60, looseness=2.00] (1) to (0);
	\end{pgfonlayer}
\end{tikzpicture}
\end{center}

By induction hypothesis it holds:
\begin{eqnarray*}
& \alpha(\pathcat{G_k}(S, F)) \in \ratlang(X^*) \\
& \alpha(\pathcat{G_k}(S, q)) \in \ratlang(X^*) \\
&\alpha(\pathcat{G_k}(r, F)) \in \ratlang(X^*) \\
& \alpha(\pathcat{G_k}(r, q)) \in \ratlang(X^*)
\end{eqnarray*}

Together we get $\alpha\left(\pathcat{G_{k+1}}(S, F)\right) \in \ratlang(X^*)$.
This concludes the proof of Kleene's theorem.
\end{proof}

