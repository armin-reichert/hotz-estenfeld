\usepackage[utf8]{inputenc}
\usepackage{hyperref}
\usepackage{parskip}
\usepackage{rotating}
\usepackage{tikz}
\usetikzlibrary{cd}
\usetikzlibrary{decorations.markings}
\usetikzlibrary{shapes.geometric}
\usetikzlibrary{shadows}
\pgfdeclarelayer{edgelayer}
\pgfdeclarelayer{nodelayer}
\pgfsetlayers{edgelayer,nodelayer,main}
\newlength{\imagewidth}
\newlength{\imagescale}
\tikzstyle{none}=[inner sep=0pt]
\tikzstyle{filled_vertex}=[circle,fill=black!50,minimum size=2pt,inner sep=2pt] 
\tikzstyle{state}=[circle,fill=white,draw=black]
\tikzstyle{oval_state}=[ellipse,fill=white,draw=black]
\tikzstyle{transition}=[-latex,draw=black]
\tikzstyle{simple}=[-,draw=black,line width=1pt]
\tikzstyle{bijection}=[latex-latex,draw=black]
\tikzstyle{path}=[-latex,draw=gray]

% theorem environments
\newtheorem{corollary}{Corollary}[section]
\newtheorem{definition}{Definition}[section]
\newtheorem{example}{Example}[section]
\newtheorem{exercise}{Exercise}[section]
\newtheorem{lemma}{Lemma}[section]
\newtheorem{maintheorem}{Main Theorem}[section]
\newtheorem{theorem}{Theorem}[section]

\newcommand{\missingfigure}{\begin{center}\fbox{A figure is missing
here!}\end{center}}

% custom commands
\newcommand{\alglang}{\bf ALG}
\newcommand{\card}[1]{\mathrm{card}({#1})}
\newcommand{\cflang}{\mbox{{\bf CF}}}
\newcommand{\dderives}[1]{\underset{#1}{\Rightarrow}}
\newcommand{\derives}[1]{\underset{{#1}}{\overset{*}{\Rightarrow}}}
\newcommand{\edge}[1]{\xrightarrow{\ {#1}\ }}
\newcommand{\faequiv}[1]{\underset{\fa{#1}}{\equiv}}
\newcommand{\fa}[1]{\mbox{$\mathfrak{#1}$}}
\newcommand{\genbyone}[1]{\mbox{${<}{#1}{>}$}}
\newcommand{\genby}[2]{\mbox{${<}{#1},{#2}{>}$}}
\newcommand{\hgroup}[1]{\mbox{${#1}^{[*]}$}}
\newcommand{\inv}[1]{{{#1}^{-1}}}
\newcommand{\lang}[1]{\mbox{$L_{\mathfrak{#1}}$}}
\newcommand{\len}[1]{\mathrm{length}({#1})}
\newcommand{\llinlang}{\mbox{{\bf l-LIN}}}
\newcommand{\pathcat}[1]{\mbox{$\mathfrak{W}({#1})$}}
\newcommand{\pocymon}[1]{\mbox{${#1}^{(*)}$}}
\newcommand{\powset}[1]{\mathfrak{P}({#1})}
\newcommand{\pathprefix}{\overset{\tilde}{<}}
\newcommand{\production}[1]{\xrightarrow{{#1}}}
\newcommand{\prefix}{\prec}
\newcommand{\proj}[1]{\mathbf{p}({#1})}
\newcommand{\ratlang}{\mbox{{\bf RAT}}}
\newcommand{\reclang}{\mbox{{\bf REC}}}
\newcommand{\reglang}{\mbox{{\bf REG}}}
\newcommand{\rlinlang}{\mbox{{\bf r-LIN}}}
\newcommand{\setof}[1]{\mbox{$\{#1\}$}}
\newcommand{\store}[1]{\mbox{$\mathcal{#1}$}}
\newcommand{\unioninv}[1]{#1 \cup {#1}^{-1}}
