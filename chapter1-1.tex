\section{Notations, basic terminology}

In this first section we want to informally define the basic terminology that is
used throughout the whole book. We use the usual notations
\begin{eqnarray*}
\mathbb{N} &=& \setof{0, 1, 2, \ldots}\text{ for the natural numbers} \\
\mathbb{Z} &=& \setof{\ldots, -2, -1, 0, 1, 2, \ldots}\text{ for the integer
numbers} \\
\mathbb{Q} &=& \setof{\frac{a}{b} \mid a,b \in \mathbb{Z},\ b \neq 0}\text{
for the rational numbers}
\end{eqnarray*}

For the operations on sets we use $\cup$ for the union and $\cap$ for the
intersection. Also $A \subset B$, $a \in A$, $a \not\in A$, $\bar{A}$, $A
\setminus B$, $A \times B$ and $\emptyset$ have their usual meaning.

For the power set of a set $A$ we write $\powset{A}$. $\card{A}$ denotes
the cardinality of $A$.

Logical implication is denoted by $\Rightarrow$.

{\bf Mappings} are denoted by $f : A \to B$ in which case $f$ is a
total mapping. We write $Q(f) = A,\ Z(f) = B$.\footnote{The letters $Q, Z$ stem 
from the German words ''Quelle'' (source) and ''Ziel'' (target).}

If $f: A \to B,\ g : B \to C$ are mappings, then $f \circ g : A
\to C$ is the {\bf composition} which one gets by applying $f$ first and then 
$g$:
\[(f \circ g)(a) = g(f(a))\]

If $f:A \to B$ and $C \subset A$, then $f(C) = \setof{f(c) \mid c \in C }$.

A subset $R \subset A \times B$ is called a {\bf relation} between $A$ and $B$.
\[R_f = \setof{(a,b) \mid b = f(a) } \subset A \times B\] 
is the relation {\bf induced by} the mapping f or also the {\bf graph} of $f$.

Let $f : A \to B$ be a mapping, $A_1 \subset A$ and $g : A_1 \to
B$ a mapping. $f$ is called the {\bf continuation} of $g$ if $f(a_1) = g(a_1),\ 
a_1 \in A_1$. In this case we also write $f \mid _{A_1} = g$, in words: $f$ {\bf
restricted to} $A_1$.

A {\bf semi-group} consists of a set $M$ and an associative operation on that
set, usually written as a multiplication. If a semi-group is commutative, we
also use ''$+$'' instead of ''$\cdot$''.

\index{monoid}
A semi-group is a {\bf monoid} if $M$ contains a neutral element. We often
denote the neutral element with $1_M$ or shortly $1$. In the commutative case we
often write $0$ instead of $1$.

For $A,B \subset M$ the set
\[ A \cdot B = \setof{a \cdot b \mid a \in A,\ b \in B} \]
is the {\bf complex product} of $A$ and $B$.

A subset $A \subset M$ is a {\bf submonoid} of $M$, if $1_M \in A$ and $A$ is
closed under the operation of $M$.

For an arbitrary subset $A \subset M$, the set $A^*$ is the smallest submonoid
of $M$ containing $A$.

More specific:
\[ A^* = \bigcap_{S \in M(A)} S	\]
where 
\[ M(A) = \setof{S \mid S\text{ is a submonoid of }M, A \subset S} \]

It is easy to see that
\[ A^* = \bigcup_{n \geq 0} A^n\text{ with }A^0 = \setof{1}\text{ and } A^{n+1}=
A^n \cdot A \]

In the same sense $A^+ = A^* \setminus \setof{1}$ is defined for semi-groups.

$A$ is called a {\bf generating system} of $A^*$ resp.\ $A^+$.

A special meaning for us has the set of {\bf words} ({\bf strings}) over
a fixed set (alphabet) )$A$. We define words as the finite sequences
of elements of $A$, for example sequences like $(a,b,d,a,c)$ over the alphabet
$A = \setof{a,b,c,d }$.

We define
\[ \mathrm{WORD}(A) := \setof{\epsilon} \cup A \cup (A \times A) \cup (A \times
A \times A) \cup \ldots \]
as the {\bf set of words (strings) over} $A$. 

The symbol $\epsilon$ denotes the {\bf empty word} over $A$, i.e.\ $A^0 =
\setof{\epsilon}$.

If $v, w \in \mathrm{WORD}(A)$, then $v \cdot w$ is the word which you get by
concatenating $v$ and $w$, formally:
\[ v = (a_1, \ldots, a_k),\ w = (a_{k+1}, \ldots, a_n) \Rightarrow v \cdot
w := (a_1, \ldots, a_n) \]

With this operation $\mathrm{WORD}(A)$ becomes a monoid which usually is also
named $A^*$.

This naming is slightly inconsistent because for the first definition of
the ${}^*$-operator it holds $(A^*)^* = A^*$ but for the second usage of the
${}^*$-operator it holds $(A^*)^* \neq A^*$.

\medskip
The following example should clarify this:

Let $A = \setof{a,b,c}$ and $(a,b,a),\ (b,a) \in A^*$.
\begin{eqnarray*}
& (a,b,a)\cdot(b,a) = (a,b,a,b,a) \in A^* \\
& ((a,b,a),(b,a)) \in (A^*)^*\text{, but }\notin A^*
\end{eqnarray*}

Instead of $(a)$ we just write $a$. In this sense it holds $A \subset A^*$. This
also holds in the sense of the first definition of $A^*$.

If $w = (w_1, \ldots, w_n)$, we set $|w| := n$ and call it the {\bf length} of
$w$. Obviously $|w \cdot v| = |w| + |v|$ and $|\epsilon| = 0$.

(Later in this book, the notation $|w|$ is mainly used for reduced words and
the string length is then denoted by $\len{w}$.)

The {\bf reverse} or {\bf mirror} of a word $w = (w_1, \ldots, w_n)$ is the word
$w^R = (w_n, \ldots, w_1)$ and $\epsilon^R = \epsilon$. It holds:
\[ (w \cdot v)^R = v^R \cdot w^R \]

In $A^*$ the {\bf cancellation rules} hold:
\begin{enumerate}
  \item $w \cdot x = w \cdot y \Rightarrow x = y$
  \item $x \cdot w = y \cdot w \Rightarrow x = y$
\end{enumerate}

We define {\bf left} and {\bf right quotient}:
\[ \inv{X} \cdot Y = \setof{w \mid \exists x \in X,\ y \in Y\text{ with }x
\cdot w = y } \]
and 
\[ X \cdot \inv{Y} = \setof{w \mid \exists x \in X,\ y \in Y\text{ with } w
\cdot y = x } \]

Because of the cancellation rules it holds: $\inv{\setof{w}} \cdot \setof{v}$
and $\inv{\setof{w}} \cdot \setof{v}$ are either empty or contain a single
element.

If $\inv{\setof{w}} \cdot \setof{v}$ is not empty, we call $w$ a {\bf
prefix} of $v$.

if $\setof{w} \cdot \inv{\setof{v}} \neq \emptyset$, we call $v$ a {\bf suffix}
of $w$.

In the future we will sometimes just write $w$ instead of $\setof{w}$ and
$w$ {\bf is prefix of} $v$, if $\inv{w} \cdot v \neq \emptyset$.

\bigskip
\begin{exercise}
Let $X$ be an alphabet, $R \subset X^* \times X^*$ a relation on $X^*$. $R$ is
called
\begin{itemize}
  \item {\bf reflexive}: $\Leftrightarrow$ for all $x \in X^*$ it holds $(x, x)
  \in R$
  \item {\bf symmetric}: $\Leftrightarrow$ for all $x, y \in X^*$ it holds $(x,
  y) \in R \Rightarrow (y, x) \in R$
  \item {\bf transitive}: $\Leftrightarrow$ for all $x, y, z \in X^*$ it holds
  $(x, y) \in R,\ (y, z) \in R \Rightarrow (x, z) \in R$
\end{itemize}

Given a relation $R$, construct a relation $R'$ such that
\begin{enumerate}
  \item $R \subset R'$
  \item $R'$ is reflexive, symmetric and transitive
  \item $R' \subset R''$ for all relations $R''$ that fulfill 1) and 2).
\end{enumerate}

$R'$ is called the {\bf equivalence relation} generated by $R$. 
\end{exercise}
