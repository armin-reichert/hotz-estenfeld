\section{Sets recognizable by homomorphisms, $REC(X^*)$}

In this section we investigate a third possibility for characterizing subsets of
$X^*$.

\begin{definition}
\begin{eqnarray*}
 REC(X^*) & := & \{ L \subset X^* \mid \mbox{ there exists a
finite monoid } H \\
& & \mbox{ and a monoid homomorphism }\mu : X^* \to H \\
& & \mbox{ with } L = \mu^{-1}(T), T \subset H \}
\end{eqnarray*}
is the class of {\bf recognizable sets} over $X$.
\end{definition}

\begin{lemma}
\[ REG(X^*) \subset REC(^*) \]
\end{lemma}

Proof: Let $L \in REG(X^*)$, then there exists a complete, deterministic finite
automaton $\fa{A} = (G, X, S, F, \alpha)$ with graph $G = (V, E)$ accepting $L$.

We define \[ H := Map(V,V) := \{ f : V \to V \mid f \mbox{ is a mapping} \} \]

and define a homomorphism $\mu : X^* \to H$ as follows:

Let $\mu$ be the homomorphic continuation of mapping $\mu'$ where
\begin{eqnarray*}
& \mu'(x)(P) = R & \\
& \Leftrightarrow & \\
& \mbox{ there exists an edge }e \in E\mbox{ with }Q(e) = P, Z(e) = R\mbox{ and
}\alpha(e) = x \in X &
\end{eqnarray*}

We define further \[ T := \{ f \in H \mid f(S) \in F \} \]

Then it holds $\lang{A} = \mu^{-1}(T)$ (exercise).

\begin{lemma}
\[ REC(X^*) \subset REG(X^*) \]
\end{lemma}

Proof: Sei $L \in REC(X^*)$, thus let $H$ be a finite monoid and $\mu : X^* \to
H$ be a monoid homomorphism and $L = \mu^{-1}(T), T \subset H$.

We construct a graph $G = (V, E)$:

Vertex set: $V := H$.

Edges: $E := \{ (h, a) \mid h \in H, a \in X \}$ with $Q((h,a)) = h, Z((h,a))
= h \cdot \mu(a)$.

FIGURE

Let $\fa{A} = (G, X, S, F, \alpha)$ with $S = \{ 1_H \},\ F = T,\ \alpha((h, a))
= a$.

We show: $L = \lang{A}$.

''$\subset$'': Let $w \in L$, so $w \in X^*$ with $\mu(w) \in T$.

If $w = x_1 \cdots w_n$ with $x_i \in X$, then $\mu(x_1) \cdots \mu(x_n) = t
\in T$.

We consider \[ v = (1_H, (1_H, x_1), (\mu(x_1), x_2), (\mu(x_1 \cdot x_2),
x_3), \ldots, (\mu(x_1 \cdots x_{n-1}), x_n), \mu(w)) \]

$v \in \pathcategory{G}$ with $\alpha(v) = x_1 \cdots x_n = w$. It even holds $v \in
\pathcategory{G(S, F)}$ because $\mu(w) \in T \Rightarrow w \in \lang{A}$.

''$\supset$'': Let $w \in \lang{A}$, then there exists a path $v \in
\pathcategory{G}(\{1_h\}, T)$ with $\alpha(v) = w$. For $w = x_1 \cdots x_n$ then
obviously holds $\mu(x_1 \cdots x_n) \in T$ thus $w \in \mu^{-1}(T) = L$ from
which the claim follows.

From both lemmata immediately follows

\begin{theorem}[Regular sets of free monoid are exactly the recognizable sets]
\[ REG(X^*) = REC(X^*) \]
\end{theorem}

In the first section of this chapter we showed closure properties for regular
sets. We show now closure under inverse homomorphism which is rather easy using
our last theorem.

\begin{theorem}[Recognizable sets are closed under inverse homomorphism]
Let $\phi : X^* \to Y^*$ a monoid homomorphism, $L \in REC(Y^*) \Rightarrow
\phi^{-1}(L) \in REC(X^*)$.
\end{theorem}

Let $\phi : X^* \to Y^*$ and $L$ should be defined by monoid homomorphism $\mu'
: Y^* \to H$ as $L = \mu'^{-1}(T)$ for a subset $T \subset H$.

We set $\mu = \phi \circ \mu'$ and have $\mu : X^* \to H$ is a monoid
homomorphism and $\phi^{-1}(L) = \mu^{-1}(T)$ from which follows $\phi^{-1}(L)
\in REC(X^*)$.

Exercise: For the integer numbers $\mathbb{Z}$ the set of recognizable sets
$REC(\mathbb{Z})$ shall be defined in analogy to $REC(X^*)$.

Characterize $REC(\mathbb{Z})$! Are there nonempty, infinite sets in
$REC(\mathbb{Z})$?
