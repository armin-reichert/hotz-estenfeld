\section{The theorem of Chomsky-Schützenberger}

The Chomsky-Schützenberger theorem is one of the fundamental theorems in the
theory of formal languages, especially the context-free languages \cite{ChSch}.

In the literature, the proof of this theorem is always based on grammars.

We will give an automata-theoretic access to this theorem. We present this
theorem at this point in the book because from a proof technical point of
view it can very well be appended to the section on pushdown languages
(chapter III.1). A similar access to this theorem has been given by Goldstine
 \cite{Goldstine77,Goldstine79,Goldstine80}.

The Chomsky-Schützenberger theorem characterizes the context-free languages
by a very simple subclass of context-free languages with the use of a
homomorphism.

In this respect the theorem is an analogon to lemma 7 in chapter II.1 where we
proved that each regular language can be presented as the homomorphic image of
a local language.

We want to formulate the theorem now.

Let $D(\Sigma)$ be the Dyck language over the alphabet $\Sigma$ (see chapter
I.3, page \pageref{dyck-language}).

Remark: In the literature, often the language \[ \setof{w\in
(\unioninv{\Sigma})^* \mid |w|= \epsilon\text{ in the free group }F(\Sigma)} \]
is called Dyck language. We denote that language as the {\em symmetric} Dyck
language.

With our notations the following theorem holds:
\begin{theorem}[Chomsky-Schützenberger] For each language $L\in \alglang(X^*)$
there exists
\begin{itemize}
  \item an alphabet $\Sigma \subset X_\infty$
  \item a regular set $R\in \reglang((\unioninv{\Sigma})^*)$
  \item and a monoid homomorphism $\phi: (\unioninv{\Sigma})^* \to X^*$ such
  that
\end{itemize}
\[ L = \phi(D(\Sigma)\cap R) \]
\end{theorem}

(The proof in the original book was hard to read, it has been reformulated
therefore.)

\begin{proof}
$L\in \alglang(X^*) \Rightarrow$ it exists a PDA $\kappa=(\fa{A},\storage{K})$ with
$L = L_\kappa$, where $\fa{A}=(G, X, S, F, \alpha)$ is the corresponding finite
automaton and $G=(V, E)$ the graph of $\fa{A}$.

For the proof we use the graph $G'=(V',E'),\ V' = V \cup \setof{s_0}$, which we
constructed in theorem 2 of section III.1 and the mapping $\gamma: E' \to 
\hgroup{Y}$.

Reminder: The graph $G'$ has the following types of edges:
\begin{itemize}
  \item A start edge $e'_s \in \setof{s_0} \times S$ labelled by $y_0$ 
  \[ e'_s: s_0 \edge{y_0} s \] 
  for each start vertex $s \in S$.
  
  \item ''PUSH($z$)''-edges $e'=(e, y) \in E \times Y$ of the form
  \[ e': Q(e) \edge{y^{-1} y z} Z(e) \]
  for each edge $e \in E$ where the pushdown store ''pushes'' $z$,
  i.e.\ $\delta(e, y) = yz$ and $|yz| \neq \epsilon$. 
  
  The prefix $y^{-1} y$ realizes the non-deterministic guess of the topmost
  stack symbol $y$.
  
  \item ''Empty-stack''-edges $e'=(e, y) \in E \times Y$ of the form
  \[ e': Q(e) \edge{y_0^{-1} y_0 \cdot \delta(e, \epsilon)} Z(e) \]
  for each edge $e \in E$.
  
  The prefix $y_0^{-1} y_0$ realizes the non-deterministic guess of the
  empty-stack symbol $y_0$.
  
  \item ''POP($z$)''-edges $e'=(e, z, y) \in E \times Y \times Y_0$ of the form
  \[ e': Q(e) \edge{z^{-1} y^{-1} y} Z(e) \]
  for each edge $e \in E$ where the pushdown store ''pops'' $z$ from the stack,
  i.e.\ $\delta(e, z) = \epsilon$.
  
  The suffix $y^{-1} y$ realizes the non-deterministic guess of the topmost
  stack symbol $y$ after the symbol $z$ has been popped.
\end{itemize}

Our goal is to construct a graph $\tilde{G}$ whose edges will constitute the
alphabet $\Sigma \cup \Sigma^{-1}$ of the Dyck language.

First we define a relation $\rho' \subset E' \times E'$ on the edges of the
graph $G'$ which relates two edges if they represent a pair of corresponding
push/pop-operations.

For edges $e', f' \in E'$ define
\[ (e', f') \in \rho \iff e' = (e, y)\text{ with }\delta(e, y) = y z,\quad
f' = (f, z, y),\quad e, f \in E,\ y,z \in Y \]

Consider a path in the graph $G$ of the original pushdown automaton:

\begin{center}
\input{figures/chapter3-3-theorem1-path-in-g.tikz}
\end{center}

The corresponding path in the graph $G'$:

\begin{center}
\begin{tikzpicture}
	\begin{pgfonlayer}{nodelayer}
		\node [style=state] (0) at (0, -0) {};
		\node [style=state] (1) at (2, -0) {};
		\node [style=state] (2) at (3, -0) {};
		\node [style=state] (3) at (5, -0) {};
		\node [style=state] (4) at (6, -0) {};
		\node [style=none] (5) at (2.5, 0.75) {$\gamma(e') = \inv{y} y z$};
		\node [style=none] (6) at (5.5, 0.75) {$\gamma(f') = \inv{z} \inv{y} y$};
		\node [style=state] (7) at (8, -0) {};
		\node [style=none] (8) at (2.5, -0.5) {PUSH $z$};
		\node [style=none] (9) at (5.5, -0.5) {POP $z$};
		\node [style=none] (10) at (2.5, -1) {};
		\node [style=none] (11) at (5.5, -1) {};
	\end{pgfonlayer}
	\begin{pgfonlayer}{edgelayer}
		\draw [style=transition] (1) to node[auto]{$e'$} (2);
		\draw [style=transition] (3) to node[auto]{$f'$} (4);
		\draw [style=simple] (0) to (1);
		\draw [style=simple] (2) to (3);
		\draw [style=simple] (4) to (7);
		\draw [style=simple, bend right=15, looseness=1.00] (10.center) to node[auto]{corresponding} (11.center);
	\end{pgfonlayer}
\end{tikzpicture}
\end{center}

The relation $\rho'$ between the edges of $G'$ can be visualized as a bipartite
graph:

\begin{center}
\begin{tikzpicture}
	\begin{pgfonlayer}{nodelayer}
		\node [style=state] (0) at (-1, 2) {$e_1$};
		\node [style=state] (1) at (-1, 1) {$e_2$};
		\node [style=state] (2) at (-1, -1) {$e_m$};
		\node [style=state] (3) at (1, 2) {$e'_1$};
		\node [style=state] (4) at (1, 1) {$e'_2$};
		\node [style=state] (5) at (1, -1) {$e'_k$};
		\node [style=none] (6) at (-1, -0) {$\vdots$};
		\node [style=none] (7) at (1, -0) {$\vdots$};
	\end{pgfonlayer}
	\begin{pgfonlayer}{edgelayer}
		\draw [style=transition] (0) to (3);
		\draw [style=transition] (0) to (4);
		\draw [style=transition] (1) to (3);
		\draw [style=transition] (1) to (5);
		\draw [style=transition] (2) to (5);
		\draw [style=transition] (2) to (7.center);
	\end{pgfonlayer}
\end{tikzpicture}
\end{center}

The graph of this relation connects each push-edge from $E'$ representing an 
''opening bracket'' with the corresponding pop-edges from $E'$ representing the
corresponding ''closing brackets''.

To get a one-to-one relation, the edges of the relation $\rho'$ are multiplied
as needed. The result is a new relation $\tilde{\rho}$ in which each
opening bracket now has a unique closing bracket:

\begin{center}
\begin{tikzpicture}
	\begin{pgfonlayer}{nodelayer}
		\node [style=state] (0) at (-1, 3) {$e_1^{(1)}$};
		\node [style=state] (1) at (-1, 1) {$e_1^{(j_1)}$};
		\node [style=state] (2) at (1, 3) {$f_{1}^{(1)}$};
		\node [style=state] (3) at (1, 1) {$f_{1}^{(j_1)}$};
		\node [style=none] (4) at (-1, -3.5) {$\vdots$};
		\node [style=none] (5) at (1, -3.5) {$\vdots$};
		\node [style=state] (6) at (-1, -0.5) {$e_{2}^{(1)}$};
		\node [style=state] (7) at (1, -0.5) {$f_{2}^{(1)}$};
		\node [style=state] (8) at (-1, -2.5) {$e_{2}^{(j_2)}$};
		\node [style=state] (9) at (1, -2.5) {$f_{2}^{(j_2)}$};
		\node [style=none] (10) at (1, -1.5) {$\vdots$};
		\node [style=none] (11) at (-1, -1.5) {$\vdots$};
		\node [style=none] (12) at (1, 2) {$\vdots$};
		\node [style=none] (13) at (-1, 2) {$\vdots$};
	\end{pgfonlayer}
	\begin{pgfonlayer}{edgelayer}
		\draw [style=bijection] (0) to (2);
		\draw [style=bijection] (1) to (3);
		\draw [style=bijection] (6) to (7);
		\draw [style=bijection] (8) to (9);
	\end{pgfonlayer}
\end{tikzpicture}
\end{center}

The edges that have been created by multiplying an edge $e' \in E'$ are called
the {\em parallel edges} of $e'$.

For edges $(\tilde{e}, \tilde f) \in \tilde{\rho}$ we also write $\tilde f =
\inv{\tilde e}$ and call $\tilde f$ the {\em inverse edge} to $\tilde e$.

In the graph $\tilde{G}$ the edges of $G'$ are replaced by their parallel edges:

\begin{center}
\begin{tikzpicture}
	\begin{pgfonlayer}{nodelayer}
		\node [style={filled_vertex}] (0) at (-5, -0) {};
		\node [style={filled_vertex}] (1) at (-3, -0) {};
		\node [style={filled_vertex}] (2) at (0, -0) {};
		\node [style={filled_vertex}] (3) at (2, -0) {};
		\node [style=none] (4) at (1, 0.25) {$\vdots$};
		\node [style=none] (5) at (-1.5, -0) {is replaced by};
	\end{pgfonlayer}
	\begin{pgfonlayer}{edgelayer}
		\draw [style=transition] (0) to node[auto]{$e' \in E'$} (1);
		\draw [style=transition, bend left=90, looseness=1.75] (2) to node[auto]{$e'[1]$} (3);
		\draw [style=transition, bend right=90, looseness=1.75] (2) to node[auto]{$e'[j_1]$} (3);
	\end{pgfonlayer}
\end{tikzpicture}
\end{center}

A parallel edge $\tilde{e}^{(i)} \in \tilde E$ has the same source and target as
the original edge $e' \in G'$:
\[ Q(\tilde{e}^{(i)}) = Q(e'),\quad Z(\tilde{e}^{(i)}) = Z(e') \]

By our construction it holds:
If $(e', f') \in \rho'$ then there exists exactly one pair of parallel edges
$(\tilde e, \tilde f) \in \tilde\rho$ such that $\tilde f = \inv{\tilde e}$.

From the graph $G'$ we construct the graph $\tilde{G} = (\tilde{V}, \tilde{E})$
by
\begin{eqnarray*}
\tilde{V} &=& V' \ (=V \cup \setof{s_0}) \\
\tilde{E} &=& \setof{\tilde f \mid \tilde f \text{ is the parallel edge for some
}e' \in E'}
\end{eqnarray*}

If $\tilde f \in \tilde{E}$ is the parallel edge for some $e' \in E'$ we write
$e' = \proj{\tilde f}$ ($e'$ is the {\em projection} of $\tilde f$). 

For each path $\tilde{\pi} \in \pathcat{\tilde{G}}$ it then holds
$\proj{\tilde{\pi}} \in \pathcat{G'}$ which means that each path of parallel
edges is mapped by the projection to a path of edges in $G'$.

\newcommand{\validpaths}{\mathrm{VALID}}
We define the set of {\em valid paths} in $\tilde{G}$ by
\begin{eqnarray*}
\validpaths &=& \{ \tilde e_1 \cdots \tilde e_k \in \pathcat{\tilde{G}}
\mid e_i \in \tilde{E} \\
& & \text{and for each pair }(\tilde e_i, \tilde e_{i+1})\text{ of consecutive
edges holds:}\\
& & \gamma(\proj{\tilde{e}_i} \cdot \proj{\tilde{e}_{i+1}}) \not\equiv 0\text{
in the polycyclic monoid }\pocymon{Y} \,\}
\end{eqnarray*} 

\newcommand{\acceptedpaths}{\mathrm{ACCEPTED}}
We define the set of {\em valid accepted} paths in $\tilde{G}$ by
\begin{equation*}
\acceptedpaths = \setof{\tilde\pi \in \validpaths \mid Q(\tilde\pi) \in
S,\ Z(\tilde\pi) \in F} \subset \validpaths
\end{equation*}

Note that $\validpaths$ and $\acceptedpaths$ both are {\em local} sets.

We define the alphabet $\Sigma$ as the set of those parallel edges that are
projected to edges which execute a push-operation:
\begin{equation*}
\Sigma := \tilde{E}^+ := \setof{\tilde e \in \tilde{E} \mid \proj{\tilde e} =
(e, y)}
\end{equation*}

Then the edge set of the graph $\tilde{G}$ is the union $\tilde{E} =
\tilde{E}^+ \cup \tilde{E}^-$.

For the proof of the theorem we need two lemmata.

A path from $G'$ is called {\em well-nested} if it can be reduced to the empty
path using the relations 
\[ e' \cdot \inv{e'} = \epsilon,\quad e', \inv{e'} \in E' \]
In the same way we define well-nested paths in the graph $\tilde{G}$.

\begin{lemma}
If $\tilde{\pi} \in \validpaths$ is a valid path that is well-nested and
non-empty, then for the projection $\pi' = \proj{\tilde{\pi}}$ holds:
\[ |\gamma(\pi')| = y \inv{y}\text{ for some }y \in Y \]
\end{lemma}

(The lemma from the original book has been reformulated.)
\begin{proof}
The proof uses induction over the path length. Observe that a well-nested
path has even length, therefore a well-nested, non-empty path has length at
least two.

Remember the definition of the Dyck language $D(\Sigma)$ over an alphabet
$\Sigma$:
\begin{eqnarray*}
d \in D(\Sigma) &\Rightarrow& \sigma \cdot d \cdot \inv{\sigma} \in
D(\Sigma)\text{ for each }\sigma \in \Sigma \\
d_1, d_2 \in D(\Sigma) &\Rightarrow& d_1 \cdot d_2 \in D(\Sigma)
\end{eqnarray*}

{\em Induction base:} $\len{\tilde{\pi}}=2$
\begin{eqnarray*}
\tilde{\pi} = \tilde e \cdot \inv{\tilde e} &\Rightarrow& (\tilde e, \inv{\tilde
e}) \in \tilde{\rho} \\
&\Rightarrow& (\proj{\tilde e}, \proj{\inv{\tilde e}}) \in \rho' \\
&& \text{where }\proj{\tilde e} \in E \times Y\text{ is a ''push''-edge} \\
&& \text{and }\proj{\inv{\tilde e}} \in E \times Y \times Y_0\text{ is the
corresponding ''pop''-edge}
\end{eqnarray*}

$\proj{\tilde{\pi}} = \proj{\tilde e} \cdot \proj{\inv{\tilde e}}$. Let
\begin{eqnarray*}
\gamma(\proj{\tilde e}) &=& \inv{y} y z \qquad\text{(''guess $y$;\ push $z$'')}
\\
\gamma(\proj{\inv{\tilde e}}) &=& \inv{z} \inv{y} y \qquad\text{(''pop $z$;\
guess $y$'')}
\end{eqnarray*}

Then
\begin{eqnarray*}
|\gamma(\proj{\tilde{\pi}})| &=& |\gamma(\proj{\tilde e}) \cdot
\gamma(\proj{\inv{\tilde e}})|
\\
&=& |\inv{y} y z \cdot \inv{z} \inv{y} y| \\
&=& |\inv{y} y| \\
&=& \inv{y} y
\end{eqnarray*}

\medskip
{\em Induction step:}

{\em Case 1:} Let the claim be true for a well-nested path $\tilde{\pi}$. Then
$|\tilde{\pi}| = \epsilon$ and for the projection of $\tilde{\pi}$ it holds
\[ |\gamma(\proj{\tilde{\pi}})| = \inv{y_1} \cdot y_1\text{ for some }y_1 \in Y
\]

Let $\tilde e \cdot \tilde{\pi} \cdot \inv{\tilde e} \in \validpaths$ be a valid
path which is well-nested, then $|\tilde e \cdot \tilde{\pi} \cdot \inv{\tilde
e}| = \epsilon$.

Consider the projection
\[ \proj{\tilde e \cdot \tilde{\pi} \cdot \inv{\tilde e}} = \proj{\tilde e}
\cdot \proj{\tilde{\pi}} \cdot \proj{\inv{\tilde e}} \in \pathcat{G'} \]
where
\begin{eqnarray*}
\proj{\tilde e} &=& (e_1, z) \qquad\text{''push $z$-edge''}\\
\proj{\inv{\tilde e}} &=& (e_2, z, y) \qquad\text{''pop $z$-edge''}\\
\end{eqnarray*}
This gives
\begin{eqnarray*}
|\gamma(\proj{\tilde e} \cdot \proj{\tilde{\pi}} \cdot \proj{\inv{\tilde
e}})|&=& |\gamma(\proj{\tilde e}) \cdot \gamma(\proj{\tilde{\pi}}) \cdot
\gamma(\proj{\inv{\tilde e}})| \\
&=&|\gamma((e_1, z)) \cdot  (\inv{y_1} y_1) \cdot \gamma((e_2, z, y))| \\
&=& |(\inv{y} \cdot y \cdot z) \cdot (\inv{y_1} \cdot y_1) \cdot (\inv{z} \cdot
\inv{y} \cdot y)| \\
\end{eqnarray*}
Because $\tilde e \cdot \tilde{\pi} \cdot \inv{\tilde e}$ is well-nested, it
must hold $z = y_1$. Otherwise we had $z \cdot \inv{y_1} \equiv 0$ and therefore
$e \cdot \tilde{\pi} \cdot \inv{e} \equiv 0$ which would be a contradiction to 
our assumption.

Therefore
\begin{eqnarray*}
|(\inv{y} \cdot y \cdot z) \cdot (\inv{y_1} \cdot y_1) \cdot (\inv{z} \cdot
\inv{y} \cdot y)| &=&
|(\inv{y} \cdot y \cdot z) \cdot (\inv{z} \cdot z) \cdot (\inv{z} \cdot
\inv{y} \cdot y)| \\
&=& |\inv{y} \cdot (y \cdot (z \cdot \inv{z}) \cdot (z \cdot \inv{z}) \cdot
\inv{y}) \cdot y| \\
&=& |\inv{y} \cdot y| \\
&=& \inv{y}  \cdot y
\end{eqnarray*}

\medskip
{\em Case 2:} Let $R \ni \tilde{\pi} = \tilde{\pi}_1 \cdot \tilde{\pi}_2$ be a
valid, well-nested path which is the product of two well-nested paths, i.e.\
$|\tilde{\pi}_1| = |\tilde{\pi}_2| = \epsilon$.

Consider 
\begin{eqnarray*}
|\gamma(\proj{\tilde{\pi}})| &=& |\gamma(\proj{\tilde{\pi}_1 \cdot
\tilde{\pi}_2})| \\
&=& |\gamma(\proj{\tilde{\pi}_1}) \cdot \gamma(\proj{\tilde{\pi}_2})| \\
&=& (\inv{y_1} \cdot y_1) \cdot (\inv{y_2} \cdot y_2),\ y_1, y_2 \in
Y\qquad\text{by induction assumption}
\end{eqnarray*}

Because $\tilde{\pi}_1 \cdot \tilde{\pi}_2 \in \validpaths$ is well-nested it 
must hold $y_1 = y_2$. Otherwise $y_1 \cdot \inv{y_2} \equiv 0$ which would 
contradict our assumptions.

Therefore we get
\begin{eqnarray*}
|\gamma(\proj{\tilde{\pi}})| &=& (\inv{y_1} \cdot y_1) \cdot (\inv{y_2} \cdot
y_2)\\
&=& \inv{y_1} \cdot y_1 \cdot \inv{y_1} \cdot y_1 \\
&=& \inv{y_1} \cdot y_1
\end{eqnarray*}
\end{proof}

Of course, the lemma also holds for valid, well-nested {\em accepted}
paths $\tilde{\pi} \in \acceptedpaths,\ |\tilde{\pi}| = \epsilon$. 

Therefore for the projection
\[ \pi = \proj{\tilde{\pi}} \in \pathcat{G'}(S, F) \]
holds
\[ |\gamma(\pi)| = \inv{y_0} y_0 \]

\bigskip
\begin{lemma}
For each path $\pi' \in \pathcat{G'}$ with
\[ |\gamma(\pi')| = \inv{y} y \quad\text{for some }y \in Y \]
there exists exactly one path $\tilde{\pi} \in \pathcat{\tilde{G}} \cap \validpaths$
with 
\[ |\tilde{\pi}| = \epsilon\text{ and }\pi' = \proj{\tilde{\pi}} \] 
\end{lemma}

\begin{proof}
TODO
\end{proof}

Similar as after lemma 1 one can conclude that for each path $\pi' \in
\pathcat{G'}(S, F)$ with $|\gamma(\pi')| = \inv{y_0} y_0$ there exists exactly
one path $\tilde{\pi} \in \pathcat{\tilde{G}}(S, F) \cap \acceptedpaths$ with
$\proj{\tilde{\pi}} = \pi'$ and $|\tilde{\pi}| = \epsilon$.

Using both lemmata the theorem can now be proved:
\begin{eqnarray*}
L_\kappa &=& \setof{\alpha(\pi) \mid \pi \in \pathcat{G}(S, F),\ \delta(\pi,
\epsilon) = \epsilon}\text{ (by definition of PDA)} \\
&=& \setof{\alpha(\pi') \mid \pi' \in \pathcat{G'}(s_0, F),\ |\gamma(\pi')| =
y_0}\text{ (by theorem 2, chapter III.1)} \\
&=& \setof{\alpha(\pi') \mid \pi' \in \pathcat{G'}(S, F),\
|\gamma(\pi')| = \inv{y_0} y_0}\text{ (by construction of $G'$)}
\end{eqnarray*}

If we define $\tilde{R} := \pathcat{\tilde G}(S, F) \cap \acceptedpaths$, 
we get $\tilde{R} \in \reglang(\unioninv{\Sigma})^*$ because the intersection of
regular (local) languages yields again a regular (local) language.

Together with lemma 1 and lemma 2 it follows
\[ \proj{\tilde{R} \cap D(\underbrace{\tilde{E}^+}_{\Sigma})} = \setof{\pi' \in
\pathcat{G'}(s_0, F) \mid |\gamma(\pi')| = y_0} \]

With homomorphism $\phi := \mathbf{p} \circ \alpha : (\unioninv{\Sigma})^*
\to X^*$ we finally get
\[ L_\kappa = \phi(D(\Sigma) \cap \tilde{R}) \]
\end{proof}

We even proved slightly more: Let 
\[ {<}\kappa, w{>} := \card{\setof{\pi \in
\pathcat{G}(S,F) \mid \alpha(\pi) = w,\ \delta(\pi, \epsilon) = \epsilon}}
\]
be the number of accepting computations of the PDA $\kappa$ for the word $w$.

\begin{corollary}
For each PDA $\kappa$ there exists a Dyck language $D(\Sigma)$ and a local set
$R$ over $\unioninv{\Sigma}$ and a monoid homomorphism $\phi:
(\unioninv{\Sigma})^* \to X^*$ such that
\begin{enumerate}
  \item $\phi(D(\Sigma\cap R)) = L_\kappa$
  \item For $w\in X^*$ it holds ${<}\kappa, w{>} = \card{\phi^{-1}(w)}$
\end{enumerate}
\end{corollary}

\begin{proof}\ 

\begin{enumerate}
  \item Chomsky-Schützenberger theorem
  \item Follows from the construction in the proof. (The inverse homomorphism
  gives the different accepting computations for a word.)
\end{enumerate}
\end{proof}

Problem: Is it possible to always find a PDA $\kappa$ accepting 
$L \in \alglang(X^*)$ such that for each word $w$ it holds 
${<}\kappa, w{>} < \infty$?

If the definition of the pushdown store is generalized such that $\delta(X
\times Y) \subset Y^*$ is allowed, is then the condition ${<}\kappa, w{>} <
\infty$ always satisfiable?

This problem will be investigated in a later chapter.

\medskip
At the end of this section, for interested readers we want to mention the
following:

The Dyck language $D(X) \in \alglang(X_\infty^*)$ is algebraic.

By theorem 1 in section III.2 it holds that the class of algebraic languages is
closed under monoid homomorphism, and by theorem 3 in chapter III.2 also closed
under intersection with regular sets.

From the Chomsky-Schützenberger theorem therefore follows that the class of
algebraic languages over the alphabet $X_\infty$ is exactly the class of
context-free languages.
