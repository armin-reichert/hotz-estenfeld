\chapter*{Introduction}

There are several reasons for the interest in the theory of formal languages in
computer science. Practical problems as they arise in the context of the
specification and translation of programming languages find an exact
description in the theory of formal languages and so get accessible to an exact
treatment. Generation processes definable by formal languages can be interpreted as 
non-deterministic automata which represent notational generalizations of a
computer.

These generalizations in general are easier to understand than
deterministic algorithms which contain more details not reflecting the
original problem but are determined by the need for unambiguosly define the
algorithm. This is part of the reason why it is difficult to prove the
correctness of programs in a clear way. Correctness proofs for grammars or
for other language generation mechanisms offer a possibility to study such
proofs on simpler objects.

The theory of formal languages in this respect contains the theory of
algorithms but most often only the theory of {\bf context-free} languages is
treated because of her extraordinary simplicity and beauty.

To the fore of the theory stand different methods for defining
formal language classes, studying their word and equivalence problems and
putting them into different hierarchical classifications.

The generation processes themselves become objects of interest in the theory
because the generation process of a language in case of programming languages
is related to the semantics of the programs.

Of course, in the context of such a pocket book you have to make a strong
selection of topics concerning language classes, generation processes as well as
basic questions. In doing that we let us guide by the wish to keep the
formal machinery as small as possible.

\transrem{In this book you mainly need some knowledge in graph and category
terminology as well as some basic algebraic notions like monoids and
congruences.}

Because the theory of finite automata is the foundation of the whole theory of
formal languages, we start our book with that topic. In developing the theory
we do not consider the technical realization of finite automata by logical
circuits and binary storage devices but rather focus on the basic algorithm 
however it will get realized. Our intuitive notion of a finite automaton
is a {\em finite, oriented graph} whose edges are labeled with the symbols from
the input alphabet of the automaton. Depending on the input string we look for a
 path in the graph labelled with that string. If the end point of a path
 starting from the dedicated {\em start point} of the automaton is one of the
 {\em end points}, our automaton {\em accepts} the input word and doesn't
 otherwise.

We prove the equivalence of this concept with the other known methods for
defining finite automata. We prove the usual closure properties of languages defined
 by finite automata. Additionally we investigate the relation between
deterministic and non-deterministic automata and also 2-way automata.

It is possible to generalize this theory in the direction of considering not
only the free monoid of strings (words) over a finite alphabet but also
arbitrary monoids.

By considering finite automata with output, that is attaching a
second label at the graph's edges, one gets the {\bf rational transducers}. A
detailed theory of these general transductions can be found in the book by
Berstel \cite{Berstel79}.

We restrict ourselves here to special generalizations of the free monoid of
words:
\begin{itemize}
  \item the {\bf free group}
  \item the {\bf H-group} (the relation $x x^{-1} = 1$ holds for $x$ from
  the generating system but not $x^{-1} x = 1$)
	\item the {\bf polycyclic monoid} (in addition to $x x^{-1} = 1$ it holds 
	$x y^{-1} = 0$ for $x \neq y$ and $0 x = x 0 = 0$ for $x,y$ from the generating
	system)
\end{itemize}

By investigating the transductions from free monoids into the polycyclic
monoids one gets now a smooth transition from the theory of finite automata into
the theory of context-free languages.

The corresponding composition of the theory of context-free languages leads to
a simple path to the most important representation theorems including the
theorems of {\bf Chomsky-Schützenberger, Shamir and Greibach}. Also for the
transformation into Greibach normal-form one gets a simple, efficient
algorithm.

As easy as in the case of finite automata one can prove the known closure
properties for context-free languages.

Finally we also prove the equivalence of this representation to the
usual representation of context-free languages by context-free grammars.

Our composition of the theory is very close to the one repeatedly recommended by
Goldstine since 1977 even if it originated independently. The
difference is that we prove Greibach's representation theorem by making our 
automaton deterministic by going from output monoids to {\em monoid
rings}. Doing so you get the theorem of Shamir in a natural way and from this
the theorem of Greibach.

From the theorem of Shamir you can get quite easily the algorithm of Valiant for
deciding the word problem of context-free languages. Because of lack of space
this could not be included into this book, the same holds for the treatment of
the  deterministic languages.

We want to emphasize at this point another advantage of this development of the
theory: As known, the exact formalization of the notion of {\em derivation}
when using grammars brings some difficulties. In our theory, the {\em derivation
tree} corresponds to a path in our graph.

Maybe the use of non-free monoids is initially a problem for readers not used to
it. But it seems to be the case that defining context-free
languages in that way supports the intuition. For example, the usage of
{\em syntax diagrams} for the definition of programming languages gives some
evidence for this.

Because we judge the former as rather important, we want to explain it on a
specific example, the so-called {\bf Dyck language}.

\transrem{Sometimes the Dyck-language is defined such that corresponding
brackets are symmetric and the language defined here is then called Semi-Dyck
language}

\index{Dyck language}
The {\bf Dyck language} $D(X_k)$ contains the correctly nested bracket sequences
over $k$ different pairs of brackets, $k \in \mathbb{N}$.

A formal definition of $D(X_k)$ is:

Let $X_k = \{ x_1, \ldots, x_k \}$ be an alphabet of $k$ elements. Define
$\bar{X}_k = \{ \bar{x}_1, \ldots, \bar{x}_k \}$ such that $\bar{x}_i$ is regarded 
as the closing bracket for $x_i$.

Then it holds:
\begin{enumerate}
  \item $\epsilon \in D(X_k)$
  \item $u, v \in D(X_k) \Rightarrow u \cdot v \in D(X_k)$
  \item $u \in D(X_k) \Rightarrow x_i \cdot u \cdot \bar{x}_i \in D(X_k),\quad i
  = 1, \ldots, k$
  \item $D(X_k)$ is minimal with (1), (2) and (3). 
\end{enumerate}

For $D(X_k)$ we get the following {\em syntax diagram}:

\begin{center}
\begin{tikzpicture}
	\begin{pgfonlayer}{nodelayer}
		\node [style=none] (0) at (0, -0) {};
		\node [style={state_node}] (1) at (-6, -0) {s};
		\node [style=none] (2) at (6, -0) {f};
	\end{pgfonlayer}
	\begin{pgfonlayer}{edgelayer}
		\draw [style=, bend left=60, looseness=1.25] (1) to node{$s_4/\epsilon$} (2.center);
		\draw [style=, bend right=45, looseness=1.00] (2.center) to node{$s_1\epsilon$} (1);
		\draw [style=, bend left, looseness=1.00] (1) to node{$s_{x_1}/x_1$} (0.center);
		\draw [style=, bend left, looseness=1.00] (0.center) to node{$s_{\bar{x}_1}/\bar{x}_1$} (2.center);
		\draw [style=, bend right, looseness=1.00] (1) to node{$s_{x_k}/x_k$} (0.center);
		\draw [style=, bend right, looseness=1.00] (0.center) to node{$s_{\bar{x}_k}/\bar{x}_k$} (2.center);
		\draw [style=, bend left=60, looseness=1.25] (0.center) to node{$s_2/\epsilon$} (1);
		\draw [style=, bend left=60, looseness=1.25] (2.center) to node{$s_3/\epsilon$} (0.center);
	\end{pgfonlayer}
\end{tikzpicture}
\end{center}

If we consider all labellings of paths from vertex $s$ to vertex $f$ we get of
course also words not contained in $D(X_k)$ like for example $x_1 x_2
\bar{x}_k$ or $x_1 \bar{x}_1 \bar{x}_2$ etc.

We have to guarantee that we get Dyck words only. 

To do that, we define a homomorphism from the paths of the graph into the
polycyclic monoid over $X_k \cup \bar{X}_k$ such that the homomorphic images of the paths 
from $S$ to $F$ have a special form, for example are equal to the unit of the
polycyclic monoid.

If we consider the word $ x_1 x_2 \bar{x}_2 \bar{x}_1 x_2 \bar{x}_2 \in D(X_2)$
then we have different paths 
\[s_{x_1} s_2 s_{x_2} s_{\bar{x}_2} s_3 s_{\bar{x}_1} s_1 s_{x_2} s_{\bar{x}_2} \] 
and 
\[s_{x_1} s_2 s_{x_2} s_{\bar{x}_2} s_3 s_{\bar{x}_1} s_3 s_2 s_{x_2}
s_{\bar{x}_2}\] 
which both have this word as label and we can easily define a homomorphism in
the sense.

We get in general different paths in our graph leading to the acceptance of the
same word.

The problem to construct a graph such that for each word in the accepted
language exactly one path exists leads to the existence of the {\bf
deterministic finite automaton with storage}.

\transrem{Finite automata with monoid-storage, also known as
{\em monoid-automata} have been investigated again in recent years, see for
example \cite{doi:10.1080/00927870802243580}, \cite{Render}.}
