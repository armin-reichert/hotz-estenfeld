\section{Regular sets in \pocymon{X}, \reglang(\pocymon{X})}

In the first section of this chapter we looked at $\reglang(X^*)$, the class of
regular languages over the free monoid $X^*$.

We replace now the free monoid by an important special case of a monoid, namely
the polycyclic monoid \pocymon{X} which is the syntactic monoid of the Dyck
language of the alphabet $\unioninv{X}$.

We represent the elements of \pocymon{X} by their reduced words (see chapter
1.3).

\begin{definition}
Let $|w|$ be the reduced word of $w \in (\unioninv{X})^*$ with respect to the
polycyclic monoid \pocymon{X} and let $L \subset (\unioninv{X})^*$ be a
language. Then
\begin{eqnarray*}
|L| &:=& \setof{|w| \mid w \in L,\ |w| \not\equiv 0}
\end{eqnarray*}
is the {\bf set of reduced words} of $L$.
\end{definition}

\bigskip
\begin{definition}
\begin{eqnarray*}
|\reglang|(\pocymon{X}) & := & \{ L \subset (X\cup X^{-1})^* \mid \mbox{there
exists }L' \in \reglang(\pocymon{X}) \\
& & \mbox{ with } L = \setof{|w| \mid [w] \in L'} \}
\end{eqnarray*}
\end{definition}

Our goal is to show the inclusion
\[ |\reglang|(\pocymon{X}) \subset \reglang((\unioninv{X})^*) \]

This inclusion does not tell that the word problem of \pocymon{X} is rational,
or in other words that $[w]$ is a rational set. But the structure of
$\reglang(\pocymon{X})$ is determined for the biggest part by $\reglang(
(\unioninv{X})^*)$.

Of course it is
\[ |\reglang|(\pocymon{X}) \neq \reglang((\unioninv{X})^*) \]

But we can completely characterize $|\reglang|(\pocymon{X})$ as a subset of $\reglang((X
\cup X^{-1})^*)$.

To do so, we first prove some lemmata.

\begin{lemma}
\[ L \subset \reglang((\unioninv{X})^*) \Rightarrow |L| \in \reglang((\unioninv{X})^*)
\]
\end{lemma}

\begin{proof}
Let $\fa{A} = (G, \unioninv{X}, S, F, \alpha)$ be a finite acceptor of $L$. We
construct in several steps an acceptor $\fa{A}'$ for $|L|$.

This construction will play an exceptional role in later chapters.

Without loss of generality we may assume that for each edge $e$ of our graph $G
= (V, E)$ it holds $\alpha(e) \in \unioninv{X} \cup \setof{\epsilon}$.

{\bf Step 1:} If there is a path $\pi \in \pathcat{G}(v, v')$ with
$|\alpha(\pi)| = \epsilon$, then we add a new edge $e: v \to v'$ to $E$ and set
$\alpha(e) = \epsilon$. Any $\epsilon$-loops are removed.

\begin{center}
\begin{tikzpicture}
	\begin{pgfonlayer}{nodelayer}
		\node [style=state] (0) at (-2, -0) {$v$};
		\node [style=state] (1) at (2, -0) {$v'$};
	\end{pgfonlayer}
	\begin{pgfonlayer}{edgelayer}
		\draw [style=path] (0) to node[auto]{$|\alpha(\pi)|=\epsilon$} (1);
		\draw [style=transition, bend left=45, looseness=1.25] (0) to node[auto]{$\epsilon$} (1);
	\end{pgfonlayer}
\end{tikzpicture}
\end{center}

We do this for each such pair $(v, v')\in V \times V$ and get a graph $G_1$ with
corresponding labelling $\alpha$.

{\bf Step 2:} For consecutive edges $e: v_1 \edge{\epsilon} v_2,\ e': v_2
\edge{x} v_3 \in E_1, x \in \unioninv{X}$, we add a ''bridge'' $b: v_1 \edge{x}
v_3$ with label $x$.

For each edge $e_{final}: v_3 \edge{\epsilon} f$, where $f \in F$ is a final
state, we add a ''bridge'' $b_{final}: v_1 \edge{x} f$ with label $x$.

\begin{center}
\begin{tikzpicture}
	\begin{pgfonlayer}{nodelayer}
		\node [style=state] (0) at (-2, -0) {$v_1$};
		\node [style=state] (1) at (0, -0) {$v_2$};
		\node [style=state] (2) at (2, -0) {$v_3$};
		\node [style=state] (3) at (4, -0) {$f$};
	\end{pgfonlayer}
	\begin{pgfonlayer}{edgelayer}
		\draw [style=transition, bend left, looseness=1.25] (0) to
		node[auto]{$\epsilon$} (1); \draw [style=transition, bend left,
		looseness=1.25] (1) to node[auto]{$x$} (2); \draw
		[style=transition, bend left, looseness=1.25] (2) to
		node[auto]{$\epsilon$} (3); \draw [style=transition, bend
		right=60, looseness=1.00] (0) to node[auto]{$x$} (2); \draw
		[style=transition, bend right=60, looseness=1.25] (0) to
		node[auto]{$x$} (3);
	\end{pgfonlayer}
\end{tikzpicture}
\end{center}

If we do this for all such pairs of edges $(e, e')$, we get a graph
$G_2=(V_2,E_2)$ with labelling $\alpha$ as described.

{\bf Step 3:} We remove all edges $e \in E_2$ where $\alpha(e) = \epsilon$. The
resulting graph shall be $G_3=(V_3,E_3)$ with labelling $\alpha$.

Because of our construction we get:
\[ L_{\fa{A}_3} = \setof{v \mid v \mbox{ is reduced word of } w \in
L_{\fa{A}} \mbox{ wrt. } x \cdot x^{-1} = 1} \]

{\bf Step 4:} We interrupt all paths with a label containing $x \cdot y^{-1},\
x\neq y$. To do so, we duplicate certain vertices and edges of $G_3$.

Let $v$ be a vertex which is the target of an edge $e$ with $\alpha(e) \in X$
and the source of an edge $e'$ with $\alpha(e') \in X^{-1}$.

We replace vertex $v$ by vertices $v$ and $v^+$ and add edges as follows:

All edges $e_1$ with $Z(e_1) = v$ and $\alpha(e_1) \in X$ will be rerouted to
vertex $v^+$. Edges $e_2$ with $Z(e_2) = v$ and $\alpha(e_2) \in X^{-1}$ are
left unchanged.

All edges $e_3$ with $Q(e_3) = v$ and $\alpha(e_3) \in X$ will be duplicated
into $e_3^+$ and $e_3^-$. One of them leads from $v^+$ and the other from $v$ to
$Z(e_3)$.

The labelling is defined by $\alpha(e_3^+) = \alpha(e_3^-) = \alpha(e_3)$.

Edges $e_4$ with label $\alpha(e_4) \in X^{-1}$ and $Q(e_4) = v$ stay unchanged.

The following figure shows the construction:

\begin{center}
\begin{tikzpicture}
	\begin{pgfonlayer}{nodelayer}
		\node [style=state] (0) at (0, -0) {$v$};
		\node [style=state] (1) at (-2, 1) {};
		\node [style=state] (2) at (2, 1) {};
		\node [style=state] (3) at (-2, -1) {};
		\node [style=state] (4) at (2, -1) {};
	\end{pgfonlayer}
	\begin{pgfonlayer}{edgelayer}
		\draw [style=transition] (1) to node[auto]{$a$} (0);
		\draw [style=transition] (0) to node[auto]{$b$} (2);
		\draw [style=transition] (0) to node[auto]{$d^{-1}$} (4);
		\draw [style=transition] (3) to node[auto]{$c^{-1}$} (0);
	\end{pgfonlayer}
\end{tikzpicture}
\end{center}

We do this with every such vertex of $G_3$. Tge resulting graph shall be $G'$.
With start and final state sets $S' := S$ and $F' := F \cup \setof{e^+ \mid e\in
F}$ we get a finite automaton
\[ \fa{A}' = (G', \unioninv{X}, ', F', \alpha') \]

It then holds: $\lang{A}' = |L|$!

We observe that $\lang{A}'$ contains only reduced words which are different from
zero. Additionally, the reversal of the construction of $G'$ from $G_3$ is a
functor which preserves the labelling of the edges.

Therefore $\lang{A}' \subset |L|$.

Let to the opposite $\pi$ be a path in $\pathcat{G_3}$ with label $\alpha(\pi)
= |\alpha(\pi)|$, then there is no $(+, -)$-sequence in the label.

That means the path $\pi$ stays unchanged in $G'$ as long as the label of the
edges is in $X^{-1}$.

If the label changes, the path switches to the $+$-vertices and stays
''positive'' until its target. If we denote this path that belongs to $\pi$ by
$w'$, then $\alpha(\pi) = \alpha'(\pi')$ from which follows $|L| \subset
\lang{A}'$.

Together we get $|L| = \lang{A'}$.
\end{proof}

In this proof we even showed more which is summarized in the following
corollary.

\begin{corollary}
To each path $\pi = e_1 \cdots e_k \in \pathcat{G}(S, -),\ e_i \in E$ with
$\alpha(e_k) \neq \epsilon$ and $\alpha(\pi) \neq \epsilon$ there is assigned a
unique $\epsilon$-free path $\pi' \in \pathcat{G_3}(S, -)$ with $\alpha(\pi) =
\alpha(\pi')$.

Additionally, we can define a functor $\nu : \pathcat{G_3} \to \pathcat{G}$ such
that the following holds:

For each accepted word $u \in L$ there exists an accepting path $\pi = e_1
\cdots e_n \in \pathcat{G_3}(S, F)$ with $e_i \in E_3,\ \alpha(e_i) \neq \epsilon$ such
that $\alpha(\nu(\pi)) = u$
\end{corollary}

Proof: Exercise.

Our construction show even more: the number of paths $\pi$ with $\alpha(\pi) =
u$ for a given $u \in |L|$ is the same in $\pathcat{G_3}$ as in $\pathcat{G'}$.
Therefore this number is in $\pathcat{G'}$ less or equal to the number in
$\pathcat{G_2}$. A similar statement holds for $\pathcat{G_1}$ and
$\pathcat{G}$.

This fact, which will be of importance later in this book, is summarized in the
following corollary. We define

\begin{definition}
\[ INDEX(|\fa{A}|, u) := \card{\setof{\pi \in \pathcat{G}(S, F) \mid
|\alpha(\pi) = u}} \]
\end{definition}

\begin{corollary}
\[ INDEX(|\fa{A}|, u) \geq INDEX(|\fa{A}'|, u),\ u \in \lang{A}' \]
\end{corollary}

Remark: Because $\lang{A}' = |\lang{A}'|$ it holds $INDEX(|\fa{A}'|, u) =
INDEX(\fa{A}', u)$.

Now we can prove the announced theorem.

\begin{theorem}
\[ |\reglang|(\pocymon{X}) = \setof{|L| \mid L \in \reglang((X\cup X^{-1})^*)} \]
\end{theorem}

\begin{proof}
 The inclusion ''$\supset$'' is clear.
 
 ''$\subset$'': Every functor $\alpha : \pathcat{G} \to \pocymon{X}$ can be
 extended to a functor $\alpha' : \pathcat{G} \to (\unioninv{X})^*$ by
 defining $\alpha'(e) := |u|$ for each edge $e \in E$ and $\alpha(e) = [u]$.
 
 $\alpha'$ is continued to a functor and for the finite automaton
 \[ \fa{A}' = (G', \unioninv{X}, S', F', \alpha') \]
 it holds:
 \[ \lang{A} = \setof{[u] \mid u \in \lang{A}'} \]
 and with lemma 1 we get:
 \[ \lang{A} = \setof{[u] \mid u \in |\lang{A}'|} \]
 which means
 \[ |\lang{A}| = \setof{|u| \mid u \in \lang{A}'} \]
\end{proof}

The next theorem gives an even more precise characterization for
$|\reglang|(\pocymon{X})$.

\begin{theorem}
Let $\fa{A}$ be a finite automaton
\[ \fa{A} = (G, \unioninv{X}, S, F, \alpha) \]
where $G = (V, E)$ and $\card{E} = c$. Then there exist $c$ pairs $(L_i^+,
L_i^-)$ with 
\[ L_i^+ \in \reglang(X^*),\qquad L_i^- \in \reglang((X^{-1})^*) \]
such that
\[ |\lang{A}| = \bigcup_{i=1}^c L_i^- \cdot L_i^+ \]
\end{theorem}

\begin{proof}
The proof follows directly from the construction of the automaton $\fa{A}'$.

If $V'$ is the set of the vertices $v$ which have an associated vertex $v^+$ in
$G'$, then every path $\pi \in \pathcat{G'}(S', F')$ can be uniquely split into
a product $\pi_1 \cdot \pi_2$ where the target $Z(\pi_1)$ is such a vertex $v$.

Therefore one can write
\[ |\lang{A}| = \bigcup_{v \in V} L_{\fa{A}(S, v)}^- \cdot L_{\fa{A}(v, F)}^+
\]
\end{proof}

\begin{corollary}
\[ INDEX(\fa{A}', u) = \sum_{v \in V' \atop u_1 \cdot u_2 = u}
INDEX(\fa{A}'(S, v), u_1) \cdot INDEX(\fa{A}'(v, F), u_2) \]
\end{corollary}

\begin{lemma}
Let $L = \lang{A}^+ \subset X^*$ and $L' = \lang{A}^- \subset (X^{-1})^*$. Then
there exist $L_1^+,\ldots,L_p^+ \in \reglang(X^*),\ L_1,\ldots,L_q^- \in
\reglang((X^{-1})^*)$ with
\[ |L \cdot L'| = \bigcup_{i=1}^{p} L_i^+ \ \cup\ \bigcup_{j=1}^{q} L_j^- \]
Here $p, q \leq \card{F(\fa{A'})}$.
\end{lemma}

\begin{proof}
We construct $\fa{A} = \fa{A}^+ \circ \fa{A}^-$ with $\lang{A} = L \cdot L'$ and
thereafter $\fa{A'}$ with $\lang{A'} = |\lang{A}|$.

Because for every reduced word $u \in (X\cup X^{-1})^*$ it holds $u = u^- \cdot
u^+$ where $u^- \in (X^{-1})^*, u^+ \in X^*$ the first part of the lemma is
proven.

From our construction of $\fa{A'}$ it follows that the set $F(\fa{A'})$ of final
states are divided into sets from $(X^{-1})^*$ and sets from $X^*$. From that
follows the rest of the lemma.
\end{proof}

We want to investigate in the following the set of regular sets which can be
defined with fixed graph and show two simple but important properties of these
sets.

Let $G = (V, E)$ be a graph and $\alpha : E \to (\unioninv{X})^*$. We may
assume that $\alpha(e) \in \unioninv{X} \cup \setof{\epsilon}$.

We define
\[ \mathcal{R}(G) := \setof{L_{\fa{A}(\tilde{S},\tilde{F})} \mid \fa{A} = (G, X
\cup X^{-1}, S, F, \alpha),\ \tilde{S} \subset V,\ \tilde{F} \subset V} \]

For $c := \card{V}$ then it trivially holds $\card{\mathcal{R}(G)} \leq 2^{2
\cdot c}$.

Remark: The set $\mathcal{R}(G)$ is in general not closed under union but it
contains a simple base for the closure $<\mathcal{R}(G)>$ under union.

For shorter notation we set $\mathcal{R}(\tilde{S},\tilde{F}) :=
L_{\fa{A}(\tilde{S},\tilde{F})}$.

\begin{lemma}
\[ \mathcal{R}(\tilde{S},\tilde{F}) = \bigcup_{\substack{v \in \tilde{S}\\v'
\in \tilde{F}}} \mathcal{R}(v, v') \]
\end{lemma}

\begin{proof}
Exercise for the reader.
\end{proof}

Let $\fa{A'}$ be the automaton constructed as in lemma 1. We construct a graph
$G' = (V', E')$ from the graph $G = (V, E)$. Define
\[ \mathcal{B}(G') := \setof{L_{\fa{A'}(v, v')} \mid v, v' \in V'} \]

\begin{lemma}
Let $L^+ \subset X^*, L^- \subset (X^{-1})^*$ and $L^+, L^- \in
\mathcal{B}(G')$. Then $|L^+ \cdot L^-|$ is contained in the union-closure
$<\mathcal{B}(G')>$ of $\mathcal{B}(G')$.
\end{lemma}

\begin{proof}
Let $L^+ = L_{\fa{A'}(v, v')}$ and $L^- = L_{\fa{A'}(q, q')}$. We create an
automaton
\[ \fa{A}'' := \fa{A'}(v, v') \circ \fa{A'}(q, q') \]
with $L_{\fa{A}''} = L^+ \cdot L^-$.

We get $\fa{A}''$ by connecting an instance of $\fa{A'}(q, q')$ using an
$\epsilon$-edge which leads from $v'$ to $q$ with another instance. Now all
words $u \in L_{\fa{A}''}$ have the form $u = u^+ \cdot u^-$. We are only
interested in the reduced words.

We have the cases:
\[ |u^+ \cdot u^-| = \begin{cases} u_1^+ \\ 0 \\ u_1^- \end{cases} \]

Because every path $\pi$ of $\fa{A}''$ leading from $v$ to $q'$ is divided in
two sections
\[ \pi = \pi^+ e \pi^- \text{ where }\pi^+; v \to v' \in \fa{A'}(v, v') \text{
and }\pi^- : q \to q' \in \fa{A'}(q, q') \]
there exists for the case $|u^+ \cdot u^-|$ a division $\pi^+ = \pi_1^+ \cdot
\pi_2^+$ with $\alpha(\pi_1^+) = u_1^+$ and $|\alpha(\pi_2^+ \cdot e \cdot
\pi^-)| = \epsilon$.

Opposite direction:

If $\bar{\pi_1}$ is a path with $Q(\bar{\pi_1}) = Q(w_1^+),\ Z(\bar{\pi_1}) =
Z(w_1^+)$, then $\alpha(\bar{\pi_1}) \in |L^+ \cdot L^-|$.

The case for $|u^+ \cdot u^-| = u_1^-$ is symmetrical.

Therefore there exist sets $V^+, V^- \subset V'$ with
\[ |L^+ \cdot L^-| = \bigcup_{v^+ \in V^+} \mathcal{R}(v, v^+) \cup
\bigcup_{q^- \in V^-} \mathcal{R}(q^-, q') \]
\end{proof}

Remark: The facts proved above will be useful in connection to the normal-form
theorems of context-free grammars (chapter IV). This is especially true for the
following

\begin{corollary}
If $\mathcal{B}(G', S, F) = \setof{L^+_{\fa{A'}(S, v)},\ L^-_{\fa{A'}(q, F)}
\mid v, q \in V}$. Then $\mathcal{B}(G', S, F)$ is a union base for
\[ \setof{|L^+ \cdot L^-| \mid L^+, L^- \in \mathcal{B}(G', S, F)} \]
\end{corollary}

This corollary gives us a smaller base. For fixed $S$ and $F$ we can represent
each set $|L^+ \cdot L^-|$ as the union of at most $2 \cdot \card{V}$ sets of
type $L^+_{\fa{A}(S, v)}$ resp.\ $L^-_{\fa{A}(q, F)}$.

We now assign to $\mathcal{B}(G)$ a system 
\begin{eqnarray*} 
\mathcal{P}(G) & \subset & \mathcal{B}(G) \times \mathcal{B}(G) \cdot \setof{;}
\cdot \mathcal{B}(G) \\
& \cup & \setof{\epsilon} \times \mathcal{B}(G)
\end{eqnarray*}
in the following way:
\begin{eqnarray*}
L \to L^+ ; L^- \in \mathcal{P}(G) & \iff & \text{in the representation } L^+
  \cdot L^- = \cup L_i \text{ from lemma 2} \\
  & & \text {there exists such an index $i$ with $L = L_i$} \\
\epsilon \to L \in \mathcal{P}(G) & \iff & \epsilon \in L
\end{eqnarray*}

\begin{lemma}
$\epsilon \in |L_1 L_2 \cdots L_n|$ with $L_i \in \mathcal{B}(G) \iff
L_1;L_2;\cdots ;L_n$ can be reduced to $\epsilon$ using the rules from
$\mathcal{P}(G)$.
\end{lemma}

\begin{proof}

''$\Rightarrow$'':

$\epsilon \in |L_1 L_2 \cdots L_n| \Rightarrow$ there exists $u_i \in L_i$ for
$i = 1, \ldots, n$ with $|u_1 \cdots u_n = \epsilon \Rightarrow$ there exists an
index $j$ with $L_j \subset X^*,\ L_{j+1} \subset (X^{-1})^*$. Because $|u_1
\cdots u_n = \epsilon$ it holds $|u_j u_{j+1}| \neq 0$. Therefore $u_j u_{j+1}$
can be reduced to a word $\bar{u}_j$ with $\bar{u}_j \in X^*$ or $\bar{u}_j \in
(X^{-1})^*$.

In the decomposition $L_j L_{j+1} = \cup L'_i$ there exists ab $L'_j$ with
$\bar{u}_j \in L'_j$ and it holds $L'_j \to L_j;L_{j+1} \in \mathcal{P}(G)$.

If we replace $L_j;L_{j+1}$ by $L'_j$ then we have again $\epsilon \in |L_1
\cdots L_{j-1} L'_j L_{j+2} \cdots L_n|$.

Inductively we may assume that $L_1 \cdots L_n$ can be reduced using
$\mathcal{P}(G)$ to some $L'$ and $\epsilon \in |L'|$ holds.

Because $|L'| = L$ we get $\epsilon \in L$. So $L_1;\cdots;L_n$ can indeed be
reduced to $\epsilon$.

''$\Leftarrow$'':

Let $L_1;\cdots;L_n$ be reducible to $\epsilon$ using $\mathcal{P}(G)$ and let
$L'_j \to L_j;L_{j+1} \in \mathcal{P}(G)$.

Then for each $u_i \in L_i,\ i = 1,\ldots,j-1,j+2,\ldots,n$ and $\bar{u}_i \in
L_j'$ there are $u_j \in L_j, u_{j+q} \in L_{j+1}$ with $|u_1 \cdots u_n| =
|u_1 \cdots u_{j-1}\bar{u}_j u_{j+2} \cdots u_n|$.

If $\epsilon \in L_j$ then $\epsilon \to L_j \in \mathcal{P}(G)$ and it holds
$\bar{u}_j = \epsilon$. Therefore $|u_1 \cdots u_{j-1} u_{j+1} \cdots u_n| =
|u_1 \cdots u_n|$.

By induction it follows: if $L_1; \cdots L_n$ can be derived in
$\mathcal{P}(G)$ from $\epsilon$, then $\epsilon \in |L_1 \cdots L_n|$.
\end{proof}

