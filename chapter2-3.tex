\section{Sets recognizable by homomorphisms,
\texorpdfstring{$\reclang(X^*)$}{REC(X*)}}

In this section we investigate a third possibility for characterizing subsets of
$X^*$.

\begin{definition}
\begin{eqnarray*}
 \reclang(X^*) & := & \{ L \subset X^* \mid \mbox{ there exists a
finite monoid } H \\
& & \mbox{ and a monoid homomorphism }\mu : X^* \to H \\
& & \mbox{ with } L = \mu^{-1}(T), T \subset H \}
\end{eqnarray*}
is the class of {\bf recognizable sets} over $X$.
\end{definition}

\begin{lemma}
\[ \reglang(X^*) \subset \reclang(^*) \]
\end{lemma}

\begin{proof}
Let $L \in \reglang(X^*)$, then there exists a complete, deterministic finite
automaton $\fa{A} = (G, X, S, F, \alpha)$ with graph $G = (V, E)$ accepting $L$.

We define 
\[ H := Map(V,V) := \setof{f : V \to V \mid f \mbox{ is a mapping}} \]
and define a homomorphism $\mu : X^* \to H$ as follows:

Let $\mu$ be the homomorphic continuation of mapping $\mu'$ where
\begin{eqnarray*}
& \mu'(x)(P) = R & \\
& \Leftrightarrow & \\
& \mbox{ there exists an edge }e \in E\mbox{ with }Q(e) = P, Z(e) = R\mbox{ and
}\alpha(e) = x \in X &
\end{eqnarray*}

We define further \[ T := \setof{f \in H \mid f(S) \in F} \]

Then it holds $\lang{A} = \mu^{-1}(T)$ (exercise).
\end{proof}

\bigskip
\begin{lemma}
\[ \reclang(X^*) \subset \reglang(X^*) \]
\end{lemma}

\begin{proof}
Sei $L \in \reclang(X^*)$, thus let $H$ be a finite monoid and $\mu : X^* \to
H$ be a monoid homomorphism and $L = \mu^{-1}(T), T \subset H$.

We construct a graph $G = (V, E)$:

Vertex set: $V := H$.

Edges: $E := \setof{(h, a) \mid h \in H, a \in X}$ with $Q((h,a)) = h,
Z((h,a)) = h \cdot \mu(a)$.

\missingfigure

Let $\fa{A} = (G, X, S, F, \alpha)$ with $S = \setof{1_H},\ F = T,\ \alpha((h,
a)) = a$.

We show: $L = \lang{A}$.

''$\subset$'': Let $w \in L$, so $w \in X^*$ with $\mu(w) \in T$.

If $w = x_1 \cdots w_n$ with $x_i \in X$, then $\mu(x_1) \cdots \mu(x_n) = t
\in T$.

We consider \[ v = (1_H, (1_H, x_1), (\mu(x_1), x_2), (\mu(x_1 \cdot x_2),
x_3), \ldots, (\mu(x_1 \cdots x_{n-1}), x_n), \mu(w)) \]

$v \in \pathcat{G}$ with $\alpha(v) = x_1 \cdots x_n = w$. It even holds $v \in
\pathcat{G(S, F)}$ because $\mu(w) \in T \Rightarrow w \in \lang{A}$.

''$\supset$'': Let $w \in \lang{A}$, then there exists a path $v \in
\pathcat{G}(\setof{1_h}, T)$ with $\alpha(v) = w$. For $w = x_1 \cdots x_n$
then obviously holds $\mu(x_1 \cdots x_n) \in T$ thus $w \in \mu^{-1}(T) = L$ from
which the claim follows.
\end{proof}

From both lemmata immediately follows

\begin{theorem}[Regular sets of free monoid are exactly the recognizable sets]
\[ \reglang(X^*) = \reclang(X^*) \]
\end{theorem}

In the first section of this chapter we showed closure properties for regular
sets. We show now closure under inverse homomorphism which is rather easy using
our last theorem.

\begin{theorem}[Recognizable sets are closed under inverse homomorphism]
Let $\phi : X^* \to Y^*$ a monoid homomorphism, $L \in \reclang(Y^*) \Rightarrow
\phi^{-1}(L) \in \reclang(X^*)$.
\end{theorem}

\begin{proof}
Let $\phi : X^* \to Y^*$ and $L$ should be defined by monoid homomorphism $\mu'
: Y^* \to H$ as $L = \mu'^{-1}(T)$ for a subset $T \subset H$.

We set $\mu = \phi \circ \mu'$ and have $\mu : X^* \to H$ is a monoid
homomorphism and $\phi^{-1}(L) = \mu^{-1}(T)$ from which follows $\phi^{-1}(L)
\in \reclang(X^*)$.
\end{proof}

Exercise: For the integer numbers $\mathbb{Z}$ the set of recognizable sets
$\reclang(\mathbb{Z})$ shall be defined in analogy to $\reclang(X^*)$.

Characterize $\reclang(\mathbb{Z})$! Are there nonempty, infinite sets in
$\reclang(\mathbb{Z})$?
