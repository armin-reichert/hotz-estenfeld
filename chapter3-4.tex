\section{The finite automaton with storage}

In the following we slightly modify the notion of a pushdown-store by admitting
arbitrary monoids instead of the free monoid $Y^*$.

Doing so, the given definition of the pushdown-store cannot be used any longer
because for arbitrary monoids the notion of a ''topmost'' factor cannot be
defined in a meaningful way.

In our finite automaton with storage the control unit puts symbols onto the
store without caring about its current content which gives an even stronger
separation between control unit and pushdown-store as we had for pushwon
automata.

With the finite automaton with storage we have a universal model of automata.

\begin{definition}
$\fa{P} = (\fa{A}, \storage{P})$ is called {\bf finite automaton with storage}
$\iff$
\begin{eqnarray*}
& \fa{A}=(G, X, S, F, \alpha),\ (G=(V,E),\text{ is a finite automaton} \\
& \fa{P}=(E, M, D, C, \gamma)\text{ is a storage with monoid }M \\
& \gamma: E^* \to M\text{ is a monoidhomomorphism} \\
& D,C \subset M\text{ ($D$=domain, $C$= codomain)}
\end{eqnarray*}
\end{definition}

\[ L_{\fa{P}} = \setof{\alpha(\pi) \mid \pi \in \pathcat{G}(S, F),\ D\cdot
\gamma(\pi) \cap C \neq \emptyset}
\]
is the {\bf language accepted by} $\fa{P}$.

Remark: In the following we will also just write ''finite automaton with storage
monoid $M$'' if this does not lead to any misunderstandings.

\bigskip
The following theorem gives a relation between the pushdown automaton and the
finite automaton with storage.

\begin{theorem}
If $\kappa = (\fa{A}, \storage{K})$ is a pushdown automaton, then there exists a
finite automaton with storage $\fa{P}$ with storage monoid $\hgroup{Y}$ such
that $L_{\kappa} = L_{\fa{P}}$.
\end{theorem}

\begin{proof}
We remember the proof of the Chomsky-Schützenberger theorem in the previous
section.

Let
\[ \fa{A}'=(G', \hgroup{Y}, S_0, F, \gamma),\ G'=(V',E') \]
be the finite automaton used in the proof with accepted language
\[ L_\kappa = \setof{\alpha(\pi) \mid \pi \in \pathcat{G'}(S_0,F),\
|\gamma(\pi)|=y_0} \]

We set
\[ M := \hgroup{Y},\qquad E' \to \hgroup{Y} \]

Then for the automaton with storage $\fa{P}=(\fa{A}, \storage{P})$ with 
\[ \storage{P} = (E', \hgroup{Y}, \setof{\epsilon}, \setof{y_0}, \gamma) \]
it holds $L_\kappa = L_{\fa{P}}$.
\end{proof}

\bigskip
It is now possible to prove similar closure properties for our new type of
automaton as we did before for pushdown automata, in the corresponding
constructions one has to complement these for the used monoids.

We do not want to execute these constructions here.

\bigskip
\begin{theorem}[]
Let $\fa{P}=(\fa{A}, \storage{P})$ be a finite automaton with storage monoid
$\pocymon{Y},\ C = D = \setof{\epsilon}$. Then there exists a pushdown automaton
$\kappa = (\fa{A}, \storage{K})$ with $L_{\fa{P}} = L_\kappa$.
\end{theorem}

\begin{proof}
Let $\fa{P}=(\fa{A}, \storage{P})$ with
\[ \fa{A}=(G, X, S, F, \alpha),\ G=(V,E),\quad \storage{P}=(E, \pocymon{Y},
\setof{\epsilon}, \setof{\epsilon}, \gamma) \]

Without loss of generality let $\gamma(e) \in Y \cup Y^{-1} \cup
\setof{\epsilon}$ (this can be achieved by breaking the graph edges).

We define the pushdown automaton $\kappa = (\fa{A}, \storage{K})$ as follows:
\[ \storage{K} = (E, Y_1^*, \delta)\text{ with }Y_1 = Y \cup \setof{0} \]

The pushdown computation $\delta$ is 
\[ \delta: E \times Y_1 \to (Y_1 \times Y_1) \cup Y_1 \cup \setof{\epsilon} \]
defined by
\[ \delta(e, y) = \begin{cases}
	|y \cdot \gamma(e)| & \text{if }\gamma(e) \in Y \cup \setof{\epsilon}\text{ or
	} \gamma(e) = y^{-1} \\
	y \cdot 0 & \text{else}
\end{cases} \]

Claim: $L_\kappa \stackrel{!}{=} L_{\fa{P}}$

''$\supset$'': Let $w \in L_{\fa{P}}$, then there exists a path $\pi \in
\pathcat{G}(S, F)$ with label $\alpha(\pi) = w$ and storage computation
$\gamma(\pi) = \epsilon$ and for each prefix path $\pi'$ of $\pi$ holds:
\[ |\gamma(\pi')| = \delta(\pi', \epsilon) \]
This is easily proved by induction and left to the reader.

Therefore $w \in L_\kappa$.

\bigskip
''$\subset$'': Let $w \in L_\kappa$, then there exists a path $\pi \in
\pathcat{G(S, F)}$ with label $\alpha(\pi)=w$ and pushdown computation
$\delta(\pi, \epsilon) = \epsilon$.

Wit part 1 the claim holds again from which follows $w \in L_{\fa{P}}$.

From both parts follows the claim.
\end{proof}

\bigskip
Remark: The theorem also holds if we use the H-group $\hgroup{X}$ as storage
monoid (exercise).

From theorem 1 and theorem 2 it follows the equivalence of the automata models
with storage monoid $\hgroup{X}$.

\bigskip
As an example we give a finite automaton with storage which accepts the language
$L$ over the alphabet $X = \setof{a, b, c}$ defined as follows:
\begin{enumerate}
  \item $d \in L$
  \item $x, y \in L \Rightarrow a\cdot x\cdot b\cdot y\cdot c\in L$
  \item $L$ is minimal with 1) and 2)
\end{enumerate}

$L$ can be understood as the language of the bracketed arithmetic expressions.

We create the following preliminary (infinite) graph $G$ for $L$:

\missingfigure

We want to remove each edge labeled with $[G]$ by adding an $\epsilon$-edge
from the source of this edge to $S$ and from $F$ to the target of this edge.

We get the following graph:

\missingfigure

The graph $G'=(V',E')$ is finite but the corresponding finite acceptor would
also accept words which are not elements of $L$. This has to be corrected by defining a
suitable storage monoid.

Define $\gamma: E' \to \pocymon{\setof{a, b, s, t}}$ by
\begin{eqnarray*}
\gamma(e_a) & = & a \\
\gamma(e) & = & s \\
\gamma(e_d) & = & \epsilon \\
\gamma(t) & = & t \\
\gamma(e_b) & = & a^{-1} \cdot b \\
\gamma(e_c) & = & b^{-1} \\
\gamma(e'_d) & = & \epsilon \\
\gamma(e''_d) & = & \epsilon \\
\gamma(e^{-1}) & = & s^{-1} \\
\gamma(t^{-1}) & = & t^{-1} \\
\end{eqnarray*}

Domain and codomain contain the empty word only, the storage monoid $M$ is the
H-group over $X$:
\[ D = C = \setof{\epsilon},\quad M = \hgroup{\setof{a, b, s, t}} \]

For
\[ \fa{A} = (G', \setof{a, b,c d}, \setof{S}, \setof{F}, \alpha) \]
and storage 
\[ \storage{P} = (E', M, D, C, \gamma) \]
the finite automaton with storage $\fa{P} = (\fa{A}, \storage{P})$ accepts L:
\[ L_{\fa{P}} = \setof{\alpha(\pi) \mid \pi \in \pathcat{G'}(S, F),\ D \cdot
\gamma(\pi) \cap C \neq \emptyset} = L \]

Proof: exercise.

Now we want to remove the $\epsilon$-edges from the graph $G'$. To do so, we use
the algorithm from chapter II, lemma 2 and get:

\missingfigure

The storage is given by

\missingfigure

We can simplify the graph even further and get

\missingfigure

This graph is deterministic and the storage has the following form:
\[ \gamma(e_a) = a,\ \gamma(e_d) = \epsilon,\ \gamma(e_b) = a^{-1} b,\
\gamma(e_c) = b^{-1} \]

\bigskip
We want to look at an example motivated by practical applications: we define
parts of the syntax of the PASCAL programming language using finite automata
with storage.

Every programming language is based on some vocabular. Programs are constructed
by linear composition of these basic symbols in a way defined by the syntax
rules.

The syntax rules are formulated such that is is easy to verify if a sequence of
basic symbols is legal.

The syntax rules are most often (with advantage over other ways) represented by
flow diagrams which are called {\em syntax diagrams}.

The possible paths in these syntax diagrams represent all possible symbol
sequences.

Starting with a diagram labeled ''program'', along the path one either switches
to another diagram if a rectangle is met, one appends a basic symbol $w$ to the
program text if a circle labelled with $w$ is met. A complete example for this
kind of syntac definition can be found in \cite{Wirth}.

We will present syntax diagrams slightly different. We use kind of a dual graph
by making the vertices of the syntax diagrams to edges of our graphs.

In the following example we always show first the syntax diagram as given in
\cite{Wirth} and afterwards show the graph of our finite automaton with storage
that accepts the language defined by the syntax diagram.

\missingfigure

Remark: A program in PASCAL by this definition is a block followed by a dot.

\missingfigure

[Block] may be interpreted as a procedure call of a procedure ''Block''
(corresponds to the substitution of chapter III.2).

Storage:
\[ \gamma(e_1) = s,\quad \gamma(e_2) = s^{-1},\qquad D = C = \setof{\epsilon} \]

In this very simple example no difficulties arise. We want to consider now a
diagram in which the same diagram's name occurs again. This corresponds to a
recursive procedure call. We have to copy the diagram at this location.

\missingfigure

This is the syntax diagram which describes a ''block'' in PASCAL. It consists of
a declaration section and a statement section where the declaration section may
be empty.

$G_{\text{block}}$ is shown below:

\missingfigure

For simplicity the nonterminal symbols are shortened to single characters.

$e$ and $e^{-1}$ are edges inverse to each other. In the storage we have to
take care that the nesting depth of the block is correct. The edge $e$ takes
care that in a function or procedure-declaration the body appears correctly and
in case of multiple nesting the end of the block is always correct.

In the way we described now every syntax diagram can be tranlated to a finite
automaton with storage.

For each edge whose label is enclosed by the brackets [ ] there exists a finite
automaton with storage accepting the language associated with the edge.

By substitution of these graphs into the original graph we can then produce
graphs without bracket labels.

With this construction we could construct a finite automaton wit storage for the
complete syntax of the PASCAL programming language that accepts the
syntactically correct PASCAL programs. We get exactly the syntactically correct
programs as the labels of paths in our graph which additionally fulfill the
codomain-condition of the storage.

By the insertion of $\epsilon$-edges when resolving the recursion the automaton
becomes non-deterministic.

In the next section we want to transform the non-deterministic finite automaton
with storage into a deterministic one.

\bigskip
\begin{exercise}
Given the alphabet $X_k = \setof{a_1, \ldots, a_K}$ and the language $L_k :=
\setof{a_1, \ldots, a_k}^{4^k}$.

Define a finite automaton with storage accepting $L_k$ and prove its
correctness.
\end{exercise}

\bigskip
\begin{exercise}
Let $X = \setof{(, ), [, ], \$, c},\ d \notin X$.
\begin{eqnarray*}
L_0 &=& \setof{\epsilon} \\
&\cup& \{x_1 c y_1 c z_1 d \cdots d x_n c y_n c
z_n d \mid n \geq 1,\ y_1 \cdots y_n \in \$ D_2 \\
& & x_i, z_i \in X^*,\ i=1, \ldots, n \} 
\end{eqnarray*}
is called the {\em Greibach language}. ($D_2$ denotes the Dynck language over
\setof{(, ), [, ]}).

Show: There exists a finite automaton with storage accepting $L_0$.
\end{exercise}















































