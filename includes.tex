\usepackage[utf8]{inputenc}
\usepackage{hyperref}
\usepackage{parskip}
\usepackage{rotating}
\usepackage{tikz}
\usetikzlibrary{arrows,cd}

% TikZiT preamble begin
%\usepackage[svgnames]{xcolor}
\usetikzlibrary{decorations.markings}
\usetikzlibrary{shapes.geometric}
\usetikzlibrary{shadows}
\pgfdeclarelayer{edgelayer}
\pgfdeclarelayer{nodelayer}
\pgfsetlayers{edgelayer,nodelayer,main}
\newlength{\imagewidth}
\newlength{\imagescale}
\tikzstyle{none}=[inner sep=0pt]
\tikzstyle{filled_vertex}=[circle,fill=black!50,minimum size=0.5cm] 
\tikzstyle{state}=[circle,fill=white,draw=black]
\tikzstyle{oval_state}=[ellipse,fill=white,draw=black]
\tikzstyle{transition}=[-latex,draw=black]
\tikzstyle{path}=[-latex,draw=gray]

% theorem environments
\newtheorem{corollary}{Corollary}[section]
\newtheorem{definition}{Definition}[section]
\newtheorem{example}{Example}[section]
\newtheorem{lemma}{Lemma}[section]
\newtheorem{maintheorem}{Main Theorem}[section]
\newtheorem{theorem}{Theorem}[section]

% custom commands
\newcommand{\card}[1]{\mathrm{card}(#1)}
\newcommand{\category}[1]{\mbox{$\mathfrak{#1}$}}
\newcommand{\dderives}[1]{\underset{#1}{\Rightarrow}}
\newcommand{\derives}[1]{\underset{#1}{\overset{*}{\Rightarrow}}}
\newcommand{\edge}[1]{\xrightarrow{\ {#1}\ }}
\newcommand{\faequiv}[1]{\underset{\fa{#1}}{\equiv}}
\newcommand{\fa}[1]{\mbox{$\mathfrak{#1}$}}
\newcommand{\generatedbyone}[1]{\mbox{${<}#1{>}$}}
\newcommand{\generatedby}[2]{\mbox{${<}#1,#2{>}$}}
\newcommand{\hgroup}[1]{\mbox{${#1}^{[*]}$}}
\newcommand{\lang}[1]{\mbox{$L_{\mathfrak{#1}}$}}
\newcommand{\pathcat}[1]{\mbox{$\mathfrak{W}(#1)$}}
\newcommand{\pdstore}[1]{\mbox{$\mathcal{#1}$}}
\newcommand{\pocymon}[1]{\mbox{${#1}^{(*)}$}}
\newcommand{\powset}[1]{\mathfrak{P}(#1)}
\newcommand{\production}[1]{\xrightarrow{#1}}
\newcommand{\reglang}{\mbox{{\bf REG}}}
\newcommand{\setof}[1]{\mbox{$\{#1\}$}}
\newcommand{\transrem}[1]{%
\begin{center}%
\fboxrule0.1mm\fboxsep0.5em\fbox{\parbox{12cm}{Translator's remark: #1}}%
\end{center}}
\newcommand{\unioninv}[1]{#1 \cup {#1}^{-1}}
