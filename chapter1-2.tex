\section{Monoid homomorphisms and congruence relations}

\begin{definition}
A {\bf monoid homomorphism} (short: homomorphism) from a monoid $M$ to a monoid
$S$ is a mapping $\phi : M \to S$ with:
\begin{enumerate}
  \item $\phi(m_1 \cdot m_2) = \phi(m_1) \cdot \phi(m_2), \quad m_1, m_2 \in M$
  \item $\phi(1_M) = 1_S$
\end{enumerate}
\index{monoid homomorphism}
\end{definition}

It can easily be shown: if $U \subset M$ is a submonoid of $M$, then
$\phi(U)$ is a submonoid of the target monoid $S$. If $V$ is a submonoid of $S$,
then $\phi^{-1}(V)$ is a submonoid of $M$.

A monoid homomorphism $\phi: M \to S$ is called
\begin{eqnarray*}
\text{\bf monomorphism} & & \text{ if $\phi$ is injective} \\
\text{\bf epimorphism} & & \text{ if $\phi$ is surjective} \\
\text{\bf isomorphism} & & \text{ if $\phi$ is bijective} \\
\text{\bf endomorphism} & & \text{ if $\phi: M\to M$} \\
\text{\bf automorphism} & & \text{ if $\phi$ is an endomorphism and isomorphism}
\end{eqnarray*}
Monoids $M$ and $S$ are called {\bf isomorphic}, if there exists an
isomorphism between $M$ and $S$.

Of course, a homomorphism cannot be defined arbitrarily on a monoid $M$.
Thus the following two questions arise:
\begin{enumerate}
  \item If $U \subset M$ is a submonoid and $\phi_1 : U \to S$ is an arbitrary
mapping. In which case is it possible to extend $\phi_1$ to a homomorphism $\phi
: M \to S$\ ?
	\item If $\phi_1, \phi_2$ both are homomorphisms from $M$ to $S$ which
	coincide on a submonoid $U \subset M$. In which way can $\phi_1$ and $\phi_2$
	be different? 
\end{enumerate}

The answers to these questions of course depend on the structure of $U$. If
$U = \{ 1_M \}$ then $\phi$ is determined uniquely on $U$ but there is
little information on the relation between $\phi_1$ and $\phi_2$.

The following two simple theorems which can be found in introductory algebra
texts hold:
\begin{enumerate}
  \item If $A$ is a generating system of $M$ and $\phi_1, \phi_2 : M \to S$ are
  monoid homomorphisms which coincide on $A$, then $\phi_1 = \phi_2$.
  \item If $A$ is a set, $M = A^*$, and $\phi_1 : A \to S$ is an arbitrary
  mapping, then there exists a unique continuation $\phi$ of $\phi_1$ which is
  a monoid homomorphism from $A^*$ to $S$.
\end{enumerate}

\begin{definition}
A subset $A \subset M$ is called a {\bf free generating system} of $M$, if each
mapping $\phi_1 : A \to S$, where $S$ is an arbitrary monoid, can be continued
to a monoid homomorphim in a unique way. 
\end{definition}

A monoid with a free generating system is called a {\bf free monoid}.

$A^*$ therefore is a free monoid and $A$ is a free generating system of $A^*$.

It also holds: If $A$ is a free generating system of $M$ and $A^*$ is the
monoid of words (strings) over $A$, then $A^*$ and $M$ are isomorphic.

A free monoid has at most one free generating system (exercise). From this fact
one can see that the length $|w|$ of a word $w \in A^*$ can be defined in a unique way for any
free monoid.

The length function $L$ is an example for a monoid homomorphism $L : A^* \to
\mathbb{N}$ into the monoid of natural numbers.

If $\phi : M \to S$ is a monoid homomorphism, then the sets 
\[ \setof{\phi^{-1}(s) \mid s \in S} \subset \powset{M} \]
form a monoid which is isomorphic to $\phi(M)$ (exercise).

We want to investigate now the following problem:

Let $M$ be a monoid, $L \subset M$ be any subset of $M$. Does there exist a
monoid $S$ and a homomorphism $\phi : M \to S$ with the following property:
There exists an $s \in S$ with $L \subset \phi^{-1}(s)$?

Of course, there always exists such an $S$: Choose $S = \{ 1 \}$ and $\phi(M) =
\{1\}$. Therefore we strengthen our task: 

Find $S$ and $\phi$ such that $L \subset \phi^{-1}(S)$ and for each other
homomorphism $\Psi$ with that property holds:
\[ L \subset \Psi^{-1}(S') \Rightarrow \phi^{-1}(S) \subset \Psi^{-1}(S') \]

We want to describe $L$ as close as possible by a monoid homomorphism.

Such an $S$ and $\phi$ exists for each $L \subset M$ (see any algebra text), it
is denoted by $synt_M(L)$ and constructed as follows:

\begin{definition}
Let $M$ be a monoid and $L \subset M$. For $a, b \in M$ we define
\[ a \equiv b (L) \iff \text{for all }u, v \in M: u \cdot a \cdot v \in L
\Leftrightarrow u \cdot b \cdot v \in L \]
\end{definition}

The equivalence relation $\equiv (L)$ is a {\bf congruence relation}, it holds:
\begin{enumerate}
  \item Let $[a]_L = \setof{ b \in M \mid a \equiv b\ (L) }$\ then\ $b \in
  [a]_L \Rightarrow [a]_L = [b]_L $
  \item If we define $[a]_L \cdot [b]_L := [a \cdot b]_L$\ (complex product),
  then \[synt_M(L) := \setof{[a]_L \mid a \in M} \] becomes a monoid under
  this operation and the mapping \[\phi_L : M \to synt_M(L),\ \phi_L(a) = [a]_L\]
   is a monoid epimorphism.
\end{enumerate}

We call $\equiv (L)$ the {\bf syntactic congruence} of $L$ and $synt_M(L)$ the
{\bf syntactic monoid} of $L$ with respect to $M$.

To motivate the name {\em syntactic monoid} we give an example.

Let $A$ be the alphabet for English and $L$ the set of English sentences.
One can consider two words $w_1$ and $w_2$ from $L$ as {\em congruent} if they
can be exchanged in each sentence. There exist words that cannot always be exchanged. 
In the sentence ''An apple is a fruit'' the word ''apple'' can always be
exchanged by ''pear'' but this is not possible in the sentence ''apple is
spelled a p p l e''.

The difficulty is of {\em semantic} nature. If you don't consider semantic
correctness of sentences you get a classification of words wrt.\ their syntactic
meaning.

The important notion of {\em syntactic congruence} was introduced by M.\
P.\ Schützen\-berger in the context of coding problems.

{\bf Exercise:} Prove the following theorem:
\begin{theorem}
It is decidable if a monoid homomorphism $\phi$ is injective.
\end{theorem}

Hint: Let $\phi: A^* \to B^*$ be a given monoid homomorphism, $A =
\{a_1,\ldots,a_n\},\ E = \{\phi(a_1),\ldots,\phi(a_n)\}$.

Consider
\begin{eqnarray*}
A_1 &:=& \setof{u^{-1} v \in B^* \mid u, v \in E,\ u\neq v }\text{ and then
inductively}\\
A_{k+1} &:=& A_k^{-1} \cdot E \cup E \cdot A_k
\end{eqnarray*}

Prove that
\begin{enumerate}
  \item The construction of the $A_k$ stops
  \item $1 \in \bigcup_{k \geq 1} A_k \iff \phi\text{ is not injective}$.
\end{enumerate}


\begin{corollary}
For a free monoid $M$ it is decidable if some finite subset is a free generating
system of $M$.
\end{corollary}
